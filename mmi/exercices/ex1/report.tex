\documentclass[a4paper,11pt]{report}

\usepackage{amsmath}
\usepackage{fullpage}

\author{Sylvain Julmy}
\date{\today}

\setlength{\parindent}{0pt}

\begin{document}

\begin{center}
  \Large{
    Mathematical Methods for Computer Science 1\\
    Fall 2017
  }
  \noindent\makebox[\linewidth]{\rule{\linewidth}{0.4pt}}

  Series 1
  \vspace*{1.4cm}

  Sylvain Julmy
  
  \noindent\makebox[\linewidth]{\rule{\linewidth}{0.4pt}}
\end{center}

\section*{\texttt{1}}
\subsection*{a)}

The number of permutation $f$ of the set $s = \{1,2,3,4,5\}$ where $f(1) \neq 1$
is computed by using the formula on the number of permutation of a set :
$NbPerm(s) = |s|!$ where $f(1) \neq 1$, so we modify the formula :
$NbPerm'(s) = (|s|-1) * ((|s|-1)!)$.

Finally we can compute $NbPerm'(s) = 4 * (4!) = 96.$

\subsection*{b)}

The number of 10-digit numbers that have at least two equal digit can be
computated by substracting the number of digit which have no duplicated digit,
noted $N_{noDuplicate}$, to the total number of 10-digit numbers, noted
$N_{tot}$.

Normally, the total number of a n-digit number could be compute by using the
function $N_{tot}(n) = 10^n$, but a number can't start with $0$ sowe have to
slightly modify to formula : $N_{tot}(n) = 9 *
10^{n-1}$ where $n \ge 0$.

Now we can compute $N_{tot}(10) = 9 * 10^9$.

Then, the number of 10-digit number that have no duplicate's digit is the same
has the number of permutation $f$ of the set $s = \{0,1,2,3,4,5,6,7,8,9\}$ where
$f(0) \neq 0$. So we take the formula $NbPerm'(s) = (|s|-1) * ((|s|-1)!)$ and
we compute $N_{noDuplicate} = NbPerm'(10) = 9 * (9!)$.

Finally, we can compute the total number of 10-digit number that have at least
two equal digits : $N_{withDuplicate} = N_{tot} - N_{noDuplicate} = 9 * 10^9 - 9
* (9!) = 8'996'734'080$.

\section*{\texttt{2}}
\subsection*{a}

From any square on the chessboard, a rook is threatening $7*2 = 14$ another
square. The total number of square on a chessboard is $8*8 = 64$ so there is
$64-14 = 50$ free square for the other rook. So we have two choice to made : put
the first rook on any square ($64$ possibility) and put the second rook on a
non-threatened square ($64 - 14 = 50$ possibility). Finally the number of
position that satisfy the constraint are $64 * 50 = 3200$.

\subsection*{b}

Putting two rooks of the same color on a chessboards is the same as computing
$\begin{pmatrix}64 \\ 2\end{pmatrix} = \frac{64!}{2! * 62!} = \frac{64 * 63}{2}
= 2016$.

The other way is to make a first choice from $64$ squares and a second choice of
$63$ squares. But because the two rooks are undistinguishable, we need to divide
the total by $2$ : $\frac{64*63}{2} = 2016$

\subsection*{c}

First, we place a rook on one of the $64$ squares of the chessboards and then we
choose a square to put the second, that gives two differents situation :
\begin{enumerate}
\item The second rook is places on a square already threatened by the other rook
\item The second rook is places on a square that it is not threatened by the other rook
\end{enumerate}

In the first situation, the total number of square controlled
$N_{ControlledSquare}$ by the two rooks is $14 + 14 - 7 - 1 = 20$. In the second
situation, the total number of square controlled by the two rooks is $14 + 14 -
2 = 26$, because two square are threatened twice.

If we consider the first situation only, we have $64 * 63 * (62 - 20) * (61 -
20) = 6943104$ situations that satisfies the constraints and in the second
situation we have $64 * 63 * (62 - 26) * (61 - 26) = 5080320$ situations that
satisfies the constraints.

The sum of the two gives the total number of position that satisfies the
constraints : $5'080'320 + 6'943'104 = 12'023'424$

\section*{\texttt{3}}
\subsection*{a}

The number of natural divisors of $60$ is $12$ ($10$ is $1$ and $60$ are not
counted). This number is equal to the number of possible sub-set of the
(multi-)set $\{2,2,3,5\}$, but we have to remove some sub-set due to the fact
that there is two $2$ in the (multi-)set : $2^4 - 4 = 16 - 4 = 12$. We remove
$4$ because $2$ and $2'$ are the same so any time $2$ could be replace by $2'$
we remove one from $2^4$.

\subsection*{b}

For any natural number $n$ such that $\not\exists k$ where $k^2 = n$, the number
of natural divisor of $n$ is even because each operand $o_1$ of the multiplication
that gives $n$ needs a second operand $o_2$ where $o_1 \neq o_2$. For example,
we take $n=12$ :

\begin{align*}
  &1 * 12 = 12
  &2 * 6 = 12
  &3 * 4 = 12
\end{align*}

For any natural number $n$ such that $\exists k$ where $k^2 = n$, the number of
natural divisor is increased by exactly one because $k * k = n$.

\section*{\texttt{4}}
\subsection*{a}

The total number of different words in the language is $4^3 = 4 * 4 * 4 = 64$.

\subsection*{b}

The total number of different words that start with a vowels is $4 * 5 * 3 * 4 *
2 = 480$.

The total number of different words that start with a consonants is $5 * 4 * 4 *
3 * 3 = 720$.

The total number of different words is $480 + 720 = 1200$.

\section*{\texttt{5}}
\subsection*{a}

The number of different ways to separate $2n$ persons into two teams is the same
as taking $n$ from $2n$ and dividing the whole by $2$ because a groups of $n$
have a symetric groups : $\begin{pmatrix}2n \\ n\end{pmatrix} * 2^{-1}$.

\subsection*{b}

First, we count the number of permutation of $n$ stone and then we divive the
whole by $n$ because the bracelets is circular, so circular permutation don't
count : $\frac{n!}{n}$.

\end{document}


