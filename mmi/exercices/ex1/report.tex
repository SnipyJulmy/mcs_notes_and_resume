\documentclass[a4paper,11pt]{report}

\usepackage{amsmath}

\author{Sylvain Julmy}
\date{\today}

\setlength{\parindent}{0pt}

\begin{document}

\begin{center}
  \Large{
    Mathematical Methods for Computer Science 1\\
    Fall 2017
  }
  \noindent\makebox[\linewidth]{\rule{\linewidth}{0.4pt}}
  \noindent\makebox[\linewidth]{\rule{\linewidth}{0.4pt}}
\end{center}

\section*{\texttt{1}}
\subsection*{a)}

The number of permutation $f$ of the set $s = \{1,2,3,4,5\}$ where $f(1) \neq 1$
is computed by using the formula on the number of permutation of a set :
$NbPerm(s) = |s|!$ where $f(1) \neq 1$, so we modify the formula :
$NbPerm'(s) = (|s|-1) * ((|s|-1)!)$.

Finally we can compute $NbPerm'(s) = 4 * (4!) = 96.$

\subsection*{b)}

The number of 10-digit numbers that have at least two equal digit can be
computated by substracting the number of digit which have no duplicated digit,
noted $N_{noDuplicate}$, to the total number of 10-digit numbers, noted
$N_{tot}$.

Normally, the total number of a n-digit number could be compute by using the
function $N_{tot}(n) = 10^n$, but a number can't start with $0$ sowe have to
slightly modify to formula : $N_{tot}(n) = 9 *
10^{n-1}$ where $n \ge 0$.

Now we can compute $N_{tot}(10) = 9 * 10^9$.

Then, the number of 10-digit number that have no duplicate's digit is the same
has the number of permutation $f$ of the set $s = \{0,1,2,3,4,5,6,7,8,9\}$ where
$f(0) \neq 0$. So we take the formula $NbPerm'(s) = (|s|-1) * ((|s|-1)!)$ and
we compute $N_{noDuplicate} = NbPerm'(10) = 9 * (9!)$.

Finally, we can compute the total number of 10-digit number that have at least
two equal digits : $N_{withDuplicate} = N_{tot} - N_{noDuplicate} = 9 * 10^9 - 9
* (9!) = 8'996'734'080$.

\section*{\texttt{2}}
\subsection*{a}

From any square on the chessboard, a rook is threatening $7*2 = 14$ another
square. The total number of square on a chessboard is $8*8 = 64$ so there is
$64-14 = 50$ free square for the other rook. So we have two choice to made : put
the first rook on any square ($64$ possibility) and put the second rook on a
non-threatened square ($64 - 14 = 50$ possibility). Finally the number of
position that satisfy the constraint are $64 * 50 = 3200$.

\subsection*{b}

Putting two rooks of the same color on a chessboards is the same as computing
$\begin{pmatrix}64 \\ 2\end{pmatrix} = \frac{64!}{2! * 62!} = \frac{64 * 63}{2}
= 2016$.

The other way is to make a first choice from $64$ squares and a second choice of
$63$ squares. But because the two rooks are undistinguishable, we need to divide
the total by $2$ : $\frac{64*63}{2} = 2016$

\subsection*{c}



\section*{\texttt{3}}
\subsection*{a}
\subsection*{b}

\section*{\texttt{4}}
\subsection*{a}
\subsection*{b}

\section*{\texttt{5}}
\subsection*{a}
\subsection*{b}

\end{document}


