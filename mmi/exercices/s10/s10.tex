\documentclass[a4paper,11pt]{report}

\usepackage{amsmath}
\usepackage{amsthm}
\usepackage{fullpage}
\usepackage{tikz}

\usetikzlibrary{graphs,graphs.standard}

\makeatletter
\pgfmathdeclarefunction{alpha}{1}{%
  \pgfmathint@{#1}%
  \edef\pgfmathresult{\pgffor@alpha{\pgfmathresult}}%
}

\usepackage{bussproofs}
\usepackage{mathpartir}
\usepackage{prooftrees}
\usepackage{color}
\usepackage{nd3}

\newcommand*{\contract}[2]{contraction of $#1$ with $#2$}

\newcommand*{\MP}{Modus Ponens }
\newcommand*{\MPr}[2]{\MP of \ref{#1} and \ref{#2}}

\author{Sylvain Julmy}
\date{\today}

\setlength{\parindent}{0pt}

\begin{document}

\begin{center}
  \Large{
    Mathematical Methods for Computer Science 1\
    Fall 2017
  }
  \noindent\makebox[\linewidth]{\rule{\linewidth}{0.4pt}}

  Series 10
  \vspace*{1.4cm}

  Sylvain Julmy
  
  \noindent\makebox[\linewidth]{\rule{\linewidth}{0.4pt}}
\end{center}

\section*{\texttt{1}}

\subsection*{\texttt{(a)}}

\begin{prooftree}
  \AxiomC{$p,q \vdash p$}
  \AxiomC{$p,q \vdash q$}
  \BinaryInfC{$p,q \vdash p \wedge q$}
  \UnaryInfC{$p \vdash q \to (p \wedge q)$}
  \UnaryInfC{$\vdash p \to (q \to (p \wedge q))$}
\end{prooftree}

$\vdash p \to (q \to (p \wedge q))$ is a tautology.

\subsection*{\texttt{(b)}}

\begin{prooftree}
  \AxiomC{$p \vdash q$}
  \UnaryInfC{$ p \vdash (p \to q), q$}
  \UnaryInfC{$ p \vdash (p \to q) \wedge q$}
  \AxiomC{$p,q \vdash q$}
  \UnaryInfC{$q \vdash (p \to q) , q$}
  \UnaryInfC{$q \vdash (p \to q) \vee q$}
  \BinaryInfC{$(p \vee q) \vdash (p \to q) \vee q$}
  \UnaryInfC{$\vdash (p \vee q) \to ((p \to q) \vee q)$}
\end{prooftree}

$\vdash (p \vee q) \to ((p \to q) \vee q)$ is falsifiable under the
interpretation

\[
  I = \{
  q \mapsto false,
  p \mapsto true
  \}
\]

\subsection*{\texttt{(c)}}

\begin{prooftree}
  \AxiomC{$q,p \vdash q,p$}
  \UnaryInfC{$ p \vdash p,q,\neg q$}
  \UnaryInfC{$ p \neg p \vdash q \neg q$}
  \UnaryInfC{$ p \vdash q,\neg p \to \neg q$}
  \UnaryInfC{$\vdash p \to q, \neg p \to \neg q$}
  \UnaryInfC{$ \vdash (p \to q) \vee (\neg p \to \neg q)$}
\end{prooftree}

$\vdash (p \to q) \to ((q \to \neg r) \to \neg p)$ is a tautology.

\section*{\texttt{2}}

\subsection*{\texttt{(a)}}

\begin{prooftree}
  \AxiomC{$ s,p \vdash r,p$}
  \AxiomC{$ s,q \vdash r,p$}
  \BinaryInfC{$ s,p \vee q \vdash r,p$}
  \AxiomC{$p \vdash r,p,q$}
  \AxiomC{$q \vdash r,p,q$}
  \BinaryInfC{$ p \vee q \vdash r,p,q$}
  \BinaryInfC{$ q \to s,p \vee q \vdash r,p$}
  \AxiomC{$r, p \vdash r, q$}
  \AxiomC{$r, q \vdash r, q$}
  \BinaryInfC{$ r, p \vee q \vdash r, q$}
  \AxiomC{$r, s , p \vdash r$}
  \AxiomC{$r, s , q \vdash r$}
  \BinaryInfC{$ r, s , p \vee q \vdash r$}
  \BinaryInfC{$ r, q \to s,p \vee q \vdash r$}
  \BinaryInfC{$ p \to r, q \to s,p \vee q \vdash r$}
  \UnaryInfC{$ p \to r, q \to s \vdash (p \vee q) \to r$}
  \UnaryInfC{$ p \to r \vdash (q \to s) \to ((p \vee q) \to r)$}
  \UnaryInfC{$\vdash (p \to r) \to ((q \to s) \to ((p \vee q) \to r))$}
\end{prooftree}

We use the following leaves to construct the CNF $s,q \vdash r,p$ all the other
leaves are axioms. So the CNF form of $(p \to r) \to ((q \to s) \to ((p \vee q)
\to r))$ is

\[
  (r \vee p \vee \neg s \vee \neg q)
\]

\subsection*{\texttt{(b)}}

\begin{prooftree}
  \AxiomC{$p, \vdash p$}
  \UnaryInfC{$p, \vdash p, p$}
  \AxiomC{$p, \neg r \vdash p$}
  \BinaryInfC{$p, p \to \neg r \vdash p$}
  \AxiomC{$p, q, \vdash p$}
  \AxiomC{$p, q, \vdash r$}
  \UnaryInfC{$p, q, \neg r \vdash$}
  \BinaryInfC{$p, q, p \to \neg r \vdash$}
  \BinaryInfC{$p, p \to q, p \to \neg r \vdash$}
  \UnaryInfC{$ p \to q, p \to \neg r \vdash \neg p$}
  \UnaryInfC{$ p \to q \vdash  (p \to \neg r) \to \neg p$}
  \UnaryInfC{$ \vdash (p \to q) \to ((p \to \neg r) \to \neg p)$}
\end{prooftree}

We use the following leaves to construct the CNF $p,q \vdash r$ all the other
leaves are axioms. So the CNF form of $(p \to q) \to ((p \to \neg r) \to \neg
p)$ is

\[
  \neg p \vee \neg q \vee r
\]

\newpage

\section*{\texttt{3}}

\subsection*{\texttt{$ p \leftrightarrow q \equiv (p \to q) \vee (q \to p)$}}

Using the truth table

\begin{tabular}{@{ }c@{ }@{ }c | c@{ }@{}c@{}@{ }c@{ }@{ }c@{ }@{ }c@{ }@{}c@{}@{ }c@{ }@{}c@{}@{ }c@{ }@{ }c@{ }@{ }c@{ }@{}c@{}@{ }c}
p & q &  & ( & p & $\to$ & q & ) & $\wedge$ & ( & q & $\to$ & p & ) & \\
\hline 
T & T &  &  & T & T & T &  & \textcolor{red}{T} &  & T & T & T &  & \\
T & F &  &  & T & F & F &  & \textcolor{red}{F} &  & F & T & T &  & \\
F & T &  &  & F & T & T &  & \textcolor{red}{F} &  & T & F & F &  & \\
F & F &  &  & F & T & F &  & \textcolor{red}{T} &  & F & T & F &  & \\
\end{tabular}

In order to falsify $p \leftrightarrow q$, we have to satisfy $p$
and falsify $q$, or satisfy $q$ and falsify $p$ :

\begin{prooftree}
  \AxiomC{$p, \Gamma \vdash q, \Delta$}
  \AxiomC{$q, \Gamma \vdash p, \Delta$}
  \BinaryInfC{$ \Gamma \vdash p \leftrightarrow q, \Delta$}
\end{prooftree}

In order to satisfy $p \leftrightarrow q$, we have to satisfy $p$
and satisfy $q$, or falsify $q$ and falsify $p$ :

\begin{prooftree}
  \AxiomC{$p,q,\Gamma \vdash \Delta$}
  \AxiomC{$\Gamma \vdash p,q,\Delta$}
  \BinaryInfC{$ p \leftrightarrow q, \Gamma \vdash \Delta$}
\end{prooftree}

\subsection*{\texttt{$ p \oplus q \equiv (p \wedge \neg q) \vee (\neg p \wedge
    q)$}}

Using the truth table

\begin{tabular}{@{ }c@{ }@{ }c | c@{ }@{}c@{}@{ }c@{ }@{ }c@{ }@{ }c@{ }@{ }c@{ }@{}c@{}@{ }c@{ }@{}c@{}@{ }c@{ }@{ }c@{ }@{ }c@{ }@{ }c@{ }@{}c@{}@{ }c}
p & q &  & ( & p & $\wedge$ & $\neg$ & q & ) & $\vee$ & ( & $\neg$ & p & $\wedge$ & q & ) & \\
\hline 
T & T &  &  & T & F & F & T &  & \textcolor{red}{F} &  & F & T & F & T &  & \\
T & F &  &  & T & T & T & F &  & \textcolor{red}{T} &  & F & T & F & F &  & \\
F & T &  &  & F & F & F & T &  & \textcolor{red}{T} &  & T & F & T & T &  & \\
F & F &  &  & F & F & T & F &  & \textcolor{red}{F} &  & T & F & F & F &  & \\
\end{tabular}

In order to falsify $p \leftrightarrow q$, we have to satisfy $p$
and satisfy $q$, or falsify $q$ and falsify $p$ :

\begin{prooftree}
  \AxiomC{$p,q, \Gamma \vdash \Delta$}
  \AxiomC{$ \Gamma \vdash p,q, \Delta$}
  \BinaryInfC{$ \Gamma \vdash p \oplus q, \Delta$}
\end{prooftree}

In order to satisfy $p \leftrightarrow q$, we have to satisfy $p$
and falsify $q$, or falsify $q$ and satisfy $p$ :

\begin{prooftree}
  \AxiomC{$p,\Gamma \vdash q,\Delta$}
  \AxiomC{$q,\Gamma \vdash p\Delta$}
  \BinaryInfC{$ p \oplus q, \Gamma \vdash \Delta$}
\end{prooftree}

\section*{\texttt{4}}

\subsection*{\texttt{(a)}}
In order to minimize the number of vertices in an expanded deduction tree, one
need to apply the inference rule that are not splitting the deduction tree in
multiple branches.

For example, the inference rule

\begin{prooftree}
  \AxiomC{$ A,\Gamma \vdash \Delta$}
  \AxiomC{$ B,\Gamma \vdash \Delta$}
  \BinaryInfC{$A \vee B, \Gamma \vdash \Delta$}
\end{prooftree}

creates two vertices from one. So one need to apply (when he have the choice),
at first, the inference rule that are not creating such split. The idea is to
split the tree in multiple branches only if we don't have the choice.

\subsection*{\texttt{(b)}}

For this, we only consider the following set of connectives : $C = \{\wedge, \vee,
\neg, \to, \leftrightarrow, \oplus\}$

The proof is by induction on the structure of the formula.

In the base case, the formula $\phi$ has only one connective :

\begin{itemize}
\item $p \wedge q$
  \begin{itemize}
  \item To satisfy the formula, we have to satisfy $p$ and satisfy $q$, so only one leaf/branch is created.
  \item To falsify the formula, we have to falsify $p$ or falsify $q$, so two leaves/branches are created.
  \end{itemize}
\item $p \vee q$ 
  \begin{itemize}
  \item To satisfy the formula, we have to satisfy $p$ or satisfy $q$, so two leaves/branches are created.
  \item To falsify the formula, we have to falsify $p$ and falsify $q$, so only one leaf/branch is created.
  \end{itemize}
\item $\neg p$ 
  \begin{itemize}
  \item To satisfy the formula, we have to falsify $p$, so only one leaf/branch is created.
  \item To falsify the formula, we have to satisfy $p$, so only one leaf/branch is created.
  \end{itemize}
\item $p \to q$ 
  \begin{itemize}
  \item To satisfy the formula, we have to falsify $p$ or satisfy $q$, so two leaves/branches are created.
  \item To falsify the formula, we have to satisfy $p$ and falsify $q$, so only one leaf/branch is created.
  \end{itemize}
\item $p \leftrightarrow q$ 
  \begin{itemize}
  \item To satisfy the formula, we have to satisfy or falsify $p$ and $q$ at the same time, so two leaves/branches are created.
  \item To falsify the formula, we have to either falsify $p$ and satisfy $q$ or satisfy $p$ and falsify $q$, so two leaves/branches are created.
  \end{itemize}
\item $p \oplus q$ 
  \begin{itemize}
  \item To satisfy the formula, we have to either falsify $p$ and satisfy $q$ or satisfy $p$ and falsify $q$, so two leaves/branches are created.
  \item To falsify the formula, we have to satisfy or falsify $p$ and $q$ at the same time, so two leaves/branches are created.
  \end{itemize}
\end{itemize}

For the induction step, at any point in the construction of the expanded
deduction tree, we have have one connective from $C$. In the worst case, at each
step in the tree we encounter either $\leftrightarrow$ or $\oplus$. For both of
them, the tree is splited into two branches both the left handside and the right
handside of the connective are keept in both branch. So, in the worst case, the
number of vertices at each height of the tree is doubled, so we would have $2^m$
leaves at the end, where $m$ is the number of connective in $\phi$.

\subsection*{\texttt{(c)}}

In order to have exactly $2^m$ leaves in a formula $\phi$, we use the connective
$\leftrightarrow$, and we construct the formula using the following function $f :
\mathbb{N} \to \Phi$, where $\Phi$ is a formula in propositional logic :

\[
  f(n) = \left\{\begin{array}{lr}
    x_i} \leftrightarrow x_j},& \text{for } n = 0 \\
    f(n-1) \leftrightarrow f(n-1),& \text{for } n > 0
     \end{array}\right\}
 \]

 where all the variables $x_i$ and $x_j$ are different in the formula $\Phi$.
 So, for example, the formula $\Phi$ constructed from $f(3)$ is the following :

 \begin{align*}
   f(2) &= f(1) \leftrightarrow f(1) \\
        &= (f(0) \leftrightarrow f(0)) \leftrightarrow (f(0) \leftrightarrow f(0)) \\
        &= ((x_0 \leftrightarrow x_1) \leftrightarrow (x_2 \leftrightarrow x_3)) \leftrightarrow ((x_4 \leftrightarrow x_5) \leftrightarrow (x_6 \leftrightarrow x_7)) \\
 \end{align*}

And the deduction tree for such of a formula has $2^m$ leaves, because the
connective $\leftrightarrow$ always split the deduction into two parts and
always keep both side of the formula, so at each step in the deduction tree, we
double the number of path from the root to a leaf.

\subsection*{\texttt{(d)}}

The formula $p \to q$ contains $1$ connective $\to$ and has $1$ leaves :

\begin{prooftree}
  \AxiomC{$p \vdash q$}
  \UnaryInfC{$\vdash p \to q$}
\end{prooftree}

More generally, a formula such $\vdash (A \to (B \to (C \to \cdots))) \to \lambda$ would
produce a tree with $m$ leaves, where $m$ is the number of connective $\to$ :

\begin{prooftree}
  \AxiomC{$\vdash A$}
  \AxiomC{$\vdash B$}
  \AxiomC{$\vdash C$}
  \AxiomC{$\vdash \cdots$}
  \BinaryInfC{$C \to \cdots \vdash$}
  \BinaryInfC{$B \to (C \to \cdots) \vdash $}
  \BinaryInfC{$A \to (B \to (C \to \cdots)) \vdash \lambda$}
  \UnaryInfC{$\vdash (A \to (B \to (C \to \cdots))) \to \lambda $}
\end{prooftree}

for example, with $5$ connective $\to$ :

\begin{prooftree}
  \AxiomC{$\vdash A$}
  \AxiomC{$\vdash B$}
  \AxiomC{$\vdash C$}
  \AxiomC{$\vdash D, \lambda$}
  \AxiomC{$E \vdash \lambda$}
  \BinaryInfC{$\vdash (D \to E),  \lambda$}
  \BinaryInfC{$C \to (D \to E) \vdash \lambda$}
  \BinaryInfC{$B \to (C \to (D \to E)) \vdash \lambda$}
  \BinaryInfC{$A \to (B \to (C \to (D \to E))) \vdash \lambda$}
  \UnaryInfC{$\vdash (A \to (B \to (C \to (D \to E)))) \to \lambda$}
\end{prooftree}

\subsection*{\texttt{(e)}}

For a given formula, the number of leaves in the deduction tree is the same.

At any step of the construction of the deduction tree, when we have a choice, no
matter which formula we infer, we have, at a certain point, to infer the one we
didn't choose. So the variable would appear during the process. The point is
that separate sub-formula can't interact with each other. For example, we could
have the following step in our deduction tree :

\[
  \Gamma \vdash A \to B, C \to D, \Delta
\]

no matter if we infer $A \to B$ or $C \to D$, we have to infer each of them at
some point, and the formula $A \to B$ can't interact with the formula $C \to D$
and then modify the construction of the deduction tree.

So an expanded deduction tree is not unique, but the leaves would always be the
same no matter the construction process we use. So the number of leaves remains
the same.

\section*{\texttt{5}}

\subsection*{\texttt{(a)}}

If every satisfiable set of formula $\Gamma$ is consistent, then our proof system is
sound.

First, all of those formulas owns a common model, so, under this model, any
formula is a tautology. We assume that this model holds at any point in order to
prove the soundness of our proof systems.

Let $A_0,A_1,\dots,A_n$ be a formal proof in our system. We need to show that
$A_n$ is a tautology. $A_k$ is a tautology for all $k$, if $C_k$ is an axiom,
then we are done. Otherwise, $C_k = B$, $C_i = A$ and $C_j = A \to B$ where $i,j
< k$. By induction assumption, they are tautologies, then from $A$ and $A \to
B$, by Modus Ponens, we deduce $B$.

Then we assume $B$ is not a tautology, we found an assignement that falsify $B$
so this assignement satisfy $A$ and falsify $A \to B$, this is a contradiction
that $A \to B$  being a tautology.

\end{document}
