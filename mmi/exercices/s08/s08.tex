\documentclass[a4paper,11pt]{report}

\usepackage{amsmath}
\usepackage{fullpage}
\usepackage{tikz}

\usetikzlibrary{graphs,graphs.standard}

\makeatletter
\pgfmathdeclarefunction{alpha}{1}{%
  \pgfmathint@{#1}%
  \edef\pgfmathresult{\pgffor@alpha{\pgfmathresult}}%
}

\usepackage{bussproofs}
\usepackage{mathpartir}
\usepackage{prooftrees}
\usepackage{color}

\newcommand*{\contract}[2]{contraction of $#1$ with $#2$}

\author{Sylvain Julmy}
\date{\today}

\setlength{\parindent}{0pt}

\begin{document}

\begin{center}
  \Large{
    Mathematical Methods for Computer Science 1\
    Fall 2017
  }
  \noindent\makebox[\linewidth]{\rule{\linewidth}{0.4pt}}

  Series 7
  \vspace*{1.4cm}

  Sylvain Julmy
  
  \noindent\makebox[\linewidth]{\rule{\linewidth}{0.4pt}}
\end{center}

\section*{\texttt{1}}

\subsection*{(a)}

In order to show that $\{\rightarrow,\neg\}$ is complete, we have to found the
equivalent formula of $a \vee b$ and $a \wedge b$ ($\neg a$ and $a \rightarrow b$
are trivial...).

\subsubsection*{$a \vee b$}

\begin{gather*}
  (\neg a) \rightarrow b = (\neg \neg a) \vee b = a \vee b
\end{gather*}

and the truth table to validate the result :

\begin{center}
  \begin{tabular}{@{ }c@{ }@{ }c | c@{ }@{ }c@{ }@{ }c@{ }@{ }c@{ }@{ }c@{ }@{ }c}
    a & b &  & $\neg$ & a & $\rightarrow$ & b & \\
    \hline 
    T & T &  & F & T & \textcolor{red}{T} & T & \\
    T & F &  & F & T & \textcolor{red}{T} & F & \\
    F & T &  & T & F & \textcolor{red}{T} & T & \\
    F & F &  & T & F & \textcolor{red}{F} & F & \\
  \end{tabular}
\end{center}

\subsubsection*{$a \wedge b$}

\begin{align*}
  \neg (a \rightarrow \neg b) &= \neg (\neg a \vee \neg b) \\
                              & = \neg \neg (a \wedge b) \\
                              &= (a \wedge b)
\end{align*}

and the truth table to validate the result :

\begin{center}
  \begin{tabular}{@{ }c@{ }@{ }c | c@{ }@{}c@{}@{ }c@{ }@{ }c@{ }@{ }c@{ }@{ }c@{ }@{}c@{ }}
    a & b & $\neg$ & ( & a & $\rightarrow$ & $\neg$ & b & )\\
    \hline 
    T & T & \textcolor{red}{T} &  & T & F & F & T & \\
    T & F & \textcolor{red}{F} &  & T & T & T & F & \\
    F & T & \textcolor{red}{F} &  & F & T & F & T & \\
    F & F & \textcolor{red}{F} &  & F & T & T & F & \\
  \end{tabular}
\end{center}

\subsection*{(b)}

In order to show that $\{\uparrow\}$ is complete, we have to found the
equivalent formula of $a \rightarrow b$ and $\neg a$, because we just showed
that $\{\neg,\rightarrow\}$ is complete.

We know that $a \uparrow b \leftrightarrow \neg (a \wedge b)$.

\subsubsection*{$\neg a$}

\[
  a \uparrow a = \neg (a \wedge a) = \neg a
\]

and the truth table to validate the result :
\begin{center}
  \begin{tabular}{@{ }c | c@{ }@{}c@{}@{ }c@{ }@{ }c@{ }@{ }c@{ }@{}c@{ }}
    a & $\neg$ & ( & a & $\&$ & a & )\\
    \hline 
    T & \textcolor{red}{F} &  & T & T & T & \\
    F & \textcolor{red}{T} &  & F & F & F & \\
  \end{tabular}
\end{center}

\subsubsection*{$a \rightarrow b$}

\begin{align*}
  a \rightarrow b &= \neg a \vee b\\
                  &= \neg \neg (\neg a \vee b)\\
                  &= \neg (a \wedge \neg b)\\
                  &= \neg (a \wedge \underbrace{\neg (b \wedge b)}_{b \uparrow b})\\
                  &= \underbrace{\neg (a \wedge (b \uparrow b))}_{a \uparrow (b \uparrow b)}\\
                  &= a \uparrow (b \uparrow b)
\end{align*}

and the truth table to validate the result :

\begin{center}
  \begin{tabular}{@{ }c@{ }@{ }c | c@{ }@{ }c@{ }@{ }c@{ }@{}c@{}@{ }c@{ }@{ }c@{ }@{ }c@{ }@{}c@{}@{ }c}
    a & b &  & a & $\uparrow$ & ( & b & $\uparrow$ & b & ) & \\
    \hline 
    T & T &  & T & \textcolor{red}{T} &  & T & F & T &  & \\
    T & F &  & T & \textcolor{red}{F} &  & F & T & F &  & \\
    F & T &  & F & \textcolor{red}{T} &  & T & F & T &  & \\
    F & F &  & F & \textcolor{red}{T} &  & F & T & F &  & \\
  \end{tabular}
\end{center}

\section*{\texttt{2}}

\subsection*{(a)}

\subsection*{(b)}

\section*{\texttt{3}}

\subsection*{(a)}

\subsection*{(b)}

\subsection*{(c)}

\section*{\texttt{4}}

\subsection*{(a)}

\subsection*{(b)}

\subsection*{(c)}

\section*{\texttt{5}}

\subsection*{(a)}

\subsection*{(b)}

\section*{\texttt{6}}

\end{document}
