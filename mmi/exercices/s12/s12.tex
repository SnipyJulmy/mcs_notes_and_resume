\documentclass[a4paper,11pt]{report}

\usepackage{amsmath}
\usepackage{amsthm}
\usepackage{fullpage}
\usepackage{tikz}

\usetikzlibrary{graphs,graphs.standard}

\makeatletter
\pgfmathdeclarefunction{alpha}{1}{%
  \pgfmathint@{#1}%
  \edef\pgfmathresult{\pgffor@alpha{\pgfmathresult}}%
}

\usepackage{bussproofs}
\usepackage{mathpartir}
\usepackage{prooftrees}
\usepackage{color}
\usepackage{nd3}

\newcommand*{\contract}[2]{contraction of $#1$ with $#2$}

\newcommand*{\MP}{Modus Ponens }
\newcommand*{\MPr}[2]{\MP of \ref{#1} and \ref{#2}}

\author{Sylvain Julmy}
\date{\today}

\setlength{\parindent}{0pt}

\begin{document}

\begin{center}
  \Large{
    Mathematical Methods for Computer Science 1\
    Fall 2017
  }
  \noindent\makebox[\linewidth]{\rule{\linewidth}{0.4pt}}

  Series 12
  \vspace*{1.4cm}

  Sylvain Julmy
  
  \noindent\makebox[\linewidth]{\rule{\linewidth}{0.4pt}}
\end{center}

\section*{\texttt{1}}

\subsection*{\texttt{(a)}}

The formula

\[
  \phi = (\forall x.(P(x) \wedge (\exists y.(\forall z.(Q(y,z)))) )) \to (\forall x.(P(x) \vee (\exists y.(\forall z.(Q(y,z)))) ))
\]

is a special case of the formula $(q \wedge p) \to (q \vee p)$, so if $(q \wedge
p) \to (q \vee p)$ is valid in our proof system, then $\phi$ is valid too.

So we have to prove $\vdash (q \wedge p) \to (q \vee p)$, by deduction lemma, we
are going to prove $q \wedge p \vdash q \vee p$.

\begin{ND}[Proof][proof1a][][][0.6\linewidth]
  \ndl{}{$q \wedge p$}{premise \label{1}}
  \ndl{}{$(q \wedge p) \to q$}{Axiom 3a \label{2}}
  \ndl{}{$q$}{\MPr{1}{2} \label{3}}
  \ndl{}{$q \to (q \vee p)$}{Axiom 5b \label{4}}
  \ndl{}{$q \vee p$}{\MPr{3}{4} \label{5}}
\end{ND}

\subsection*{\texttt{(b)}}

The formula

\[
  \phi = Q(x,y) \to (\forall z.(P(z)) \to (Q(x,y) \to \forall z.(P(z))))
\]

is a special case of the formula $p \to (q \to (p \wedge q))$ and $p \to (q \to
(p \wedge q))$ is an axiom in our proof system, then $\phi$ is a valid formula.

\section*{\texttt{2}}

\subsection*{\texttt{(a)}}

\[
  \phi = (\forall x.(\exists y.(P(x,y)))) \vee (\forall x.(\exists y.(\neg P(x,y))))
\]

\begin{align*}
  & P(x,y) = \{(x,y) | x = y\} \\
  & U = \{0,1\}
\end{align*}

In order to falsify $\phi$, we have to falsify both side of the $\vee$ connective.

To falsify $\forall x.(\exists y.(P(x,y)))$, for any $x$, we can found an $y$
that is not equal to $x$, and then falsify the formula. For $x=1$, we pick $y=0$
and for $x=0$ we pick $y=1$.

To falsify $\forall x.(\exists y.(\neg P(x,y)))$, for any $x$, we can found an
$y$ that is equal to $x$, and then falsify the formula. For $x=1$, we pick $y=1$
and for $x=0$, we pick $y=0$.

We can't make $U$ smaller because :
\begin{itemize}
\item $|U| = 0$ is not possible, because $U \neq \emptyset$
\item $|U| = 1$ is not possible, because the predicate have to ``return''
  $false$ or $true$, and it is not possible to construct a predicate that is
  non-deterministic. For example, $P(1,1)$ can't ``return'' $true$ one time and
  $false$ on another time.
\end{itemize}

\subsection*{\texttt{(b)}}

\[
  \phi = \forall x.(\exists y.(P(x,y))) \to \exists x.( \forall y.(P(x,y)))
\]

\begin{align*}
  & P(x,y) = \{(x,y) | x = y\} \\
  & U = \{0,1\}
\end{align*}

In order to falsify $\phi$, we have to satisfy the left hand side and falsify
the right hand side of the $\to$ connective.

To satisfy $\forall x.(\exists y.(P(x,y)))$, we have to find, for any $x$, and
$y$ which ois equal to $x$. For $x=1$, we pick $y=1$ and for $x=0$, we pick
$y=0$.

To falsify $\exists x.( \forall y.(P(x,y)))$, we have to find an $x$, that for
all $y$, $x$ is not always equal to $y$. We pick $x=1$, then if $y=1$, the
formula is satisfy, but for $y=0$, the formula is not.

We can't make $U$ smaller because :
\begin{itemize}
\item $|U| = 0$ is not possible, because $U \neq \emptyset$
\item $|U| = 1$ is not possible, we use the same argument as before. The
  predicate $P(x,y)$ is deterministic. So if we have only one choice to fill
  $P$, with only one single value, then $P(\lambda,\lambda)$ would always
  ``return'' the same value, either $true$ or $false$. Then both left and right
  hand side of the $\to$ connective have the same value and $true \to true$ and
  $false \to false$ are valid.
\end{itemize}

\section*{\texttt{3}}

\subsection*{\texttt{(a)}}

We will prove $\vdash \neg \forall x.( \neg \phi) \to \exists x.(\phi)$, by
proving the contrapositive : $\vdash \neg \exists x.(\phi) \to \forall x.(\neg
\phi)$ and by deduction lemma, $\neg \exists x.(\phi) \vdash \forall x.(\neg
\phi)$.

\begin{ND}[Proof][proof3a][][][0.85\linewidth]
  \ndl{}{$\neg \exists x.(\phi)$}{Premise \label{1}}
  \ndl{}{$\phi \to \exists x.(\phi)$}{Axiom 11 \label{2}}
  \ndl{}{$\neg \exists x.(\phi) \to (\phi \to \neg \exists x.(\phi))$}{Axiom 1 \label{3}}
  \ndl{}{$\phi \to \neg \exists x.(\phi)$}{\MPr{1}{3} \label{4}}
  \ndl{}{$(\phi \to \exists x.(\phi)) \to ((\phi \to \neg \exists x.(\phi)) \to \neg \phi)$}{Axiom 8 \label{5}}
  \ndl{}{$(\phi \to \neg \exists x.(\phi)) \to \neg \phi$}{\MPr{2}{5} \label{6}}
  \ndl{}{$\neg \phi$}{\MPr{5}{6} \label{7}}
  \ndl{}{$\neg \phi \to \forall x.(\neg \phi)$}{Generalisation rule \label{8}}
  \ndl{}{$\forall x.(\neg \phi)$}{\MPr{7}{8} \label{9}}
\end{ND}

\subsection*{\texttt{(b)}}

We will prove $\vdash \neg \exists x.(\neg \phi) \to \forall x.(\phi)$, and by
deduction lemma, we prove $\neg \exists x.(\neg \phi) \vdash \forall x.(\phi)$

\begin{ND}[Proof][proof3b][][][0.95\linewidth]
  \ndl{}{$\neg \exists x.(\neg \phi) $}{Premise \label{1}}
  \ndl{}{$\neg \phi \to \exists x.(\neg \phi) $}{Axiom 11 \label{2}}
  \ndl{}{$\neg \exists x.(\neg \phi) \to (\neg \phi \to \neg \exists x.(\neg \phi))$}{Axiom 1 \label{3}}
  \ndl{}{$\neg \phi \to \neg \exists x.(\neg \phi) $}{\MPr{1}{3} \label{4}}
  \ndl{}{$ \neg \phi \to \exists x.(\neg \phi) \to ((\neg \phi \to \neg \exists
    x.(\neg \phi)) \to \neg \neg \phi)$}{Axiom 8 \label{5}}
  \ndl{}{$(\neg \phi \to \neg \exists x.(\neg \phi)) \to \neg \neg \phi$}{\MPr{2}{5} \label{6}}
  \ndl{}{$\neg \neg \phi$}{\MPr{4}{5} \label{7}}
  \ndl{}{$\phi$}{Double negation elimination \label{8}}
  \ndl{}{$\phi \to \forall x.(\phi)$}{Generalisation rule \label{9}}
  \ndl{}{$\forall x.(\phi)$}{\MPr{8}{9} \label{10}}
\end{ND}

\section*{\texttt{4, 5}}

Due to exam preparation, I didn't have that much time to do the rest.

Thanks anyway for correcting exerices 1,2 and 3 !

\end{document}
