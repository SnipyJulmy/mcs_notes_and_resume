\documentclass[a4paper,11pt]{report}

\usepackage{amsmath}
\usepackage{amsthm}
\usepackage{fullpage}
\usepackage{tikz}

\usetikzlibrary{graphs,graphs.standard}

\makeatletter
\pgfmathdeclarefunction{alpha}{1}{%
  \pgfmathint@{#1}%
  \edef\pgfmathresult{\pgffor@alpha{\pgfmathresult}}%
}

\usepackage{bussproofs}
\usepackage{mathpartir}
\usepackage{prooftrees}
\usepackage{color}
\usepackage{nd3}

\newcommand*{\contract}[2]{contraction of $#1$ with $#2$}

\newcommand*{\MP}{Modus Ponens }
\newcommand*{\MPr}[2]{\MP of \ref{#1} and \ref{#2}}

\author{Sylvain Julmy}
\date{\today}

\setlength{\parindent}{0pt}

\begin{document}

\begin{center}
  \Large{
    Mathematical Methods for Computer Science 1\
    Fall 2017
  }
  \noindent\makebox[\linewidth]{\rule{\linewidth}{0.4pt}}

  Series 11
  \vspace*{1.4cm}

  Sylvain Julmy
  
  \noindent\makebox[\linewidth]{\rule{\linewidth}{0.4pt}}
\end{center}

\section*{\texttt{1}}

\subsection*{\texttt{(a)}}
``Every ice hockey team has a goalkeeper''
\[
  \forall x.(\exists y. ( H(x) \wedge G(x,y) ))
\]

\subsection*{\texttt{(b)}}
``Nobody can be goalkeeper in two different teams''
\[
  \neg(\exists x.( G(x,y) \wedge G(x,z) \wedge \neg(z = y)))
\]

\subsection*{\texttt{(c)}}
``If Gottéron beats Berne, then Gottéron does not lose to every team''
\[
  B(f,b) \to \exists x.(\neg(L(f,x)))
\]

\subsection*{\texttt{(d)}}
``Gottéron beats some team, which beats Berne''
\[
  \exists x.(B(f,x) \wedge B(x,b))
\]

\section*{\texttt{2}}

\subsection*{\texttt{(a)}}

Specification :

\begin{align*}
  R(x) : && \text{the things $x$ is red} \\
  B(x) : && \text{the things $x$ is in the box}
\end{align*}

\subsubsection*{(i)}
\[
  \forall x.(R(x) \to B(x))
\]

\subsubsection*{(ii)}
\[
  \forall x.(B(x) \to R(x))
\]

\subsection*{\texttt{(b)}}

Specification :

\begin{align*}
  f(x) : && \text{function which return the father of $x$} \\
  a : && \text{constant that represent the person ``Arther''} \\
  g : && \text{constant that represent the person ``Gaspard''} \\
  s : && \text{constant that represent the person ``Ben'''}
\end{align*}


\subsubsection*{(i)}

``Ben is a grandfather''

\[
  \exists x.(f(f(x)) = s)
\]

\subsubsection*{(ii)}

``Arthur and Gaspard have the same father''

\[
  f(a) = f(g)
\]

\subsection*{\texttt{(c)}}

Specification :

\begin{align*}
  F(x,y) : && \text{predicate that is true if $x$ is the father of $y$} \\
  a : && \text{constant that represent the person ``Arther''} \\
  g : && \text{constant that represent the person ``Gaspard''} \\
  s : && \text{constant that represent the person ``Ben'''}
\end{align*}

\subsubsection*{(i)}

``Ben is a grandfather''

\[
  \exists x.(F(x,y) \wedge F(s,x))
\]

\subsubsection*{(ii)}

``Arthur and Gaspard have the same father''

\[
  \exists x.(F(x,a) \wedge F(x,g))
\]

\section*{\texttt{3}}

\subsection*{\texttt{(a)}}

\subsubsection*{(i)}
\[
  P(f(x,y))
\]

is not a valid formula in the predicate logic, because $f$ is a unary function
symbol, and, in the formula, $f$ is used as a binary function.

\subsubsection*{(ii)}
\[
  Q(m,f(m))
\]

is a valid formula in the predicate logic.

\subsubsection*{(iii)}
\[
  Q(Q(m,x),y)
\]

is not a valid formula in the predicate logic, because the predicate $Q$ has a
predicate in its parameters, which is not allowed. Predicate only allows terms
as parameters, and $Q$ is not a term.

\subsubsection*{(iv)}
\[
  Q(x,y) \to \exists x.(Q(z,y))
\]

is a valid formula in the predicate logic.


\subsection*{\texttt{(b)}}

In the formula

\[
  \exists x.(P(y,z) \wedge (\forall y.(\neg Q(y,x) \wedge P(y,z))))
\]

\begin{itemize}
\item $x$ is a bound variable.
\item $y$ is a free variable in the scope $P(y,z) \wedge (\forall y.(\neg Q(y,x) \wedge P(y,z)))$.
\item $y$ is a bound variable in the scope $\neg Q(y,x) \wedge P(y,z)$.
\item $z$ is a free variable.
\end{itemize}

\subsection*{\texttt{(c)}}

We change the free variable $y$ and $z$ to $v$ and $w$ and add universal
quantifier in front of the formula for both.

\[
  \forall v.(\forall w.( \exists x.(P(v,w) \wedge (\forall y.(\neg Q(y,x) \wedge P(y,w))))))
\]

\section*{\texttt{4}}

\subsection*{\texttt{(a)}}

The formula $\forall x.(\exists y.(P(x,y) \wedge (Q(y))))$ is satisfy by the
first interpretation. For any $x$, we can find an $y$ which is greater than $x$
and greater than $0$. If $x$ is greater or equals than $0$, then $x + 1$ is
greater than $0$ and greater than $x$, else, if $x$ is less than $0$, we can
pick $1$, which is greater than any negative number and greater than $0$. So the
formula is satisfiable by the first interpretation.

On the other hand, the second interpretation does not satisfy the formula,
because if we assign $x$ to a negative number, then we can't find an $y$ which
is lesser than $x$ and greater than $0$.

\subsection*{\texttt{(b)}}

The formula is satisfiable under the following interpretation :

\begin{align*}
  U &= \mathbb{Z} \\
  P &= \{(x,y) | x^2 = y\} \\
  Q &= \{x | x > 0\}
\end{align*}

but the formula is not satisfiable under the interpretation

\begin{align*}
  U &=  \{a,b,c\} && \text{} \\
  P(x,y) &= \{(a,b),(b,a)\} &&  \text{predicate that is true only if $y$ is blue and $x$ is yellow} \\
  Q(x)  &= \{a,b,c\} && \text{predicate that is true if $x$ is red}
\end{align*}

so the formula is not valid because we found an interpretation that falsify the
formula. We can't satisfy the formula $P(x,y)$ for all $y$, because $P(x,y)$ can't be
$true$ if $x$ or $y$ is $c$.

\section*{\texttt{5}}

The formula $\phi = \forall x.(\exists y.(P(x,y) \to Q(x,y)))$ is satisfiable
under the given interpretation. In order to make the formula $true$, we found
that $P(x,y)$ is false. To make that $\forall x.(\exists y.(P(x,y)))$ is
$false$, we found an $y$ for any $x$ that falsify $P$ :

\begin{itemize}
\item if $x$ is $a$, then we put $y$ is $c$, so $P$ is falsify
\item if $x$ is $b$, then we put $y$ is $a$, so $P$ is falsify
\item if $x$ is $c$, then we put $y$ is $a$, so $P$ is falsify
\end{itemize}

Then, $\phi$ is satisfiable.

The formula $\psi$ is not satisfiable, because we have to make it $true$ for any
$x$. For all $x$ means that we have to check for any assignement of $x$ with an
element of $U$, if $x$ is $c$, then whatever we pick for $y$, the predicate $P$
would be satisfy :

\begin{itemize}
\item $(c,a)$ is $true$, because $(c,a) \in Q$
\item $(c,b)$ is $true$, because $(c,b) \in Q$
\item $(c,c)$ is $true$, because $(c,c) \in Q$
\end{itemize}

So the first part of the implication is satisfy, but then, we can't satisfy
$P(x,y)$ because $(c,a) \not\in P$, $(c,b) \not\in P$ and $(c,c) \not\in P$.



\end{document}
