\documentclass[a4paper,11pt]{report}

\usepackage{amsmath}
\usepackage{fullpage}
\usepackage[cache=false]{minted}

\author{Sylvain Julmy}
\date{\today}

\setlength{\parindent}{0pt}

\newminted{prolog}{frame=single, framesep=6pt,
  breaklines=true,fontsize=\scriptsize,linenos}
\newmintedfile{prolog}{frame=single, framesep=6pt,
  breaklines=true,fontsize=\scriptsize}
\newmintinline{prolog}{}
\newcommand{\ex}[3]{\prologfile[firstline=#1,lastline=#2]{#3.pl}}

\begin{document}

\begin{center}
  \Large{
    Functional and Logic Programming\
    Fall 2017
  }
  
  \noindent\makebox[\linewidth]{\rule{\linewidth}{0.4pt}}
  S07 : Prolog (Introduction)
  \noindent\makebox[\linewidth]{\rule{\linewidth}{0.4pt}}

  \begin{flushleft}
    Professor : Le Peutrec Stephane
    
    Assistant : Lauper Jonathan
  \end{flushleft}

  \noindent\makebox[\linewidth]{\rule{\textwidth}{1pt}}
\end{center}

\section*{Exercise 1}

\subsection*{(\protect\Verb+=+)}

The predicat \prologinline|=/2|, is the unification operator. It unify the two
terms given in argument. Note : the standard unification algorithm does not
check occurences, use \prologinline|unify_with_occurs_check/2| for that.

\subsection*{(\protect\Verb+$\backslash$=+)}

Succeed if and only if both argument of \prologinline|\=/2| can't be unified.
For example, \prologinline|a(B) \= a(C).| is false because we could found that
$A=C$ or $C=A$.

\subsection*{(\protect\Verb+==+)}

Succeed if and only if both argument of \prologinline|==| are the same atom or
the same variable. For example, \prologinline|X == Y| is false and
\prologinline|X == X| is true.

\subsection*{(\protect\Verb+$\backslash$==+)}

Succeed if and only if both argument of \prologinline|\==/2| are not the same atom
or not the same variable. For example, \prologinline|X \== Y| is true and
\prologinline|X \== X| is false.

\subsection*{(\protect\Verb+=$\backslash$=+)}

Succeed if and only if both argument are instancied variable, number and both
number are different. For example, \prologinline|1 =\= 2| is true.

\subsection*{(\protect\Verb+=:=+)}

Succeed if and only if both argument are instancied variable, number and both
number are the same. For example, \prologinline|2 =:= 2| is true.

\subsection*{(\protect\Verb+is+)}

Unify the first argument with the second argument which must be an expression.
For example, \prologinline|1 is 3 - 2| is true and \prologinline|A is 5 - 3| is
true and $A$ is unified with $2$. Note that in the left expression, all variable
has to be instancied, so we can't write something like \prologinline|2 is 3 -
X.|, this cause a failure.

\ex{9}{35}{ex2}

\section*{Exercise 2}

\subsection*{\prologinline|child/2|}
\ex{11}{12}{ex3}

\subsection*{\prologinline|father/2|}
\ex{14}{17}{ex3}

\subsection*{\prologinline|mother/2|}
\ex{19}{22}{ex3}

\subsection*{\prologinline|grandParent/2|}
\ex{24}{27}{ex3}

\subsection*{\prologinline|grandFather/2|}
\ex{29}{32}{ex3}

\subsection*{\prologinline|uncle/2|}
\ex{34}{40}{ex3}

\subsection*{\prologinline|ancestor/2|}

\begin{prologcode}
% ancestor(?X,?Y), succeed if X is the ancestor of Y
ancestor(X,Y) :- parent(X,Y).
ancestor(X,Y) :-
    parent(X,SomeBody),
    ancestor(SomeBody,Y).
\end{prologcode}

Note : inverting the order of the two predicates \prologinline|ancestor/2| does
not affect its behaviour, because we call parent anyway and
\prologinline|parent/2| always unify both variable. On the other hand, if we
invert the $4th$ and the $5th$ line, we would have an infinite recursion and
then a stack overflow, because ancestor would call itself with variables again
and again.

\section*{Exercise 3}

\ex{9}{16}{ex4}

\end{document}

%%% Local Variables:
%%% TeX-command-extra-options: "-shell-escape"
%%% mode: latex
%%% End:
