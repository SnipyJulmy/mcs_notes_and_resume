\documentclass[a4paper,11pt]{report}

\usepackage{amsmath}
\usepackage{fullpage}
\usepackage[cache=false]{minted}

\author{Sylvain Julmy}
\date{\today}

\setlength{\parindent}{0pt}

\newmintedfile{prolog}{frame=single, framesep=6pt,
  breaklines=true,fontsize=\scriptsize}
\newmintinline{prolog}{}
\newcommand{\ex}[3]{\prologfile[firstline=#1,lastline=#2]{#3.pl}}

\begin{document}

\begin{center}
  \Large{
    Functional and Logic Programming\
    Fall 2017
  }
  
  \noindent\makebox[\linewidth]{\rule{\linewidth}{0.4pt}}
  S07 : Prolog (Introduction)
  \noindent\makebox[\linewidth]{\rule{\linewidth}{0.4pt}}

  \begin{flushleft}
    Professor : Le Peutrec Stephane
    
    Assistant : Lauper Jonathan
  \end{flushleft}

  \noindent\makebox[\linewidth]{\rule{\textwidth}{1pt}}
\end{center}

\section*{Exercise 1}

\subsection*{(\protect\Verb+=+)}

The predicat \prologinline|=/2|, is the unification operator. It unify the two
terms given in argument.

\subsection*{(\protect\Verb+$\backslash$=+)}

\subsection*{(\protect\Verb+==+)}

\subsection*{(\protect\Verb+$\backslash$==+)}

\subsection*{(\protect\Verb+=$\backslash$=+)}

\subsection*{(\protect\Verb+=:=+)}

\subsection*{(\protect\Verb+is+)}

\section*{Exercise 2}

\end{document}

%%% Local Variables:
%%% TeX-command-extra-options: "-shell-escape"
%%% mode: latex
%%% End:
