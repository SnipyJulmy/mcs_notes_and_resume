\documentclass[a4paper,11pt]{report}

\usepackage{amsmath}
\usepackage{fullpage}
\usepackage[cache=false]{minted}

\author{Sylvain Julmy}
\date{\today}

\setlength{\parindent}{0pt}

\newmintedfile{haskell}{frame=single, framesep=6pt, breaklines=true,fontsize=\scriptsize}
\newcommand{\ex}[3]{\haskellfile[firstline=#1,lastline=#2]{#3.hs}}

\begin{document}

\begin{center}
  \Large{
    Functional and Logic Programming\
    Fall 2017
  }
  
  \noindent\makebox[\linewidth]{\rule{\linewidth}{0.4pt}}
  S03 : Haskell (Lists and lexical analysis)
  \noindent\makebox[\linewidth]{\rule{\linewidth}{0.4pt}}

  \begin{flushleft}
    Professor : Le Peutrec Stephane
    
    Assistant : Lauper Jonathan
  \end{flushleft}

  \noindent\makebox[\linewidth]{\rule{\textwidth}{1pt}}
\end{center}

\section*{Exercise 1}

\subsection*{Fibonnacci}
\ex{16}{28}{ex1}

\subsection*{Product}
\ex{30}{40}{ex1}

\subsection*{Flatten}
\ex{42}{51}{ex1}

\subsection*{DeleteAll}
\ex{53}{65}{ex1}

\subsection*{Insert}
\ex{67}{79}{ex1}

\subsection*{Reverse}
\ex{84}{89}{ex1}

\section*{Exercise 2}

\subsection*{Type declaration}
\ex{21}{26}{ex2}

\newpage
\subsection*{Tokens definition}
\ex{29}{95}{ex2}

\subsection*{Function from series 03}
\ex{105}{124}{ex2}

\subsection*{GetToken}
\ex{99}{103}{ex2}

\subsection*{LexAnalyse}
\ex{126}{130}{ex2}

\end{document}

%%% Local Variables:
%%% TeX-command-extra-options: "-shell-escape"
%%% mode: latex
%%% End:
