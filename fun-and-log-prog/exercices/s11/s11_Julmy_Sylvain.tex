\documentclass[a4paper,11pt]{report}

\usepackage{amsmath}
\usepackage{fullpage}
\usepackage{bussproofs}
\usepackage{mathpartir}

\usepackage{prooftrees}
\newcommand*{\equal}{=}
\newcommand*{\disj}{,}

\usepackage[cache=false]{minted}

\author{Sylvain Julmy}
\date{\today}

\setlength{\parindent}{0pt}

\newminted{prolog}{frame=single, framesep=6pt,
  breaklines=true,fontsize=\scriptsize,linenos}
\newmintedfile{prolog}{frame=single, framesep=6pt,
  breaklines=true,fontsize=\scriptsize}
\newmintinline{prolog}{}
\newcommand{\ex}[3]{\prologfile[firstline=#1,lastline=#2]{#3.pl}}

\begin{document}

\begin{center}
  \Large{
    Functional and Logic Programming\
    Fall 2017
  }
  
  \noindent\makebox[\linewidth]{\rule{\linewidth}{0.4pt}}
  S12 : Prolog
  \noindent\makebox[\linewidth]{\rule{\linewidth}{0.4pt}}

  \begin{flushleft}
    Professor : Le Peutrec Stephane
    
    Assistant : Lauper Jonathan
  \end{flushleft}

  \noindent\makebox[\linewidth]{\rule{\textwidth}{1pt}}
\end{center}

\section*{Exercise 1}

\subsection*{(a)}
\begin{forest}
  for tree={s sep=30mm}
  [$?- p(X).$
  [$p(1)$,edge label={node[midway,above,font=\scriptsize]{$X \equal 1$}}
  [$\square$]]
  [$!$,edge label={node[midway,left,font=\scriptsize]{$X \equal 2$}}
  [$\square$]]
  [$cut$]
  ]
\end{forest}

\subsection*{(b)}
\begin{forest}
  for tree={s sep=20mm}
  [$?- p(X) \disj p(Y).$
  [$ p(Y) $,edge label={node[midway,left,font=\scriptsize]{$X \equal 1$}}
  [$\square$,edge label={node[midway,left,font=\scriptsize]{$Y \equal 1$}}]
  [$!$,edge label={node[midway,left,font=\scriptsize]{$Y \equal 2$}} [$\square$]]
  [$cut$]
  ]
  [$ ! \disj p(Y) $,edge label={node[midway,left,font=\scriptsize]{$X \equal2$}}
  [$p(Y)$
  [$\square$,edge label={node[midway,left,font=\scriptsize]{$Y \equal 1$}}]
  [$!$,edge label={node[midway,left,font=\scriptsize]{$Y \equal 2$}} [$\square$]]
  [$cut$
  [$\square$]]
  ]
  ]
  [$cut$]
  ]
\end{forest}

\subsection*{(c)}
\begin{forest}
  for tree={s sep=20mm}
  [$?- p(X) \disj ! \disj p(Y).$
  [$ ! \disj p(Y) $,edge label={node[midway,left,font=\scriptsize]{$X \equal 1$}}
  [$p(Y)$
  [$\square$,edge label={node[midway,left,font=\scriptsize]{$Y \equal 1$}}]
  [$!$,edge label={node[midway,left,font=\scriptsize]{$Y \equal 2$}}]
  [$cut$]
  ]
  ]
  [$cut$]
  [$cut$]
  ]
\end{forest}

\section*{Exercise 2}

\subsection*{$separate/4$}
\ex{42}{63}{ex2}

\subsection*{$myUnion/4$}
\ex{66}{85}{ex2}

\newpage

\subsection*{$myIntersection/4$}
\ex{88}{109}{ex2}

\subsection*{$maxMin/4$}
\ex{112}{154}{ex2}

\newpage

In addition, we use the following $go/1$ in order to simply visualize the search
tree of the interpreter and show that each implementation reduce the number of
search.

\ex{9}{39}{ex2}

\section*{Exercise 3}

The following implementation of $myMax/3$
\ex{15}{16}{ex3}

would fail for the query \verb|?- myMax(6,5,5)| :

\begin{forest}
  for tree={s sep=20mm}
  [$?- myMax(6 \disj 5 \disj 5).$
  [$ \square $,edge label={node[midway,left,font=\scriptsize]{$X \equal 6 \disj Y
      \equal 5$}}]
  ]
\end{forest}

The interpreter would only explore one branch, because he can't unify
$myMax(6,5,5)$ with $myMax(X,Y,X)$, so it would only explore the branch where we
unify $myMax(6,5,5)$ with $myMax(Y,X,X)$ and then the maximal between $6$ and
$5$ would be $5$, which is wrong.

We should use the following implementation :

\ex{18}{19}{ex3}

\newpage

\section*{Exercise 4}
\ex{10}{27}{ex4}

\newpage

\section*{Exercise 5}
\ex{9}{20}{ex5}

We use the following $go/0$ predicate in order to ``test'' our implementation :

\ex{22}{50}{ex5}

\end{document}

%%% Local Variables:
%%% TeX-command-extra-options: "-shell-escape"
%%% mode: latex
%%% End:
