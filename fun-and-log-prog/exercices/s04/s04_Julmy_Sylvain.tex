\documentclass[a4paper,11pt]{report}

\usepackage{amsmath}
\usepackage{fullpage}
\usepackage[cache=false]{minted}

\author{Sylvain Julmy}
\date{\today}

\setlength{\parindent}{0pt}

\newmintedfile{haskell}{frame=single, framesep=6pt, breaklines=true,fontsize=\scriptsize}
\newcommand{\ex}[3]{\haskellfile[firstline=#1,lastline=#2]{#3.hs}}

\begin{document}

\begin{center}
  \Large{
    Functional and Logic Programming\
    Fall 2017
  }
  
  \noindent\makebox[\linewidth]{\rule{\linewidth}{0.4pt}}
  S03 : Haskell (Lists and lexical analysis)
  \noindent\makebox[\linewidth]{\rule{\linewidth}{0.4pt}}

  \begin{flushleft}
    Professor : Le Peutrec Stephane
    
    Assistant : Lauper Jonathan
  \end{flushleft}

  \noindent\makebox[\linewidth]{\rule{\textwidth}{1pt}}
\end{center}

\section*{Exercise 1}

\subsection*{insert'}
\ex{20}{24}{s04}

\subsection*{insertionSort}
\ex{26}{28}{s04}

\subsection*{takeWhile'}
\ex{30}{30}{s04}

\subsection*{zipWith'}
\ex{36}{39}{s04}

\subsection*{intersect'}
\ex{41}{58}{s04}

\subsection*{divisorList}
\ex{60}{80}{s04}

\subsection*{perfectNumber}
\ex{82}{83}{s04}

\subsection*{perfectNumbers}
\ex{85}{87}{s04}

\section*{Exercise 2 : calculatePolynomial}

\ex{91}{104}{s04}

\end{document}

%%% Local Variables:
%%% TeX-command-extra-options: "-shell-escape"
%%% mode: latex
%%% End: