\documentclass[a4paper,11pt]{report}

\usepackage{amsmath}
\usepackage{fullpage}
\usepackage[cache=false]{minted}

\usemintedstyle{tango}

\newminted{haskell}{frame=single, framesep=6pt, breaklines=true,
  fontsize=\scriptsize}

\author{Sylvain Julmy}
\date{\today}

\setlength{\parindent}{0pt}

\begin{document}

\begin{center}
\Large{
    Functionnal and Logic Programming\\
    Fall 2017
  }
  
  \noindent\makebox[\linewidth]{\rule{\linewidth}{0.4pt}}
  S01 : Haskell Introduction

  \vspace*{1.4cm}

  Author : Sylvain Julmy
  \noindent\makebox[\linewidth]{\rule{\linewidth}{0.4pt}}

  \begin{flushleft}
    Professor : Le Peutrec Stephane
    
    Assistant : Lauper Jonathan
  \end{flushleft}

  \noindent\makebox[\linewidth]{\rule{\textwidth}{1pt}}
\end{center}

\section*{3. Function declaration}
\subsection*{a)}
\begin{haskellcode}
-- Could also replace Int by Integer
sum4 :: Int -> Int -> Int -> Int -> Int
sum4 a b c d = a + b + c + d
\end{haskellcode}


\subsection*{b)}
\begin{haskellcode}
-- Could also replace Int by Integer
max3 :: Int -> Int -> Int -> Int
max3' :: Int -> Int -> Int -> Int
max3'' :: Int -> Int -> Int -> Int
max3''' :: Int -> Int -> Int -> Int
max3'''' :: Int -> Int -> Int -> Int
max3''''' :: Int -> Int -> Int -> Int

-- 3.b.1
max3 a b c =
  if a > b then
    if a > c then
      a
    else
      c
  else if b > c then
      b
  else c

-- 3.b.2
max3' a b c = case (a,b,c) of
  (x,y,z) | x > y && x > z -> x
  (x,y,z) | y > x && y > z -> y
  (_,_,z) -> c

-- 3.b.3
max3'' a b c
  | a > b && a > c = a
  | b > a && b > c = b
  | otherwise = c

-- 3.b.4
max3''' a b c = max' (max' a b) c where
  max' x y = if x > y then x else y

-- 3.b.5
max3'''' a b c = foldl (\x -> \y -> max x y) a [b,c]

-- 3.b.6
max3''''' a b c = m (m a b) c where
  m = \x -> \y -> max x y
\end{haskellcode}

\subsection*{c)}
\begin{haskellcode}
-- a must be a Number (Num) and must be Ordable (Ord), Eq is not required
signe :: (Num a,Ord a, Eq a) => a -> String
signe' :: (Num a,Ord a, Eq a) => a -> String
signe'' :: (Num a,Ord a, Eq a) => a -> String
signe''' :: (Num a,Ord a, Eq a) => a -> String

thisNumberIs :: String -> String
thisNumberIs str = "Ce nombre est " ++ str
pos = thisNumberIs "positif"
nul = thisNumberIs "nulle"
neg = thisNumberIs "negatif"

-- 3.c.1
signe n | n < 0 = "Ce nombre est negatif"
        | n == 0 = "Ce nombre est nulle"
        | n > 0 = "Ce nombre est positif"

-- 3.c.2
signe' n =
  if n < 0 then
    neg
  else if n == 0 then
    nul
  else
    pos

-- 3.c.3
signe'' n = case n of
  x | x < 0 -> neg
  x | x > 0 -> pos
  x | x == 0 -> nul

-- 3.c.4
signe''' 0 = nul
signe''' x | x > 0 = pos
signe''' x | x < 0 = neg
\end{haskellcode}


\end{document}

%%% Local Variables:
%%% TeX-command-extra-options: "-shell-escape"
%%% mode: latex
%%% End: