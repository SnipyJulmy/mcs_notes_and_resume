\documentclass[a4paper,11pt]{report}

\usepackage{fullpage}
\usepackage{graphicx}
\usepackage{amsmath,amssymb}

\usepackage{bussproofs}
\usepackage{mathpartir}
\usepackage{prooftrees}
\usepackage{color}

\newcommand*\circled[1]{\tikz[baseline=(char.base)]{
    \node[shape=circle,draw,inner sep=2pt] (char) {#1};}}

\usepackage[cache=false]{minted}

\newminted{sql}{
  frame=single,
  framesep=6pt,
  breaklines=true,
  fontsize=\scriptsize,
  linenos
}

\newmintedfile{js}{
  frame=single,
  framesep=6pt,
  breaklines=true,
  fontsize=\scriptsize,
  linenos
}

\makeatletter
\pgfmathdeclarefunction{alpha}{1}{%
  \pgfmathint@{#1}%
  \edef\pgfmathresult{\pgffor@alpha{\pgfmathresult}}%
}

% tikz
\usepackage{tikz}
\usetikzlibrary{snakes}

\date{\today}

\setlength{\parindent}{0pt}
\setlength{\parskip}{2.5pt}

\begin{document}

\begin{center}
\Large{
    Big Data Infrastructures\\
    Fall 2018
  }
  
  \noindent\makebox[\linewidth]{\rule{\linewidth}{0.4pt}}
  Lab 01 : SQL Review

  \vspace*{1.4cm}

  Author : Thomas Schaller, Sylvain Julmy
  \noindent\makebox[\linewidth]{\rule{\linewidth}{0.4pt}}

  \begin{flushleft}
    Professor : Philippe Cudré-Mauroux
    
    Assistant : Akansha Bhardwaj
  \end{flushleft}

  \noindent\makebox[\linewidth]{\rule{\textwidth}{1pt}}
\end{center}

\section*{Exercice A}

We use the following to create a new database :
\begin{sqlcode}
  CREATE DATABASE homework_1
  WITH
  OWNER = postgres
  ENCODING = 'UTF8'
  -- LC_COLLATE = 'French_Switzerland.1252'
  -- LC_CTYPE = 'French_Switzerland.1252'
  TABLESPACE = pg_default
  CONNECTION LIMIT = -1;
\end{sqlcode}

\section*{Exercice B}

We create the tables using the following queries :

\subsection*{Paper}
\begin{sqlcode}
  CREATE TABLE Paper (
    paperID  integer primary key,
    title    char(255),
    abstract text
  );
\end{sqlcode}

\subsection*{Author}
\begin{sqlcode}
  CREATE TABLE Author (
    authorID    integer primary key,
    name        char(255),
    email       char(255),
    affiliation char(255)
  );
\end{sqlcode}

\subsection*{Conference}
\begin{sqlcode}
  CREATE TABLE Conference (
    confID  integer primary key,
    name    char(255),
    ranking integer
  );
\end{sqlcode}

\subsection*{Writes}
\begin{sqlcode}
  CREATE TABLE Writes(
    authorID integer,
    paperID integer,
    PRIMARY KEY (authorID, paperID),
    CONSTRAINT fk_writes_author
    REIGN KEY (authorID)
    FERENCES Author(authorID)
    DELETE CASCADE,
    CONSTRAINT fk_writes_paper
    REIGN KEY (paperID)
    FERENCES Paper(paperID)
    DELETE CASCADE
  );
\end{sqlcode}

\subsection*{Submits}
\begin{sqlcode}
  CREATE TABLE Submits(
    paperID integer,
    confID integer,
    isAccepted boolean,
    date date,
    PRIMARY KEY (paperID, confID),
    CONSTRAINT fk_submits_conf
    FOREIGN KEY (confID)
    REFERENCES Conference(confID)
    ON DELETE CASCADE,
    CONSTRAINT fk_submits_paper
    FOREIGN KEY (paperID)
    REFERENCES Paper(paperID)
    ON DELETE CASCADE
  );
\end{sqlcode}

\subsection*{Cites}
\begin{sqlcode}
  CREATE TABLE Cites(
    paperIDfrom integer, 
    paperIDto integer,
    PRIMARY KEY (paperIDfrom, paperIDto),
    CONSTRAINT fk_cites_paperfrom
    FOREIGN KEY (paperIDfrom)
    REFERENCES Paper(paperID)
    ON DELETE CASCADE,
    CONSTRAINT fk_cites_paperto
    FOREIGN KEY (paperIDto)
    REFERENCES Paper(paperID)
    ON DELETE CASCADE
  );
\end{sqlcode}

Note : We put all of the foreign key to "ON DELETE CASCADE", because for
example, for the writes relation, if we delete a paper or an author, we have to
delete also the writes row corresponding to this author.

\section*{Exercice C}

In order to populate the database, we have written a Node.js application using
knex. The following listings shows how we are doing it.

\jsfile{./seed_app/seeds/1_Paper.js}
\jsfile{./seed_app/seeds/2_Author.js}
\newpage
\jsfile{./seed_app/seeds/3_Conference.js}
\jsfile{./seed_app/seeds/4_Writes.js}
\jsfile{./seed_app/seeds/5_Submits.js}
\jsfile{./seed_app/seeds/6_Cites.js}

\section*{Exercice D}

\subsection*{1)}

\begin{sqlcode}
  select affiliation, count(*)
  from author
  group by affiliation;
\end{sqlcode}

\subsection*{2)}

\begin{sqlcode}
  select p.abstract, a.authorId
  from paper as p
  inner join writes as w
    on p.paperId = w.paperId
  inner join author as a
    on w.authorId = a.authorId
  where a.authorId = 2;
\end{sqlcode}

\subsection*{3)}

\begin{sqlcode}
  create view PublishesIn1(authorID, confID) as
    select a.authorID, c.confID
      from author as a
      inner join writes as w
        on a.authorId = w.authorId
      inner join paper as p
        on w.paperId = p.paperId
      inner join submits as s
        on p.paperId = s.paperId
      inner join conference as c
        on s.confId = c.confId
      where s.isAccepted = true;
\end{sqlcode}

\subsection*{4)}

\begin{sqlcode}
  select distinct(w.authorId)
  from writes as w
  inner join cites as c
    on c.paperIdFrom = w.paperId
  inner join writes as w2
    on c.paperIdTo = w.paperId
  where w.authorId = w2.authorId;
\end{sqlcode}

\newpage

\subsection*{5)}

\begin{sqlcode}
  select title 
  from paper 
  where paperid in (
    select paperId
    from writes
    where paperId in (
      select paperid 
      from writes 
      where authorid = 2
    )
    group by paperId
    having count(paperId) > 1
  );
\end{sqlcode}

\end{document}