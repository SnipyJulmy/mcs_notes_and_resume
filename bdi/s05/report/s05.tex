\documentclass[a4paper,11pt]{report}

\usepackage{fullpage}
\usepackage{graphicx}
\usepackage{amsmath,amssymb}

\usepackage{bussproofs}
\usepackage{mathpartir}
\usepackage{prooftrees}
\usepackage{color}
\usepackage{multicol}
\usepackage{geometry}
\usepackage{pdflscape}

\newcommand*\circled[1]{\tikz[baseline=(char.base)]{
    \node[shape=circle,draw,inner sep=2pt] (char) {#1};}}

\usepackage[cache=false]{minted}

\newmintedfile{java}{
  frame=single,
  framesep=6pt,
  breaklines=true,
  fontsize=\scriptsize,
  linenos,
  mathescape,
  firstnumber=1
}

\newmintedfile{bash}{
  frame=single,
  framesep=6pt,
  breaklines=true,
  fontsize=\scriptsize,
  linenos,
  mathescape,
  firstnumber=1
}

\newminted{java}{
  frame=single,
  framesep=2mm,
  breaklines=true,
  fontsize=\scriptsize,
  linenos,
  mathescape
}

\newminted{bash}{
  frame=single,
  framesep=2mm,
  breaklines=true,
  fontsize=\scriptsize,
  linenos,
  mathescape
}


\makeatletter
\pgfmathdeclarefunction{alpha}{1}{%
  \pgfmathint@{#1}%
  \edef\pgfmathresult{\pgffor@alpha{\pgfmathresult}}%
}

% tikz
\usepackage{tikz}
\usetikzlibrary{snakes}

\date{\today}

\setlength{\parindent}{0pt}
\setlength{\parskip}{2.5pt}

\begin{document}

\begin{center}
\Large{
    Big Data Infrastructures\\
    Fall 2018
  }
  
  \noindent\makebox[\linewidth]{\rule{\linewidth}{0.4pt}}
  Lab 05 : Hadoop and MapReduce

  \vspace*{1.4cm}

  Author : Thomas Schaller, Sylvain Julmy
  \noindent\makebox[\linewidth]{\rule{\linewidth}{0.4pt}}

  \begin{flushleft}
    Professor : Philippe Cudré-Mauroux
  \end{flushleft}

  \noindent\makebox[\linewidth]{\rule{\textwidth}{1pt}}
\end{center}

\section*{Running the experiment}

In order to easely run the experiments, we develop a small bash script (see
\ref{lst:src-bash-1}) which compile, run, clean and move the output of the
experiments to our local directory in order to watch them.

\begin{listing}[ht]
\centering
\bashfile{../src/main/java/run.sh}
\caption{Script to run the exercises on the Hadoop cluster.}
\label{lst:scr-bash-1}
\end{listing}

\section*{Wordcount : }

Listing \ref{lst:ex1}, \ref{lst:ex2} and \ref{lst:ex3} shows the code for the
various version of the Wordcount.  The function to initialize the map used for
$htmlEntities$ is shown in listing \ref{lst:ex3-b}.

\begin{listing}[ht]
\centering
\javafile[firstline=14]{../src/main/java/WordCountEx1.java}
\caption{First implementation for Wordcount.}
\label{lst:ex1}
\end{listing}

\begin{listing}[ht]
\centering
\javafile[firstline=14]{../src/main/java/WordCountEx2.java}
\caption{First improvement for Wordcount.}
\label{lst:ex2}
\end{listing}

\begin{listing}[ht]
\centering
\begin{javacode}
public class WordCountEx3 {

    public static void main(String[] args) throws Exception {
        Configuration conf = new Configuration();
        Job job = Job.getInstance(conf, "word count");
        job.setJar("WordCountEx3.jar");
        job.setJarByClass(WordCountEx3.class);
        job.setMapperClass(TokenizerMapper.class);
        job.setCombinerClass(IntSumReducer.class);
        job.setReducerClass(IntSumReducer.class);
        job.setOutputKeyClass(Text.class);
        job.setOutputValueClass(IntWritable.class);
        FileInputFormat.addInputPath(job, new Path(args[0]));
        FileOutputFormat.setOutputPath(job, new Path(args[1]));
        System.exit(job.waitForCompletion(true) ? 0 : 1);
    }

    public static class TokenizerMapper extends Mapper<Object, Text, Text, IntWritable> {
        private final static IntWritable one = new IntWritable(1);
        public HashMap<String, String> htmlEntities;
        private Text word = new Text();

        public void map(
                Object key,
                Text value,
                Context context
        ) throws IOException, InterruptedException {

            initHashMap();
            String value2 = value.toString();
            value2 = value2.replaceAll("[,\\.:()\"@?!]", "")
                           .replaceAll("&rdquo;", "")
                           .toLowerCase();
            for (Map.Entry<String, String> entry : htmlEntities.entrySet()) {
                String key1 = entry.getKey();
                String value1 = entry.getValue();
                value2 = value2.replaceAll(key1, value1);
            }
            StringTokenizer itr = new StringTokenizer(value2);
            while (itr.hasMoreTokens()) {
                word.set(itr.nextToken());
                context.write(word, one);
            }
        }
    }

    public static class IntSumReducer extends Reducer<Text, IntWritable, Text, IntWritable> {
        private IntWritable result = new IntWritable();

        public void reduce(
                Text key,
                Iterable<IntWritable> values,
                Context context
        ) throws IOException, InterruptedException {

            int sum = 0;
            for (IntWritable val : values) {
                sum += val.get();
            }
            result.set(sum);
            context.write(key, result);
        }
    }
}
\end{javacode}
\caption{First improvement for Wordcount.}
\label{lst:ex3}
\end{listing}

\begin{listing}[ht]
\centering
\begin{javacode}
  public void initHashMap() {
            htmlEntities = new HashMap<String, String>();
            htmlEntities.put("&lt;", "<"); htmlEntities.put("&gt;", ">");
            htmlEntities.put("&amp;", "&"); htmlEntities.put("&quot;", "\"");
            htmlEntities.put("&agrave;", "à"); htmlEntities.put("&Agrave;", "À");
            htmlEntities.put("&acirc;", "â"); htmlEntities.put("&auml;", "ä");
            htmlEntities.put("&Auml;", "Ä"); htmlEntities.put("&Acirc;", "Â");
            htmlEntities.put("&aring;", "å"); htmlEntities.put("&Aring;", "Å");
            htmlEntities.put("&aelig;", "æ"); htmlEntities.put("&AElig;", "Æ");
            htmlEntities.put("&ccedil;", "ç"); htmlEntities.put("&Ccedil;", "Ç");
            htmlEntities.put("&eacute;", "é"); htmlEntities.put("&Eacute;", "É");
            htmlEntities.put("&egrave;", "è"); htmlEntities.put("&Egrave;", "È");
            htmlEntities.put("&ecirc;", "ê"); htmlEntities.put("&Ecirc;", "Ê");
            htmlEntities.put("&euml;", "ë"); htmlEntities.put("&Euml;", "Ë");
            htmlEntities.put("&iuml;", "ï"); htmlEntities.put("&Iuml;", "Ï");
            htmlEntities.put("&ocirc;", "ô"); htmlEntities.put("&Ocirc;", "Ô");
            htmlEntities.put("&ouml;", "ö"); htmlEntities.put("&Ouml;", "Ö");
            htmlEntities.put("&oslash;", "ø"); htmlEntities.put("&Oslash;", "Ø");
            htmlEntities.put("&szlig;", "ß"); htmlEntities.put("&ugrave;", "ù");
            htmlEntities.put("&Ugrave;", "Ù"); htmlEntities.put("&ucirc;", "û");
            htmlEntities.put("&Ucirc;", "Û"); htmlEntities.put("&uuml;", "ü");
            htmlEntities.put("&Uuml;", "Ü"); htmlEntities.put("&nbsp;", " ");
            htmlEntities.put("&copy;", "\u00a9"); htmlEntities.put("&reg;", "\u00ae");
            htmlEntities.put("&euro;", "\u20a0"); htmlEntities.put("&rsquo;", "'");
        }
\end{javacode}
\caption{Initialization of $htmlEntities$.}
\label{lst:ex3-b}
\end{listing}

\section*{Quadruples : }

Listing \ref{lst:ex4} shows the code for counting the number of literals for
each node, and listing \ref{lst:ex5} shows the code for computing the In and Out
degree of each node. Note that we have also created a class $IntTriple$ (in
listing \ref{lst:ex5-b}) to simplify and pass Triple of int between the map and
the reduce parts.

\begin{listing}[ht]
\centering
\javafile[firstline=15]{../src/main/java/QuadruplesCounter.java}
\caption{Map-Reduce program for counting the number of literals for each node.}
\label{lst:ex4}
\end{listing}

\begin{listing}[ht]
\centering
\begin{javacode}
public class InOutDegreeCounter {

    public static void main(String[] args) throws Exception {
        Configuration conf = new Configuration();

        Job job = Job.getInstance(conf, "in and out degree count");

        job.setJar("InOutDegreeCounter.jar");
        job.setJarByClass(InOutDegreeCounter.class);

        job.setMapperClass(TokenizerMapper.class);
        job.setCombinerClass(IntSumReducer.class);
        job.setReducerClass(IntSumReducer.class);

        // Map output
        job.setMapOutputKeyClass(Text.class);
        job.setMapOutputValueClass(IntTriple.class);

        // Global output
        job.setOutputKeyClass(Text.class);
        job.setOutputValueClass(IntTriple.class);

        FileInputFormat.addInputPath(job, new Path(args[0]));
        FileOutputFormat.setOutputPath(job, new Path(args[1]));
        System.exit(job.waitForCompletion(true) ? 0 : 1);
    }

    // Test if s is a valid URI or not
    private static boolean isValidURI(String s) {
        try {
            URI uri = new URI(s);
            return true;
        } catch (URISyntaxException e) {
            return false;
        }
    }
 }
\end{javacode}
\caption{Configuration for the last exercise and a small method to check URI.}
\label{lst:ex5}
\end{listing}

\begin{listing}[ht]
\centering
\begin{javacode}
public static class TokenizerMapper extends Mapper<Object, Text, Text, IntTriple> {

    private Text word = new Text();

    public void map(Object key, Text value, Context context)
            throws IOException, InterruptedException {

        String[] values = value.toString().split("\t");
        String subject = values[0];
        String predicate = values[1];
        String object = values[2];
        String provenance = values[3];

        assert isValidURI(subject);
        assert isValidURI(predicate);
        assert isValidURI(provenance);

        // Structure for the triple is (#Literal,#In,#Out)

        if (isValidURI(object)) { // object is a valid URI
            word.set(subject);
            // increment the out count
            IntTriple triple = new IntTriple(0, 0, 1);
            context.write(word, triple);
        } else { // object is a literal
            word.set(subject);
            // increment the literal count and the out count
            IntTriple triple = new IntTriple(1, 0, 1);
            context.write(word, triple);
        }

        word.set(object);
        // increment the in count for the object
        IntTriple triple2 = new IntTriple(0, 1, 0);
        context.write(word, triple2);
    }
}
\end{javacode}
\caption{Mapper class for the last exercise.}
\label{lst:ex5-b}
\end{listing}

\begin{listing}[h]
\centering
\begin{javacode}
public static class IntSumReducer extends Reducer<Text, IntTriple, Text, IntTriple> {
    public void reduce(Text key, Iterable<IntTriple> values, Context context)
            throws IOException, InterruptedException {

        int sumLiteral = 0, sumIn = 0, sumOut = 0;

        for (IntTriple triple : values) {
            sumLiteral += triple.a.get();
            sumIn += triple.b.get();
            sumOut += triple.c.get();
        }

        assert isValidURI(key.toString());

        if (sumLiteral >= 10) {
            context.write(key, new IntTriple(sumLiteral, sumIn, sumOut));
        }
    }
}
\end{javacode}
\caption{Reducer class for the last exercise.}
\label{lst:ex5-b}
\end{listing}

\begin{listing}[h]
\centering
\begin{javacode}
static class IntTriple implements Writable {

    private IntWritable a;
    private IntWritable b;
    private IntWritable c;

    public IntTriple() {
        set(new IntWritable(0), new IntWritable(0), new IntWritable(0));
    }

    public IntTriple(IntWritable a, IntWritable b, IntWritable c) {
        this.a = a;
        this.b = b;
        this.c = c;
    }

    public IntTriple(int a, int b, int c) {
        this.a = new IntWritable(a);
        this.b = new IntWritable(b);
        this.c = new IntWritable(c);
    }

    public int compareTo(IntTriple that) {
        int cmp = this.a.compareTo(that.a);
        if (cmp != 0) return cmp;
        cmp = this.b.compareTo(that.b);
        if (cmp != 0) return cmp;
        return this.c.compareTo(that.c);
    }

    public void set(IntWritable a, IntWritable b, IntWritable c) {
        this.a = a;
        this.b = b;
        this.c = c;
    }

    public void write(DataOutput dataOutput) throws IOException {
        a.write(dataOutput);
        b.write(dataOutput);
        c.write(dataOutput);
    }

    public void readFields(DataInput dataInput) throws IOException {
        a.readFields(dataInput);
        b.readFields(dataInput);
        c.readFields(dataInput);
    }

    @Override
    public int hashCode() {
        return a.hashCode() * 163 * 163 + b.hashCode() * 163 + c.hashCode();
    }

    @Override
    public boolean equals(Object obj) {
        if (obj instanceof IntTriple) {
            IntTriple that = (IntTriple) obj;
            return this.a.equals(that.a) &&
                   this.b.equals(that.b) &&
                   this.c.equals(that.c);
        }
        return false;
    }

    @Override
    public String toString() {
        return String.format("(%d,%d,%d)", a.get(), b.get(), c.get());
    }
}
\end{javacode}
\caption{Implementation of a IntTriple class which represent a Triple of IntWritable.}
\label{lst:ex5-b}
\end{listing}


\section*{Outputs : }

The various outputs of our programm are available under the following HDFS
folder :
\begin{verbatim}
/bdi_2018/bdi18_07/ex${exerice_number}_output
\end{verbatim}

\end{document}