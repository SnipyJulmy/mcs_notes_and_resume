\documentclass[a4paper,11pt]{report}

\usepackage{amsmath}
\usepackage{fullpage}
\usepackage{bussproofs}
\usepackage{mathpartir}
\usepackage{prooftrees}
\usepackage{color}

\usepackage{tikz}
\usetikzlibrary{automata,positioning}

\author{Sylvain Julmy}
\date{\today}

\setlength{\parindent}{0pt}

\newcommand*{\equal}{=}
\newcommand*{\Pc}{\mathcal{P}}
\newcommand*{\NPc}{$\mathcal{NP}$}
\newcommand*{\NPCc}{$\mathcal{NPC}$}
\newcommand*{\NPcp}{$\mathcal{NP}$-complete }
\newcommand*{\NPce}{$\mathcal{NP}$-completeness }
\newcommand*{\N}{NAE-$4$-SAT }

\begin{document}

\begin{center}
  \large{
    Formal Methods\\
    Fall 2017
  }
  
  \noindent\makebox[\linewidth]{\rule{\linewidth}{0.4pt}}
  S09
  \noindent\makebox[\linewidth]{\rule{\linewidth}{0.4pt}}

  \begin{flushleft}
    Professor : Ultes-Nitsche Ulrich

    Assistant : Christophe Stammet
  \end{flushleft}

  
  \noindent\makebox[\linewidth]{\rule{\linewidth}{0.4pt}}

  Submitted by Sylvain Julmy
  
  \noindent\makebox[\linewidth]{\rule{\textwidth}{1pt}}
\end{center}

\section*{Exercise 1}

\subsection*{\texttt{(1)}}

\[
  \phi = \forall e.(S(e) \to \exists d.(P(d)))
\]

\begin{itemize}
\item The scope of $\forall e$ is $F[e] = S(e) \to \exists d.(P(d))$
\item The scope of $\exists d$ is $F[d] = P(d)$
\item There is no free variable
\item The formula $\phi$ is closed, because $e$ is bound in $F[e]$ by the
  $\forall$ quantifier and $d$ is bound in $F[d]$ by the $\exists$ quantifier
\end{itemize}

\subsection*{\texttt{(2)}}

\[
  \phi = \forall e.P(a,e) \to \forall e.P(b,e)
\]

\begin{itemize}
\item The scope of $\forall e.P(a,e)$ is $F[e] = P(a,e)$
\item The scope of $\forall e.P(b,e)$ is $F[e]' = P(b,e)$
\item $a$ and $b$ are free variables
\item The formula $\phi$ is not closed because $a$ and $b$ are free variables
\end{itemize}

\subsection*{\texttt{(3)}}

\[
  \phi = \exists x.(S(x,b) \wedge \forall y.(S(y,b) \to (x = y)))
\]

\begin{itemize}
\item The scope of $\exists x$ is $F[x] = S(x,b) \wedge \forall y.(S(y,b) \to (x = y))$
\item The scope of $\forall y$ is $F[y] = S(y,b) \to (x = y)$
\item $x$ is a free variables in $\phi$
\item The formula $\phi$ is not closed because $x$ is a free variable
\end{itemize}

\section*{Exercise 2}

\subsection*{\texttt{(1)}}

``For all successfull exam, we are going to make a party one day''.

\subsection*{\texttt{(2)}}

``If anna passes any exam, bill would pass that exam too''.

\subsection*{\texttt{(3)}}

``Bill has a sister and if anyone is the sister of Bill, this is the same
person''.

\vspace*{0.3cm}

Maybe one can argue that ``Bill has only one sister'', because Bill can't have
less than $1$ sister, and all the person that are a sister of Bill is the same
person.

\section*{Exercise 3}

The only part that differ from the three given formula is the type of quantifier
used in front of their scope. So each formula is solved the same ways and only
after we indicates which formula is true or not.

\begin{gather*}
  I \lhd \{x = anna, y = anna \} \overset{?}{\models} \phi \\
  \longrightarrow \\
  cat(anna,anna) = cat(anna,anna) \vee cat(anna,anna) = cat(anna,anna) \Leftrightarrow \\
  annaanna = annaanna \vee annaanna = annaanna \Leftrightarrow \\
  true \vee true \Leftrightarrow \\
  true
\end{gather*}

\begin{gather*}
  I \lhd \{x = anna, y = bob \} \overset{?}{\models} \phi \\
  \longrightarrow \\
  cat(anna,bob) = cat(anna,anna) \vee cat(anna,bob) = cat(bob,bob) \Leftrightarrow \\
  annabob = annaanna \vee annabob = bobbob \Leftrightarrow \\
  false \vee false \Leftrightarrow \\
  false
\end{gather*}

\begin{gather*}
  I \lhd \{x = bob, y = anna \} \overset{?}{\models} \phi \\
  \longrightarrow \\
  cat(bob,anna) = cat(bob,bob) \vee cat(bob,anna) = cat(anna,anna) \Leftrightarrow \\
  bobanna = bobbob \vee bobanna = annaanna \Leftrightarrow \\
  false \vee false \Leftrightarrow \\
  false
\end{gather*}

\begin{gather*}
  I \lhd \{x = bob, y = bob \} \overset{?}{\models} \phi \\
  \longrightarrow \\
  cat(bob,bob) = cat(bob,bob) \vee cat(bob,bob) = cat(bob,bob) \Leftrightarrow \\
  bobbob = bobbob \vee bobbob = bobbob \Leftrightarrow \\
  true \vee true \Leftrightarrow \\
  true
\end{gather*}

In resume, we have the following :

\begin{itemize}
\item $I \lhd \{x = anna, y = anna \} \models \phi$
\item $I \lhd \{x = anna, y = bob \} \not\models \phi$
\item $I \lhd \{x = bob, y = anna \} \not\models \phi$
\item $I \lhd \{x = bob, y = bob \} \models \phi$
\end{itemize}

\subsection*{\texttt{(1)}}

\[
  \phi = \forall x. \forall y. ((cat(x,y) = cat(x,x)) \vee (cat(x,y) = cat(y,y)))
\]

$\phi$ is not true under the interpretation $I$.

\subsection*{\texttt{(2)}}

\[
  \phi = \forall x. \exists y. ((cat(x,y) = cat(x,x)) \vee (cat(x,y) = cat(y,y)))
\]

$\phi$ is not true under the interpretation $I$ because we can't find a $x$
such that for all $x$, $\phi$ is true.

\subsection*{\texttt{(3)}}

\[
  \phi = \exists x. \exists y. ((cat(x,y) = cat(x,x)) \vee (cat(x,y) = cat(y,y)))
\]

$\phi$ is true under the interpretation $I$ for the following assignement :

\begin{itemize}
\item $I \lhd \{x = anna, y = anna \} \models \phi$
\item $I \lhd \{x = bob, y = bob \} \models \phi$
\end{itemize}




\end{document}

%%% Local Variables:
%%% mode: latex
%%% End:
