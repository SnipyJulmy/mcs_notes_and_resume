\documentclass[a4paper,11pt]{report}

\usepackage{amsmath}
\usepackage{fullpage}
\usepackage{bussproofs}
\usepackage{mathpartir}
\usepackage{prooftrees}
\usepackage{color}

\usepackage{tikz}
\usetikzlibrary{automata,positioning}

\author{Sylvain Julmy}
\date{\today}

\setlength{\parindent}{0pt}

\newcommand*{\equal}{=}
\newcommand*{\Pc}{\mathcal{P}}
\newcommand*{\NPc}{\mathcal{NP}}
\newcommand*{\NPCc}{\mathcal{NPC}}
\newcommand*{\NPcp}{$\mathcal{NP}$-complete }
\newcommand*{\NPce}{$\mathcal{NP}$-completeness }

\begin{document}

\begin{center}
  \large{
    Formal Methods\\
    Fall 2017
  }
  
  \noindent\makebox[\linewidth]{\rule{\linewidth}{0.4pt}}
  S07
  \noindent\makebox[\linewidth]{\rule{\linewidth}{0.4pt}}

  \begin{flushleft}
    Professor : Ultes-Nitsche Ulrich

    Assistant : Christophe Stammet
  \end{flushleft}

  
  \noindent\makebox[\linewidth]{\rule{\linewidth}{0.4pt}}

  Submitted by Sylvain Julmy
  
  \noindent\makebox[\linewidth]{\rule{\textwidth}{1pt}}
\end{center}

\section*{Exercise 1}

\subsection*{(1)}
Cook's theorem is important for complexity theory because it it a proof about
the \NPce of the \texttt{SAT} algorithm. Usually, we demonstrate that a problem
is in the \NPcp class by reducing it in polynomial time to a problem which is
already \NPcp. The problem is to have a known \NPcp problem to use and Cook
demonstrate that \texttt{SAT} is \NPcp.

\subsection*{(2)}

In order to demonstrate that any problem can be reduce to \texttt{SAT}, we don't
encode the problem itself but the algorithm that is solving it. We know there
exist such an algorithm because we encode problem that are in $\NPc$. $\phi$
represents the algorithm in propositional logic.

The variables we are using for the proof are the following :
\begin{itemize}
\item $X_{i,j}$ which represents the $j$th symbols after step $i$,
\item $y_{i,j,z}$ which represents that's after step $i$, the square $j$
  contains the symbol $z$.
\end{itemize}

An interpretation of $\phi$ sets to true means that $\phi$ is an accepting
computation of the non-deterministic Turing Machine $M$.

\subsection*{(3)}

$\phi$ consist of $4$ sub-formula $\phi = U \wedge S \wedge N \wedge F$ where
\begin{itemize}
\item $U$ represents that each symbol in the ID sequence is unique, on other
  words, each square has exactly one symbol after each step.
\item $S$ represents that's the start ID is correct, which means that there
  exist only one unique initial state. 
\item $N$ represents that's the next ID is correct, which means that after a
  step, the head of $M$ is in exactly one position.
\item $F$ represents that $M$ is ending in an accepting state, which is
  $X_{p(n),y}$ is set to $true$ where $q_y$ is the accepting state.
\end{itemize}

All of those formulae are necessary to validate that our initial configuration
is well-defined.

\subsection*{(4)}

We know that $p(n)$ is polynomial and the formula $N$ is used to validate that
the Turing Machine is in exactly one state after each step. We constuct the $N$
formula like the following :

\[
  (N_0,N_1,\cdot,N_{p(n)})
\]

where

\[
  N_i = (A_{i,0} \vee B_{i,0}) \wedge (A_{i,1} \vee B_{i,1}) \wedge \cdots \wedge (A_{i,p(n)} \vee B_{i,p(n)})
\]

$A_{i,j}$ represents that the $j$th symbol in the $i$th id is the state and
$B_{i,j}$ represents that the $j$th symbol in the $i$th id is not the state.

So it means that the Turing machine is in at least one state after each step and
that it can't be in two states at once.

This construction is in $O(p(n)^2)$ which is polynomial, because $p(n)$ is polynomial.

\subsection*{(5)}

For this exercice, we would use a slightly different notation in order to be
more expressive about the formula itself :
\begin{itemize}
\item $Q_{i,j}$ represents that's after step $i$, $M$ is in state $q_j$.
\item $H_{i,j}$ represents that's the head of $M$ is in the $j$th square after
  the step $i$.
\item $S_{i,j,z}$ represents that's after step $i$, the square $j$ contains
  symbol $z$. This is just some redefinition of $X_{i,j}$ and $y_{i,j,z}$.
\end{itemize}

The sentence ``$M$ starts on input ''1001``'' is is descrive by the following
formulas :

\begin{align*}
  S 
  =& Q_{0,0}\ \wedge && \text{the initial state (at step $0$) is $q_0$} \\
  & H_{0,1}\ \wedge && \text{the head is on the first square of the tape at step $0$} \\
  & S_{0,1,1} \wedge S_{0,2,0} \wedge S_{0,3,0} \wedge S_{0,4,1}\ \wedge && \text{the tapes contains $1$ at square $1$, $0$ at square $2$, \dots} \\
  & S_{0,1,1} \wedge S_{0,1,1} \wedge \cdots \wedge S_{0,p(n),B} && \text{the rest of the tape is containing blank symbol}.
\end{align*}



\section*{Exercise 2}

The idea of Cook in order to prove that \texttt{SAT} is \NPcp is to demonstrate
that any algorithm that is solving a problem can be encode into a propositional
formula in a polynomial time. We already know that \texttt{SAT} is
$\mathbb{NP}$-Hard because a non-deterministic Turing Machine can solve
\texttt{SAT} in a polynomial time. Then, if we show that any algorithm can be
encoded as a propositional formula in a polynomial time, it would prove that
\texttt{SAT} is \NPcp.

To do so, we don't encode the problem itself but the Turing Machine that is
solving the problem. We know that there exist a Turing Machine because we wan't
to demonstrate \NPce of known problem with algorithm.

Using the variables $Q$,$H$ and $S$, we can describe any computation of a Turing
machine in a polynomial size using the function $p(n)$. We know that $p(n)$ is
polynomial because it runs on a non-deterministic Machine in polynomial time (we
know the algorithm).

The proof also show that constucting the formulas used in order to represent the
variables, configurations clauses and transitions clauses are in a polynomial
complexity with respect to $p(n)$ : $O(p(n)^{O(n)})$.

Finally, we have show that any algorithm are encodable into a propositional
formula in complexity $O(p(n)^{O(n)})$. By any algorithm, there are also
$\mathbb{NP}$-Hard ones then \texttt{SAT} is \NPcp.

\end{document}

%%% Local Variables:
%%% mode: latex
%%% End:
