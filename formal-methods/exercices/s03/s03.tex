\documentclass[a4paper,11pt]{report}

\usepackage{amsmath}
\usepackage{fullpage}
\usepackage{bussproofs}
\usepackage{mathpartir}
\usepackage{prooftrees}
\usepackage{color}

\author{Sylvain Julmy}
\date{\today}

\setlength{\parindent}{0pt}

\newcommand*{\equal}{=}

\begin{document}

\begin{center}
  \large{
    Formal Methods\\
    Fall 2017
  }
  
  \noindent\makebox[\linewidth]{\rule{\linewidth}{0.4pt}}
  S03
  \noindent\makebox[\linewidth]{\rule{\linewidth}{0.4pt}}

  \begin{flushleft}
    Professor : Ultes-Nitsche Ulrich

    Assistant : Christophe Stammet
  \end{flushleft}

  
  \noindent\makebox[\linewidth]{\rule{\linewidth}{0.4pt}}

  Submitted by Sylvain Julmy
  
  \noindent\makebox[\linewidth]{\rule{\textwidth}{1pt}}
\end{center}

\section*{Exercise 1 : }

We denote $V$ the variable $P$ from the SAT-algorithm given in the course. Note
: we assume that the algorithm is smart enough to simplify expressions like $\top
\wedge P = P$ or $\bot \vee Q = Q$, in order to decrease the size of the
following resolution :

\begin{center}
\begin{forest}
  [$SAT((P \vee \neg Q) \leftrightarrow (\neg P \wedge Q))$
  [$V \equal P$
  [$SAT((\top \vee \neg Q) \leftrightarrow (\neg \top \wedge Q)) \vee SAT((\bot
  \vee \neg Q) \leftrightarrow (\neg \bot \wedge Q))$
  [$SAT((\top \vee \neg Q) \leftrightarrow (\neg \top \wedge Q))$
  [$SAT((\top) \leftrightarrow (\bot \wedge Q))$
  [$SAT((\top) \leftrightarrow (\bot))$
  [$SAT(\bot)$
  [$false$]]]]]
  [$SAT((\bot \vee \neg Q) \leftrightarrow (\neg \bot \wedge Q))$
  [$SAT((\neg Q) \leftrightarrow (\top \wedge Q))$
  [$SAT(\neg Q \leftrightarrow Q)$
  [$V \equal Q$
  [$SAT(\neg \top \leftrightarrow \top ) \vee SAT(\neg \bot \leftrightarrow
  \bot)$
  [$SAT(\neg \top \leftrightarrow \top)$
  [$SAT(\bot \leftrightarrow \top)$
  [$SAT(\bot)$
  [$false$]]]]
  [$SAT(\neg \bot \leftrightarrow \bot)$
  [$SAT(\top \leftrightarrow \bot)$
  [$SAT(\bot)$
  [$false$]]]]
  ]]]]]]]]
\end{forest}
\end{center}

So we have $false \vee false \vee false$ is $false$.

\section*{Exercice 2}

\subsection*{(a)}

Two formulae are equivalent if they have the same models, equisatisfiability is
weaker than equivalence. If two formulae $A$ and $B$ are equisatisfiable, it
means that if $A$ is satisfiable, $B$ is satisfiable too and if $B$ is
satisfiable, $A$ is satisfiable. Two formulae can be satisfiable but not
equivalent.

For example, $A$ and $\top$ are equisatisfiable, if $\top$ is satisfiable (which
is always the case), so $A$ is satisfiable, but $A$ is not equivalent to $\top$.

The two formulae $(P \vee \neg P)$ and $(P \implies P)$ are equivalent, both are
tautology. Because they are equivalent, they are also equisatisfiable. So the
equivalence implies the equisatisfiability, but the equisatisfiability don't
imply the equivalence.

Another example of equisatisfiability : $A \vee B$ and $(A \vee n) \wedge (B
\vee \neg n)$, but they are not equivalent.

\subsection*{(b)}

The transformation of a formula $\phi$ into an equivalent formula $\phi'$ in CNF
has an exponential complexity (in the worst case). Turning $\phi$ into an
equisatisfiability formula $\phi_{eq}$ has a linear complexity. Because we just
have to check for the satisfiability of a formula, we can use a equisatisfiable
one to reduce the space and time complexity.

\section*{Exercise 3}

\[
  F \equiv \neg P \wedge (Q \to R)
\]

\subsection*{Transformation into a equisatisfiability formula $F'$}

\begin{align*}
  rep(F) &= rep(\neg P \wedge (Q \to R)) = P_1 \\
  rep(\neg P)   &= P_2\\
  rep(Q \to R) &= P_3 \\
  rep(Q) &= Q \\
  rep(P) &= P \\
  rep(R) &= R
\end{align*}

\begin{align*}
  F' = P1 &\wedge (\neg P_1 \vee P_2) \wedge (\neg P_1 \vee P_3) \wedge (P_1 \vee \neg P_2 \vee \neg P_3)
  && \text{($enc(\neg P \wedge (Q \to R))$)} \\
          &\wedge (\neg P_2 \vee \neg P) \wedge (P_2 \vee P)
  && \text{($enc(\neg P)$)} \\
          &\wedge (P_3 \vee Q) \wedge (P_3 \vee \neg R) \wedge (\neg P_3 \vee \neg Q \vee R)
  && \text{($enc(Q \to R)$)}
\end{align*}

\subsection*{Applying the resolution}

\subsubsection*{Using $P_1$}
\begin{gather*}
P1 \wedge (\neg P_1 \vee P_2) \wedge (\neg P_1 \vee P_3) \wedge (P_1 \vee \neg
P_2 \vee \neg P_3) \wedge \cdots = \\
P_3 \wedge P_2 \wedge \underbrace{(P_2 \vee \neg P_2 \vee \neg P_3)}_\top \underbrace{\wedge (P_3 \vee \neg
P_2 \vee \neg P_3)}_\top \wedge \cdots = \\
P_3 \wedge P_2 \wedge \cdots
\end{gather*}

\subsubsection*{Using $P_2$}

\begin{gather*}
  P_3 \wedge P_2 \wedge (\neg P_2 \vee \neg P) \wedge (P_2 \vee P) \wedge \cdots
  = \\
  P_3 \wedge \neg P \wedge \underbrace{(\neg P \vee P)}_\top \wedge \cdots = \\
  P_3 \wedge \neg P \wedge \cdots
\end{gather*}

\subsubsection*{Using $P_3$}

\begin{gather*}
  P_3 \wedge \neg P \wedge (P_3 \vee Q) \wedge (P_3 \vee \neg R) \wedge (\neg
  P_3 \vee \neg Q \vee R) = \\
  \neg P \wedge (\neg Q \vee R) \wedge \underbrace{Q \vee \neg Q \vee R}_\top
  \wedge \underbrace{\neg R \vee \neg Q \vee R}_\top = \\
  \neg P \wedge (\neg Q \vee R)
\end{gather*}

$F$ is satisfiable with the interpretation $I : \{ P \mapsto false, Q \mapsto
false, R \mapsto true \}$

\end{document}

%%% Local Variables:
%%% mode: latex
%%% End:
