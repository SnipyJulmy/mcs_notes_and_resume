\documentclass[a4paper,11pt]{report}

\usepackage{amsmath}
\usepackage{fullpage}
\usepackage{bussproofs}
\usepackage{mathpartir}
\usepackage{prooftrees}
\usepackage{color}

\usepackage{tikz}
\usetikzlibrary{automata,positioning}

\author{Sylvain Julmy}
\date{\today}

\setlength{\parindent}{0pt}

\newcommand*{\equal}{=}
\newcommand*{\Pc}{\mathcal{P}}
\newcommand*{\NPc}{$\mathcal{NP}$}
\newcommand*{\NPCc}{$\mathcal{NPC}$}
\newcommand*{\NPcp}{$\mathcal{NP}$-complete }
\newcommand*{\NPce}{$\mathcal{NP}$-completeness }
\newcommand*{\N}{NAE-$4$-SAT }

\begin{document}

\begin{center}
  \large{
    Formal Methods\\
    Fall 2017
  }
  
  \noindent\makebox[\linewidth]{\rule{\linewidth}{0.4pt}}
  S08
  \noindent\makebox[\linewidth]{\rule{\linewidth}{0.4pt}}

  \begin{flushleft}
    Professor : Ultes-Nitsche Ulrich

    Assistant : Christophe Stammet
  \end{flushleft}

  
  \noindent\makebox[\linewidth]{\rule{\linewidth}{0.4pt}}

  Submitted by Sylvain Julmy
  
  \noindent\makebox[\linewidth]{\rule{\textwidth}{1pt}}
\end{center}

\section*{Exercise 1}

In order to show that $4$TA-SAT is \NPcp, we first have to show that $4$TA-SAT is in
$\mathcal{NP}$, then we proof that $4$TA-SAT is \NPcp by reducing it in
polynomial time to the known \NPcp problem SAT.

\subsection*{$4$TA-SAT is in $\mathcal{NP}$}

We can construct a non-deterministic Turing Machine that solve the $4$TA-SAT
problem in polynomial time. The TM can check all the truth assignments to the
propositional formula in parallel and find $4$ evaluation of the formula in
polynomial time.

\subsection*{$4$TA-SAT is \NPcp}

In order to use an $4$TA-SAT algorithm to solve SAT, we are going to transform a
formula $phi$ which is in CNF and the input of the SAT algorithm. $4$TA-SAT is
$true$ if and only if there are $4$ different interpretations which satisfies
$\phi$.

We use the following transformation, where $phi^\prime$ is the transformed
formula $phi$ for the $4$TA-SAT algorithm :

\[
  \phi^\prime = \phi \vee (x \wedge y)
\]

Were $x$ and $y$ are two new variables that does not occur in $\phi$.

Now we consider the following case :
\begin{itemize}
\item $\phi$ is not satisfiable, then the $4$TA-SAT algorithm would find only
  $1$ case where $\phi^\prime$ is satisfiable, where $x = true$ and $y = true$,
  then there would be only $1$ satisfiable formula and the $4$TA-SAT would
  return $false$
\item only one interpretation satisfies $\phi$, then the $4$TA-SAT would find
  $4$ different ways of satisfying $\phi$. $I$ is the interpretation that
  satisfies $\phi$, then $\phi^\prime$ would be satisfied by the four following
  interpretation :
  \begin{itemize}
  \item $I^\prime = I \cup (\{x \mapsto false, y \mapsto false\})$
  \item $I^\prime = I \cup (\{x \mapsto false, y \mapsto true\})$
  \item $I^\prime = I \cup (\{x \mapsto true, y \mapsto false\})$
  \item $I^\prime = I \cup (\{x \mapsto true, y \mapsto true\})$
  \end{itemize}
\end{itemize}

The reduction of $\phi$ to $\phi^\prime$ is in polynomial time because we just
add two variables.

\section*{Exercise 2}

In order to show that \N is \NPcp, we first show that \N is in \NPc, then we
proof that the $3$-SAT is reducible to \N in a polynomial time. Because $3$-SAT
is \NPcp, therefore \N would be \NPcp too.

\subsection*{\N is in $\mathcal{NP}$}

\N is in $\mathcal{NP}$ since we can every clause in polynomial time with a 
non-deterministic Turing Machine, we check if the interpretation is satisfiable
(polynomial time) and if the interpretation does not contain a clause where all
of her literals are $true$ (polynomial time).

\subsection*{\N is \NPcp}

In order to reduce $3$-SAT to \N, we transform a $3$-SAT formula $\phi$ into an
equivalent ones which is accepted by the \N problem.

For each clause $(\alpha_i \vee \alpha_j \vee \alpha_k)$ of $\phi$, we represent it by a new \N
clause $(\beta_i,\beta_j,\beta_k,\lambda)$ which is the input of the \N algorithm.

Each $\alpha_i$ variable of $\phi$ is represented by a variable $\beta_i$ in the
transformed formula $\phi^\prime$. $\alpha_i$ would be $true$ if $\beta_i \neq \lambda$ and false
otherwise. The clause $(\beta_i,\beta_j,\beta_k,\lambda)$ is satisfied only if one of the
$\beta$ is different from $\lambda$, so if we found a satisfying interpretation of
$\phi^\prime$, the $\lambda$ variable would make it correct w.r.t the \N constraint.

Then, $\phi$ is satisfiable if and only if $phi^\prime$ is satisfiable, because
$\lambda$ is shared with all the clauses of \N.

The transformation is polynomial because we just use the same variable of each
clause and add a new one (always the same).

\end{document}

%%% Local Variables:
%%% mode: latex
%%% End:
