\documentclass[a4paper,11pt]{report}

\usepackage{amsmath}
\usepackage{fullpage}
\usepackage{bussproofs}
\usepackage{mathpartir}
\usepackage{prooftrees}
\usepackage{color}

\author{Sylvain Julmy}
\date{\today}

\setlength{\parindent}{0pt}

\begin{document}

\begin{center}
  \large{
    Formal Methods\\
    Fall 2017
  }
  
  \noindent\makebox[\linewidth]{\rule{\linewidth}{0.4pt}}
  S02 : Hoare Logic and Propositional Logic
  \noindent\makebox[\linewidth]{\rule{\linewidth}{0.4pt}}

  \begin{flushleft}
    Professor : Ultes-Nitsche Ulrich

    Assistant : Christophe Stammet
  \end{flushleft}

  
  \noindent\makebox[\linewidth]{\rule{\linewidth}{0.4pt}}

  Submitted by Sylvain Julmy
  
  \noindent\makebox[\linewidth]{\rule{\textwidth}{1pt}}
\end{center}

\section*{Exercise 2.1 : }

In order to check the satisfiability of a formula, we just have to demonstrate
his validity.

\subsection*{(a)}
We have to find an interpretation $I$ such that 
\[
  I \models (A \to B) \leftrightarrow (\neg B \to \neg A)
\]

Using the inference rules, we get

\begin{forest}
  [$I \models (A \to B) \leftrightarrow (\neg B \to \neg A)$
  [$I \models (A \to B) \wedge (\neg B \to \neg A)$
    [$\begin{gathered}I \models (A \to B) \\ I \models (\neg B \to \neg
      A)\end{gathered}$
    [$\begin{gathered} I \not\models A \\ I \models (\neg B \to \neg
      A) \end{gathered}$
    [$\begin{gathered}I \not\models A \\ I \not\models \neg B \end{gathered}$
    [$\begin{gathered}I \not\models A \\ I \models B\end{gathered}$]
    ]
    [$\begin{gathered}I \not\models A \\ I \models \neg A \end{gathered}$
    [$\begin{gathered}I \not\models A \\ I \not\models  A\end{gathered}$
    [$I \not\models A$]
    ]
    ]
    ]
    [$\begin{gathered} I \models B \\ I \models (\neg B \to \neg
      A) \end{gathered}$
    [$\begin{gathered} I \models B \\ I \not\models \neg B \end{gathered}$
    [$\begin{gathered}I \models B \\ I \models B \end{gathered}$
    [$I \models B$]
    ]
    ]
    [$\begin{gathered} I \models B \\ I \models \neg A \end{gathered}$
    [$\begin{gathered}I \models B \\ I \not\models A\end{gathered}$]
    ]
    ]
    ]
  ]
  [$I \not\models (A \to B) \vee (\neg B \to \neg A)$
  [$\begin{gathered}I \not\models (A \to B) \\ I \not\models (\neg B \to \neg
    A)\end{gathered}$
  [$\begin{gathered}I \models A \\ I \not\models B \\ I \not\models (\neg B \to
    \neg A)\end{gathered}$
  [$\begin{gathered}I \models A \\ I \not\models B \\ I \models \neg B \\ I
    \not\models \neg A\end{gathered}$
  [$\begin{gathered} I \models A \\ I \not\models B \\ I \not\models B \\ I
    \models A \end{gathered}$
  [$\begin{gathered} I \models A \\ I \not\models B \end{gathered}$]
  ]
  ]
  ]
  ]
  ]
  ]
\end{forest}

And by unifiying the lefting formula, we have
$$
\begin{matrix} I \not\models A \\ I \models B \end{matrix}\ \bigg| \  I \not\models A
\ \bigg| \ I \models B \ \bigg| \ \begin{matrix}I \models B \\ I \not\models
  A \end{matrix} \ \bigg| \ \begin{matrix}I \models A \\ I \not\models B\end{matrix}
$$

we have $5$ ways to find an interpretation $I$ that satisfies the base
formula. Because we have $I \models A$ and $I \not\models A$, $A$ could take any
value $true$ or $false$ and satisfies the formula, and it is the same for $B$,
so the formula is satisfiable for any value for $A$ and $B$ so the formula is
valid.

To verify this, we could use the truth table of the formula and find out that
the result is always $1$ :

\begin{center}
\begin{tabular}{@{ }c@{ }@{ }c | c@{ }@{}c@{}@{ }c@{ }@{ }c@{ }@{ }c@{ }@{}c@{}@{ }c@{ }@{}c@{}@{ }c@{ }@{ }c@{ }@{ }c@{ }@{ }c@{ }@{ }c@{ }@{}c@{}@{ }c}
a & b &  & ( & a & $\rightarrow$ & b & ) & $\leftrightarrow$ & ( & $\neg$ & b & $\rightarrow$ & $\neg$ & a & ) & \\
\hline 
T & T &  &  & T & T & T &  & \textcolor{red}{T} &  & F & T & T & F & T &  & \\
T & F &  &  & T & F & F &  & \textcolor{red}{T} &  & T & F & F & F & T &  & \\
F & T &  &  & F & T & T &  & \textcolor{red}{T} &  & F & T & T & T & F &  & \\
F & F &  &  & F & T & F &  & \textcolor{red}{T} &  & T & F & T & T & F &  & \\
\end{tabular}
\end{center}

\subsection*{(b)}

We have to find an interpretation $I$ such that 
\[
  I \models (A \vee B) \to ( A \wedge B)
\]

Using the inference rules, we get

\begin{forest}
  [$I \models (A \vee B) \to ( A \wedge B)$
  [$\begin{gathered} I \not\models (A \vee B) \end{gathered}$
  [$\begin{gathered} I \not\models A \\ I \not\models B \end{gathered}$]
  ]
  [$\begin{gathered} I \models (A \wedge B) \end{gathered}$
  [$\begin{gathered} I \models A \\ I \models B \end{gathered}$]
  ]
  ]
\end{forest}

So the formula is satisfies for $I : \{A \mapsto true, B \mapsto true \}$ or
$I : \{ A \mapsto false, B \mapsto false \}$, but it is not valid, for example,
the interpretation $I : \{ A \mapsto true, B \mapsto false \}$ do not satisfies
$(A \vee B) \to (A \wedge B)$.

\section*{Exercise 2.2 :}

\subsection*{(a)}

\paragraph*{NNF:}

\begin{align*}
  \neg ((\neg P \vee Q) \to \neg R) &= \neg(\neg (\neg P \vee Q) \vee \neg R)\\
                                    &= \neg((\neg\neg P \wedge \neg Q) \vee \neg R)\\
                                    &= \neg((P \wedge \neg Q) \vee \neg R)\\
                                    &= \neg (P \wedge \neg Q) \wedge \neg \neg R \\
                                    &= \neg (P \wedge \neg Q) \wedge R \\
                                    &= (\neg P \vee \neg \neg Q) \wedge R \\
                                    &= (\neg P \vee Q) \wedge R
\end{align*}

\paragraph*{CNF:}
$$
(\neg P \vee Q) \wedge R
$$

\paragraph*{DNF:}
$$
(\neg P \vee Q) \wedge R = (\neg P \wedge R) \vee (Q \wedge R)
$$

\subsection*{(b)}

\begin{align*}
  ((P \wedge Q) \to (Q \to (P \wedge Q))) \wedge P &= (( \neg(P \wedge Q)) \vee (Q \to (P \wedge Q))) \wedge P \\
                                                   &= ((\neg P \vee \neg Q) \vee (\neg Q \vee (P \wedge Q))) \wedge P\\
                                                   &= ((\neg P \vee \neg Q) \wedge P) \vee ((\neg Q \vee (P \wedge Q)) \wedge P)\\
                                                   &= \underbrace{(\neg P \wedge P)}_{false} \vee (\neg Q \wedge P) \vee ((\neg Q \vee (P \wedge Q)) \wedge P)\\
                                                   &= (\neg Q \wedge P) \vee (\neg Q \wedge P) \vee (P \wedge Q)\\
                                                   &= (\neg Q \wedge P) \vee (P \wedge Q)\\
                                                   &= \underbrace{(\neg Q \vee Q)}_{true} \wedge P\\
                                                   &= P
\end{align*}

$P$ is in NNF, CNF and DNF form.

\end{document}

%%% Local Variables:
%%% mode: latex
%%% End: