\documentclass[a4paper,11pt]{report}

\usepackage{amsmath}
\usepackage{fullpage}

\author{Sylvain Julmy}
\date{\today}

\setlength{\parindent}{0pt}

\begin{document}

\begin{center}
  \Large{
    Functionnal and Logic Programming\
    Fall 2017
  }
  
  \noindent\makebox[\linewidth]{\rule{\linewidth}{0.4pt}}
  S01 : Hoare Logic
  \noindent\makebox[\linewidth]{\rule{\linewidth}{0.4pt}}

  \begin{flushleft}
    Professor : Ultes-Nitsche Ulrich

    Assistant : Christophe Stammet
  \end{flushleft}

  \noindent\makebox[\linewidth]{\rule{\textwidth}{1pt}}
\end{center}

\section*{Exerice 1 : Complete Hoare Triple}

\begin{enumerate}
\item $\{true\} \ y = 25 \ \{y = 25\}$
\item $\{x \leq 6\} \ y = 6 \ \{y \geq x\}$
\item $\{x=2\} \ x = x - 4 \ \{x=-2\}$
\item $\{int\ x \vee int\ y\} \ x=16;y=2;while(x > 3)\{x = \frac{x}{y}\} \ \{???\}$
\item $\{???\} \ if(x \equiv 0 \mod 2)\{y=2x\} else \{x=y\} \ \{???\}$
\item $\{x=5 \wedge int\ x \} \ x = x-2; y = x; x=y-x; \ \{???\}$
\end{enumerate}

\section*{Exerice 2 : Weird Hoare Triple}

\section*{Exerice 3 : Formal Proof of Hoare Triple : if clause}

\section*{Exerice 3 : Formal Proof of Hoare Triple : while loop}

\end{document}

%%% Local Variables:
%%% TeX-command-extra-options: "-shell-escape"
%%% mode: latex
%%% End: