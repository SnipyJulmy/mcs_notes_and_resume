\documentclass[a4paper,11pt]{report}

\usepackage{amsmath}
\usepackage{fullpage}

\author{Sylvain Julmy}
\date{\today}

\setlength{\parindent}{0pt}

\begin{document}

\begin{center}
  \large{
    Formal Methods\\
    Fall 2017
  }
  
  \noindent\makebox[\linewidth]{\rule{\linewidth}{0.4pt}}
  S01 : Hoare Logic
  \noindent\makebox[\linewidth]{\rule{\linewidth}{0.4pt}}

  \begin{flushleft}
    Professor : Ultes-Nitsche Ulrich

    Assistant : Christophe Stammet
  \end{flushleft}

  \noindent\makebox[\linewidth]{\rule{\linewidth}{0.4pt}}

  Submitted by Sylvain Julmy
  
  \noindent\makebox[\linewidth]{\rule{\textwidth}{1pt}}
\end{center}

\section*{Exercise 1 : Complete Hoare Triple}

\begin{enumerate}
\item $\{true\} \ y = 25 \ \{y = 25\}$ : if $y$ it is assigned with $25$, then $y$
  must be equal to $25$.
\item $\{x \leq 6\} \ y = 6 \ \{y \geq x\}$ : if $y$ it is assigned to $6$ and then
  $y$ is greater or equals to $x$, $x$ must be less or equals to $6$ so we would
  have $x \leq 6 \vee y \geq x \vee y = 6$ evaluated to $true$.
\item $\{x=2\} \ x = x - 4 \ \{x=-2\}$ : if the precondition is $x=2$ and the
  post-condition is $x=-2$, we have to find a sequence of statements that
  transforms $2$ to $-2$.
\item $\{int\ x \wedge int\ y\} \ x=16;y=2;while(x > 3)\{x = \frac{x}{y}\} \ \{x'
  \leq 3\}$ : the post-condition is equivalent to the negation of the
  loop condition.
\item $\{x = 3\} \ if\ (x \equiv 0 \mod 2)\{y=2x\}\ else\ \{x=y\} \ \{x' = y\}$ :
  Because of the pre-condition $P$, we have $P \implies x = 3$, so the $true$
  branch of the \textit{if} will never happen so we can reduce the body of $S$
  to $x = y$. Finally, we can put the post-condition $Q$ to $x = y$.
\item $\{x=5 \wedge int\ y \} \ x = x-2; y = x; x=y-x; \ \{x' = 0\}$ : here we
  assume that the precondition $x=5$ implies that the $x$ variable is an
  integer too. Due to $y = x$, the last statement $x = y - x$ is equivalent to
  $x = x - x$, because $y = x$. So we would have $x = x - x = x' = 0$.
\end{enumerate}

\section*{Exercise 2 : Weird Hoare Triple}

\subsection*{(1)}

Using the following transformation $\{P\}S\{Q\} \rightarrow P \wedge \Phi_S
\implies Q$, we can translate the Hoare Triple :
$$
\{int\ x \wedge int\ y\}P\{true\}
$$

to

$$
int\ x \wedge int\ y \wedge P \implies true
$$

Because $Q = true$, we can write any program $P$ so that the Hoare Triple is
valid, provided that $P$ terminates.

So $P$ could be, for example $$y = 2x; x = 2y; y = y / 2;$$.

\subsection*{(2)}

Using the same transformation as before, we can translate the Hoare Triple :
$$
\{int\ x \wedge int\ y\} P \{false\}
$$

to

$$
int\ x \wedge int\ y \wedge P \implies false
$$

In order to obtain a \textbf{valid} Hoare Triple, we have to demonstrate that
the program $P$ we are going to write will never terminates. If $P$ terminates,
the post-condition will be evaluated to $false$ and so the Hoare Triple would
not be a valid one.

For example, the following Hoare Triple is valid because $P$ will never
terminate :

$$
\{int\ x \wedge int\ y\}\ while(true)\ \{skip;\} \{false\}
$$

\section*{Exercise 3 : Formal Proof of Hoare Triple : if clause}

\begin{align*}
  & \{int\ a \wedge int\ b \wedge b > 0\} \\
  & if(a < 0)\ \{a = 2a\} \\
  & else\ \{a = b\} \\
  & \{b \geq a\}
\end{align*}

In order to prove the previous Hoare Triple, we have to prove the two following
ones due to the If-Then-Else construction.

\begin{equation}
\begin{gathered}
  \{int\ a \wedge int\ b \wedge b > 0 \wedge a < 0\}\ a=2a;\ \{b \geq a\} \\
  = \\
  int\ a \wedge int\ b \wedge b > 0 \wedge a < 0 \wedge a'=2a \implies b \geq a'
\end{gathered}
\end{equation}

and

\begin{equation}
\begin{gathered}
  \{int\ a \wedge int\ b \wedge b > 0 \wedge a \geq 0\}\ a=b;\ \{b \geq a\} \\
  = \\
  int\ a \wedge int\ b \wedge b > 0 \wedge a \geq 0 \wedge a' = b \implies b \geq a'
\end{gathered}
\end{equation}

\subsection*{Proving $(1)$}

Because of $a < 0$, multiplying $a$ by $2$ using $a' = 2a$ will always implies
$a' < 0$, $a'$ will never become positive due to the usage of mathematical
integers, no ``computer'' ones which are cyclic. Therefore, $b \geq a'$ will
always be true if and only if $int\ a \wedge int\ b \wedge b > 0 \wedge a < 0$
is true and after executing $a=2a$.

\subsection*{Proving $(2)$}

Because of $b > 0$, applying $a' = b$ will always imply $a' > 0 \wedge a' =
b$. Because $a' = b \implies b \geq a'$, $b \geq a'$ will always be true if and
only if $int\ a \wedge int\ b \wedge b > 0 \wedge a \geq 0$ is true and after
executing $a = b$.

\section*{Exercise 4 : Formal Proof of Hoare Triple : while loop}

\begin{equation}
\label{eqn:ex4-hoare-triple}
\begin{aligned}
  & \{int\ n \wedge n > 0 \wedge int\ x \wedge x > 0\} \\
  & i = 0;\\
  & power=1;\\
  & while(i < n)\{\\
  & \quad power = power * x\\
  & \quad i = i + 1\\
  & \}\\
  & \{power = x^n\}
\end{aligned}
\end{equation}

In order to prove the previous Hoare Triple, we have to prove the $\text{\textit{total
    correctness}} = \text{\textit{partial correctness}} +
\text{\textit{termination}}$.

\subsection*{Proving \textit{Termination}}

In order to prove the termination of~\eqref{eqn:ex4-hoare-triple}, we need to
find a variant $var$ which is a non-negative integer expression that is
decreased by $1$ in each execution of the loop body and cannot go below $0$.

We transform~\eqref{eqn:ex4-hoare-triple} into
\[
  \{int\ var \wedge var > 0\}\ power = power*x;\ i = i + 1;\ \{var > var' \geq 0\}
\]

where

\[
  var = n - i
\]

We know that $int\ n \wedge n > 0 \wedge i = 0 \wedge int\ i$ ($i = 0 \implies
int\ i$, then $int\ (n-i) \wedge (n-i) > 0$ is true, so the pre-condition is
fulfilled.

Now we transform the previous Hoare Triple into
\[
  int\ (n-i) \wedge (n-i) > 0 \wedge power' = power * x \wedge i' = i + 1
  \implies n - i > n - i' \geq 0
\]

Due to $n - i > 0$ and $i' = i + 1$, we would have $n - i' = n - (i + 1) \geq 0$
since the lowest value greater than $0$ is $1$, $n - i + 1$ could,at the lowest,
be $1$, therefore $n - i' \geq 0$ is true.

Due to $n > 0 \wedge int\ n \wedge int\ i \wedge n-i > 0$, we know that $n - i >
n - i + 1$. So we have proved that~\eqref{eqn:ex4-hoare-triple} is terminating.

\subsection*{Proving \textit{Partial Correctness}}

We have to find a loop invariant $inv$ that is true at the following points :
\begin{itemize}
\item before the loop
\item before each execution of the loop body
\item after each execution of the loop body
\item after the loop
\end{itemize}

Then, we could transform the equation \ref{eqn:ex4-hoare-triple} to

\begin{align}
  & \{int\ n \wedge n > 0 \wedge int\ x \wedge x > 0\}\ i=0;\ power=1;\ \{inv\} \\
  & \{inv \wedge i < n \}\ power = power * x;\ i = i + 1;\ \{inv\} \\
  & \{inv \wedge \neg(i < n)\}\ skip;\ \{power = x^n\}
\end{align}  

where

$$
inv = (power = x^i)
$$

and we will prove the Hoare Triple $(4)$, $(5)$ and $(6)$ in order to
demonstrate the partial correctness of \ref{eqn:ex4-hoare-triple}.

\subsubsection*{Proving $(4)$}

To prove that $(4)$ is true, we transform it into
$$
int\ n \wedge n > 0 \wedge int\ x \wedge x > 0 \wedge i = 0 \wedge power = 1
\implies power = x^i
$$

Due to $i=0$, $power = 1$ and $x > 0$, $x^i = x^0 = 1 = power$, so $(4)$ is true.

\subsubsection*{Proving $(5)$}

To prove that $(5)$ is true, we transform it into
\[
  power = x^i \wedge i > n \wedge power' = power * x \wedge i' = i + 1 \implies
  power' = x^{i'}
\]

Due to $power = x^i$, executing $power' = power * x$ is the same as $power' =
x^i * x = x^{i+1}$ and due to $i' = i + 1$, the invariant $power' = x^{i'} =
x^{i + 1}$ is true.

\subsubsection*{Proving $(6)$}

To prove that $(6)$ is true, we transform it into
\[
  power = x^i \wedge i \geq n \implies power = x^n
\]

We know that the variable $i$ would be equals to $n$ after the loop : $i = 0
\implies int\ i$, $i$ is increased by $1$ only and the loop's end when $i \geq
n$.

Finally, we have $(i = n \implies i \geq n) \wedge (x^i = x^n = power)$, so we
have demonstrated the partial correctness of~\eqref{eqn:ex4-hoare-triple}.

\end{document}

%%% Local Variables:
%%% TeX-command-extra-options: "-shell-escape"
%%% mode: latex
%%% End: