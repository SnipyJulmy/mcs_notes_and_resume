\documentclass[a4paper,11pt]{report}

\usepackage{amsmath}
\usepackage{fullpage}
\usepackage[cache=false]{minted}

\usemintedstyle{tango}

\newminted[promelacode]{C}{
  frame=single,
  framesep=6pt,
  breaklines=true,
  fontsize=\scriptsize
}

\newmintinline[promelainline]{C}{breaklines=true,fontsize=\small}

\author{Sylvain Julmy}
\date{\today}

\setlength{\parindent}{0pt}
\setlength{\parskip}{2.5pt}

\begin{document}

\begin{center}
\Large{
    Verification of Cyber-Physical System\\
    Fall 2017
  }
  
  \noindent\makebox[\linewidth]{\rule{\linewidth}{0.4pt}}
  Exercice Sheet 1

  \vspace*{1.4cm}

  Author : Sylvain Julmy
  \noindent\makebox[\linewidth]{\rule{\linewidth}{0.4pt}}

  \begin{flushleft}
    Professor : Ultes-Nitsche Ulrich
    
    Assistant : Prisca Dotti
  \end{flushleft}

  \noindent\makebox[\linewidth]{\rule{\textwidth}{1pt}}
\end{center}

\section*{Exercice 1}

The following \textit{Promela} model contains two processes and a global
variable (named \textbf{global}) that is initialised to $0$.

\begin{listing}[H]
\centering
\begin{promelacode}
byte global = 0;

/*
 * Indefinitely increase the "global" variable if it is lower than 255
 */
active proctype incProcess() {
    repeat :
    // If the assertion fail, it indicates that the "global" variable have reaches 255
    assert (global != 255);
    global < 255;
    global++;
    goto repeat;
}

/*
 * Indefinitely decrease the "global" variable if it is greater than 0
 */
active proctype decProcess() {
    repeat :
    // If the assertion fail, it indicates that the "global" variable have reaches 255
    assert (global != 255);
    global > 0;
    global--;
    goto repeat;
}
\end{promelacode}
\end{listing}

The first process increases the variable by $1$ if it is lower than $255$ and the
second decrease the variable by $1$ if it is greater than $0$. Both processes are
running indefinitely.

In order to use the \textit{Spin} verifier to find out if the variable can reach
the value $255$, we use the assertion \promelainline|assert(global != 255)|. We
check the \textbf{assertion violations} property of \textit{Spin} to find out if
the \textbf{global} variable reaches $255$ or not. In our case, \textit{Spin}
find that the assertion is violated, so we know that the \textbf{global}
variable can reach $255$.

\section*{Exercice 2}

The following \textit{Promela} model is equivalent to the one given for the
exercise $2$, except it does not use \promelainline|if|,\promelainline|->| or
\promelainline|goto| statements :

\begin{listing}[H]
\centering
\begin{promelacode}
mtype = {P,C};

mtype turn = P;

active proctype producer() {
    /* do-notation : defines a loop */

    // Guards : wait until (turn == P) because it is the only available statement
    do :: turn == P ;
        printf("Produce\n");
        turn = C;
    od
}

active proctype consumer() {
    /* do-notation : defines a loop */

    // Guards : wait until (turn == C) because it is the only available statement
    do :: turn == C;
        printf("Consume\n");
        turn = P;
    od
}
\end{promelacode}
\end{listing}



\section*{Exercice 3}

When looking for non-progress cycles :

\paragraph*{Does Spin detect an error when using the \textit{week fairness}
  constraint ?} No

\paragraph*{Does Spin detect an error when is not using the \textit{week
    fairness} constraint ?} No

It looks like that Spin does not detect any non-progress cycles with the given
code :

\begin{listing}[H]
\centering
\begin{promelacode}
byte x = 2;

active proctype A(){
    do :: x = 3 - x; progress : skip; od
}

active proctype B(){
    do :: x = 3 - x; progress : skip; od
}
\end{promelacode}
\end{listing}

By the way, if we remove one progress label (either from proc A or proc
B), Spin does detect a non-progress cycle when not using \textit{week fairness}
constraint:

\begin{listing}[H]
\centering
\begin{promelacode}
byte x = 2;

active proctype A(){
    do :: x = 3 - x; od
}

active proctype B(){
    do :: x = 3 - x; progress : skip; od
}
\end{promelacode}
\end{listing}

In order to prove the fairness for both process, we need to use an additional
process called \textit{monitor} which is going to monitor both process $A$
and $B$ using the \promelainline|progress :| label of Spin. Both processes will
use a flag in order to indicates that they have fullfiled their job and are
waiting. The monitor process wait until both $A$ and $B$ process have completed
their work to launch them again, so we have proved the fairness for both processes.

Examples using a \textit{monitor} :

\begin{listing}[H]
\centering
\begin{promelacode}
byte x = 2;
byte checkProgress[4];

active proctype A(){
       do ::
           checkProgress[0]==0;
           x = 3 - x;
           checkProgress[0]++;
       od
}

active proctype B(){
       do ::
           checkProgress[1]==0;
           x = 3 - x;
           checkProgress[1]++;
       od
}

active proctype monitor(){
       repeat :
       checkProgress[0] + checkProgress[1] == 2;
       progress :
       checkProgress[0] = 0;
       checkProgress[1] = 0;
       goto repeat;
}
\end{promelacode}
\end{listing}

\end{document}

%%% Local Variables:
%%% TeX-command-extra-options: "-shell-escape"
%%% mode: latex
%%% End: