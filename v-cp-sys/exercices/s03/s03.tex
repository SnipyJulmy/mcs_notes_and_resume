\documentclass[a4paper,11pt]{report}

\usepackage{amsmath,amssymb}
\usepackage{fullpage}
\usepackage{graphicx}
\usepackage[cache=false]{minted}

\usemintedstyle{tango}

\newminted[promelacode]{C}{
  frame=single,
  framesep=6pt,
  breaklines=true,
  fontsize=\scriptsize
}

\newmintinline[promelainline]{C}{breaklines=true,fontsize=\small}

\newmintedfile{c}{frame=single, framesep=6pt, breaklines=true,fontsize=\scriptsize}
\newcommand{\ex}[3]{\cfile[firstline=#1,lastline=#2]{ex#3.pml}}

% tikz
\usepackage{tikz}
\usetikzlibrary{snakes}

\author{Sylvain Julmy}
\date{\today}

\setlength{\parindent}{0pt}
\setlength{\parskip}{2.5pt}

\begin{document}

\begin{center}
\Large{
    Verification of Cyber-Physical System\\
    Fall 2017
  }
  
  \noindent\makebox[\linewidth]{\rule{\linewidth}{0.4pt}}
  Exercice Sheet 2

  \vspace*{1.4cm}

  Author : Sylvain Julmy
  \noindent\makebox[\linewidth]{\rule{\linewidth}{0.4pt}}

  \begin{flushleft}
    Professor : Ultes-Nitsche Ulrich
    
    Assistant : Prisca Dotti
  \end{flushleft}

  \noindent\makebox[\linewidth]{\rule{\textwidth}{1pt}}
\end{center}

\section*{Exercice 1}

The following \textit{Promela} model is equivalent to the one given in exercice:

\ex{8}{42}{1}

we use a generic \textit{Promela} process to simulate two process which want to
access a critical section represented by the \promelainline|critical:| label. We
use a never claim too : if both process are at the \promelainline|critical:|
label, the never claim is allowed to end, which should not be permitted by a
valid model.

\section*{Exercice 2}

The mutual exclusion problem states that any of the process $A$ and $B$ are at
the same time in the critical section. We can translate this to the $LTL$
formula $f = \square (\neg inCS_A \wedge \neg inCS_b)$, where $inCS_p$ denots
``process p is in critical section''. Because that is a
behavior of the system which should never happen, we use the following spin
command to generate the never claim :

\begin{verbatim}
spin -f '!( []!(p && q) )'
\end{verbatim}

which give us the following never claim :

\begin{promelacode}
  never  {    /* !( []!(p && q) ) */
    T0_init:
    do
    :: atomic { ((p && q)) -> assert(!((p && q))) }
    :: (1) -> goto T0_init
    od;

    accept_all:
    skip
  }
\end{promelacode}

Finally we would have the following promela model :

\ex{8}{52}{2}

We can also check manually the $LTL$ formula by forcing $p$ and $q$ to $true$ :

\begin{promelacode}
/* LTL formula and variable */

#define p true
#define q true
\end{promelacode}

Which leads into an error on the verification of the \textit{Promela} model.

\section*{Exercice 3}

The \textit{Promela} model given in exercice $3$ is corresponding to the $LTL$
formula $f = \diamondsuit \square p$, which means, with respect to the never
claim, ``It should never happen that, at the moment, $p$ will always be $true$
in the future''.

Its like the following sequence should never happen :
\[
  (1) , (0) , (0) , (1) , (0) , \underbrace{(1)}_{a} , (\cdots) , (1) , (\cdots)
\]

because at the moment $a$, $p$ never change back to $false$.

\end{document}
