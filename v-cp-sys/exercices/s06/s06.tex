\documentclass[a4paper,11pt]{report}

\usepackage{amsmath,amssymb}
\usepackage{fullpage}
\usepackage{graphicx}
\usepackage[cache=false]{minted}

\usepackage{bussproofs}
\usepackage{mathpartir}
\usepackage{prooftrees}
\usepackage{color}

\usepackage{tikz}
\usetikzlibrary{automata,positioning}

\newcommand*\circled[1]{\tikz[baseline=(char.base)]{
            \node[shape=circle,draw,inner sep=2pt] (char) {#1};}}

\makeatletter
\pgfmathdeclarefunction{alpha}{1}{%
  \pgfmathint@{#1}%
  \edef\pgfmathresult{\pgffor@alpha{\pgfmathresult}}%
}

\newcommand*{\until}{U}
\newcommand*{\disj}{\ ,\ }
\newcommand*{\A}{\square}  % Always
\newcommand*{\D}{\diamondsuit} % eventually

\newcommand*{\Pq}{(\top,\bot)}
\newcommand*{\pQ}{(\bot,\top)}
\newcommand*{\PQ}{(\top,\top)}
\newcommand*{\pq}{(\bot,\bot)}

\usemintedstyle{tango}

\newminted[promelacode]{C}{
  frame=single,
  framesep=6pt,
  breaklines=true,
  fontsize=\scriptsize
}

\newmintinline[promelainline]{C}{breaklines=true,fontsize=\small}

\newmintedfile{c}{frame=single, framesep=6pt, breaklines=true,fontsize=\scriptsize}
\newcommand{\ex}[3]{\cfile[firstline=#1,lastline=#2]{ex#3.pml}}

% tikz
\usepackage{tikz}
\usetikzlibrary{snakes}

\author{Sylvain Julmy}
\date{\today}

\setlength{\parindent}{0pt}
\setlength{\parskip}{2.5pt}

\begin{document}

\begin{center}
\Large{
    Verification of Cyber-Physical System\\
    Fall 2017
  }
  
  \noindent\makebox[\linewidth]{\rule{\linewidth}{0.4pt}}
  Exercice Sheet 6

  \vspace*{1.4cm}

  Author : Sylvain Julmy
  \noindent\makebox[\linewidth]{\rule{\linewidth}{0.4pt}}

  \begin{flushleft}
    Professor : Ultes-Nitsche Ulrich
    
    Assistant : Prisca Dotti
  \end{flushleft}

  \noindent\makebox[\linewidth]{\rule{\textwidth}{1pt}}
\end{center}

\section*{Exercice 1}

\subsection*{(1)}

The task $A_x$ is always executable, but if $x$ is even, it may be possible that
$A_x$ would never change again and only $A_y$ is going to be executed. So $A_x$
is not strongly fair.

That is an extreme case, if we assume that the extreme case would not appear,
$A_x$ is strongly fair, because it would always be executable and would always
be executed in the future.

\subsection*{(2)}

The task $A_y$ would always be executable, but it may be possible that only
$A_x$ is going to be executed so $A_Y$ would never be executed again.

That is an extreme case, if we assume that the extreme case would not appear,
$A_y$ would always be executable and would always be executed in the future.

\subsection*{(3)}

The execution is weakly fair with respect to the task $A_x$, because $A_x$ would
always be executable, and (by assuming the extreme case won't appear) going to
be executed.

\subsection*{(4)}

The execution is weakly fair with respect to the task $A_y$, because $A_y$ would
always be executable, and (by assuming the extreme case won't appear) going to
be executed.

\section*{Exercice 2}

\subsection*{(1)}

There is no garantee that the value of $x$ would eventually exceeds $5$ (the
system could always execute the task $Process_2$ and never $Process_1$). In
order to garantee that, we need to add a strong fairness for the task
$Process_1$. We don't have to add any kind of fairness to the task $Process_2$
(if we have to, we could add a weak fairness for task $Process_2$).

\subsection*{(2)}

There is no garantee that the value of $x$ would eventually exceeds $5$ (the
system could always execute the task $Process_1$ and never $Process_2$). In
order to garantee that, we need to add a strong fairness for the task
$Process_2$ and a weak fairness to the task $Process_1$. So $Process_1$ would be
executed at least one time and then $y$ would increase in order to exceeds $5$.

\subsection*{(3)}

There is no garantee that the values $x$ and $y$ would become equal in the
execution. In order to garantee that, we have to add a strong fairness
assumption for both task $Process_1$ and $Process_2$. So it would exist an
infinite amount of execution in which $x==y$ would holds at a certain time.


\end{document}