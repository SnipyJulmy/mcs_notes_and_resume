\documentclass[a4paper,11pt]{article}

\usepackage[T1]{fontenc}
\usepackage[english]{babel}
\usepackage{etoolbox}
\usepackage{sourcecodepro}
\usepackage{amsmath}
\usepackage{amsthm}
\usepackage{fullpage}
\usepackage{tcolorbox}

\setlength{\parindent}{0pt}

\title{Resume : Programming Language}
\author{Sylvain Julmy}

\begin{document}

\maketitle

\section{Introduction}

\begin{tcolorbox}[title=What is a Programming Language ?]
A programming language is a notational system for describing computation in a
machine-readable and human-readable form.

- Louden
\end{tcolorbox}

A programming language is a tool for developing executable models for a class of
problem domains.

Generations of programming languages (higher level of abstraction at each level):
\begin{itemize}
\item 1GL : machine codes
\item 2GL : symbolic assemblers
\item 3GL : (machine-independent) imperative languages
\item 4GL : domain specific application generators
\item 5GL : AI languages ...
\end{itemize}

Programming languages differ on
\begin{itemize}
\item \textbf{Common constructs} : basic data types (numbers, etc.); variables;
  expressions; statements; keywords; control constructs; procedures; comments;
  errors ...
\item \textbf{Uncommon constructs} : type declarations; special types (strings, arrays,
  matrices, ...); sequential execution; concurrency constructs;
  packages/modules; objects; general functions; generics; modifiable state; ...
\end{itemize}

A programming language is a problem solving tool and different paradigm are good
at something :
\begin{itemize}
\item \textbf{Imperative style} : $program = algorithms + data$ good for decomposition.
\item \textbf{Functional style} : $program = functions \circ functions$ good for
  reasoning.
\item \textbf{Logic programming style} : $program = facts + rules$ good for searching.
\item \textbf{Object-oriented style} : $program = objects + messages$ good for modeling.
\end{itemize}

\begin{tcolorbox}[title = What\, exactly\, is a programming language ?]
  A programming language is a tool used to describe and solve problems in a
  human and computer readable format.
\end{tcolorbox}
\begin{tcolorbox}[title=How do compilers and interpreters differ ?]
  \begin{itemize}
\item A complier converts the high level instruction into machine language while
  an interpreter converts the high level instruction into an intermediate form.
\item Before execution, entire program is executed by the compiler whereas after
  translating the first line, an interpreter then executes it and so on.
\item List of errors is created by the compiler after the compilation process
  while an interpreter stops translating after the first error.
\item An independent executable file is created by the compiler whereas
  interpreter is required by an interpreted program each time.
\end{itemize}
\end{tcolorbox}
\begin{tcolorbox}[title=Why was FORTRAN developed ?]
  To write programs in conventional mathematical notation, and generate code
  comparable to good assembly programs. Most effort spent on code generation and
  optimization, its easy to learn and promoted by IMB. Innovations :
  \begin{itemize}
  \item Symbolic notation for subroutines and functions
  \item Assignments to variables of complex expressions
  \item DO loops
  \item Comments
  \item Input/output formats
  \item Machine-independence
  \end{itemize}
\end{tcolorbox}
\begin{tcolorbox}[title=What were the main achievements of ALGOL 60?]
  \begin{itemize}
  \item BNF (Backus-Naur Form) introduced to define syntax (led to syntax-directed compilers)
  \item First block-structured language; variables with local scope
  \item Structured control statements
  \item Recursive procedures
  \item Variable size arrays
  \end{itemize}
\end{tcolorbox}
\begin{tcolorbox}[title=Why do we call C a ``Third Generation Language'' ?]
  C is a third generation language because it is completely machine-independent
  (we can code on a computer and compile it to any architecture we want) and its
  imperative.
\end{tcolorbox}
\begin{tcolorbox}[title=What is a ``Fourth Generation Language'' ?]
  A fourth generation language is a language design to solve a very specific
  (area of) problem(s). Like SQL is to specificaly query database.
\end{tcolorbox}
\begin{tcolorbox}[title=Why are there so many programming languages ?]
  They are a lot of different programming languages because there exists a lot
  of different problem to solve and each programming language owns advantages
  and disadvantages to solve some problem or another.
\end{tcolorbox}
\begin{tcolorbox}[title=Why are FORTRAN and COBOL still important programming
  languages ?]
  A lot of business application are written in FORTRAN and in COBOL, making the
  move to another language is not easy.
\end{tcolorbox}
\begin{tcolorbox}[title=Which language should you use to implement a spelling
  checker ?]
  Prolog (DCG)
\end{tcolorbox}
\begin{tcolorbox}[title=Which language should you use to implement a filter to
  translate upper-to-lower case ?]
  Perl and another scripting language
\end{tcolorbox}
\begin{tcolorbox}[title=Which language should you use to implement a theorem
  prover ?]
  Haskell or ML-like languages due to is type system.
\end{tcolorbox}
\begin{tcolorbox}[title=Which language should you use to implement an address
  database ?]
  Object-oriented one and SQL to query it.
\end{tcolorbox}
\begin{tcolorbox}[title=Which language should you use to implement an expert
  system ?]
  C/C++ or Rust ?
\end{tcolorbox}
\begin{tcolorbox}[title=Which language should you use to implement a game server
  for initiating chess games on the internet ?]
  Java or OOP language, like Scala.
\end{tcolorbox}
\begin{tcolorbox}[title=Which language should you use to implement a user
  interface for a network chess client ?]
  TCL or QT, design to build GUI.
\end{tcolorbox}

\section{Stack based programming}

\begin{tcolorbox}[title=What is PostScript ?]
  is a simple interpretive programming language ... to describe the appearance
  of text, graphical shapes, and sampled images on printed or displayed pages.

  \begin{itemize}
  \item introduced in 1985 by Adobe
  \item display standard supported by all major printer vendors
  \item simple, stack-based programming language
  \item minimal syntax
  \item large set of built-in operators
  \item PostScript programs are usually generated from applications, rather than hand-coded
  \end{itemize}
\end{tcolorbox}

A PostScript program is a sequence of tokens, representing typed objects, that
is interpreted to manipulate the display and four stacks that represent the
execution state of a PostScript program:

\begin{itemize}
\item \textbf{Operand stack} : holds (arbitrary) operands and results of PostScript
  operators.
\item \textbf{Dictionnary stack} : holds only dictionaries where keys and values may be
  stored.
\item \textbf{Execution stack} : holds executable objects (e.g. procedures) in stages of
  execution.
\item \textbf{Graphic state stack} : keeps track of current coordinates etc...
\end{itemize}

\section{}

\section{}

\section{}

\section{}

\section{}

\section{}

\section{}

\section{}

\section{}

\section{}

\section{}

\end{document}