\documentclass[a4paper,11pt]{report}

\usepackage{amsmath,amssymb,amsthm}
\usepackage{fullpage}
\usepackage{graphicx}

\usepackage{bussproofs}
\usepackage{mathpartir}
\usepackage{prooftrees}
\usepackage{color}
\usepackage{rotating}



\usepackage{tikz}
\usetikzlibrary{automata,positioning}
\usetikzlibrary{fit}

\newcommand*\circled[1]{\tikz[baseline=(char.base)]{
    \node[shape=circle,draw,inner sep=2pt] (char) {#1};}}

\makeatletter
\pgfmathdeclarefunction{alpha}{1}{%
  \pgfmathint@{#1}%
  \edef\pgfmathresult{\pgffor@alpha{\pgfmathresult}}%
}

\newcommand*{\until}{U}
\newcommand*{\disj}{\ ,\ }
\newcommand*{\A}{\square}  % Always
\newcommand*{\D}{\diamondsuit} % eventually

\newcommand*{\Pq}{(\top,\bot)}
\newcommand*{\pQ}{(\bot,\top)}
\newcommand*{\PQ}{(\top,\top)}
\newcommand*{\pq}{(\bot,\bot)}


% tikz
\usepackage{tikz}
\usetikzlibrary{snakes}


\author{Sylvain Julmy}
\date{\today}

\setlength{\parindent}{0pt}
\setlength{\parskip}{2.5pt}

\newtheorem*{thm}{Theorem}

\begin{document}

\begin{center}
  \Large{
    Automata on Infinite Structure\\
    Fall 2018
  }
  
  \noindent\makebox[\linewidth]{\rule{\linewidth}{0.4pt}}
  Exercice Sheet 7

  \vspace*{1.4cm}

  Author : Sylvain Julmy
  \noindent\makebox[\linewidth]{\rule{\linewidth}{0.4pt}}

  \begin{flushleft}
    Professor : Ultes-Nitsche Ulrich
    
    Assistant : Stammet Christophe
  \end{flushleft}

  \noindent\makebox[\linewidth]{\rule{\textwidth}{1pt}}
\end{center}

\section*{Exercise 1}

\begin{thm}
  $\mathcal{PSPACE}$ is closed under union, intersection and complement.
\end{thm}

\begin{proof}
  Assume $L_1,L_2 \in \mathcal{PSPACE}$, hence there exist TM $M_1$ and $M_2$
  that are using polynomial space, such that $M_1$ decides $L_1$ in
  nondeterministic time $O(n^k)$ and in polynomial space $O(n^{k'})$, and $M_2$
  decides $L_2$ in nondeterministic time $O(n^l)$ and in polynomial space
  $O(n^{l'})$. We show that
  \begin{enumerate}
  \item there exist a decider $M$ in nondeterministic polynomial time and
    polynomial space such that $L(M) = L_1 \cup L_2$.
  \item there exist a decider $M$ in nondeterministic polynomial time and
    polynomial space such that $L(M) = L_1 \cap L_2$.
  \end{enumerate}

  The constructions are standard ones.

  \begin{enumerate}
  \item Intersection : we run the word $\omega$ on $M_1$, if $M_1$ reject it,
    then $M$ reject it too, else we run $\omega$ on $M_2$, if $M_2$ reject it,
    then $M$ reject it too, else $M$ accept $\omega$. Clearly, $M$ is a
    poly-time nondeterministic decider and a poly-space decider for $L_1 \cap
    L_2$, for the input $\omega$ of length $n$, the time complexity is
    $O(n^{max(k,l)})$ and the space complexity is $O(n^{max(k',l')})$.

  \item Union : we run the word $\omega$ on $M_1$, if $M_1$ accept it, then $M$
    accept it too, else we run $\omega$ on $M_2$, if $M_2$ accept it, then $M$
    accept it too, else $M$ reject $\omega$. Clearly again, $M$ is a
    poly-time nondeterministic decider and a poly-space decider for $L_1 \cap
    L_2$, for the input $\omega$ of length $n$, the time complexity is
    $O(n^{max(k,l)})$ and the space complexity is $O(n^{max(k',l')})$. We could
    also choose non-deterministicaly either $M_1$ or $M_2$ and use only the
    selected machine.
  \end{enumerate}

  To prove that $\mathcal{PSPACE}$ is closed under complement, we use the fact
  that every language $L \in \mathcal{PSPACE}$ has a deterministic poly-space TM
  $M$, we can swap the accepting and non-accepting states in $M$ in polynomial
  time and get a poly-space decider for $\bar{L}$, hence
  $\bar{L}\in\mathcal{PSPACE}$.
\end{proof}

\section*{Exercise 2}

\subsection*{$w_1$}

\begin{mathpar}
  q_0 1010 \longrightarrow
  1 q_0 010 \longrightarrow
  10 q_0 10 \longrightarrow
  101 q_0 0 \longrightarrow
  1010 q_0  \longrightarrow
  101 q_1 0 \longrightarrow
  10 q_1 11 \longrightarrow
  101 q_2 1
\end{mathpar}

\subsection*{$w_2$}
\begin{mathpar}
  q_0 0100 \longrightarrow
  0 q_0 100 \longrightarrow
  01 q_0 00 \longrightarrow
  010 q_0 0 \longrightarrow
  0100 q_0  \longrightarrow
  010 q_1 0 \longrightarrow
  01 q_1 01 \longrightarrow
  0 q_1 111 \longrightarrow
  01 q_2 11
\end{mathpar}
  
\end{document}