\documentclass[a4paper,11pt]{report}

\usepackage{fullpage}

\usepackage{amsmath}
\usepackage{bussproofs}
\usepackage{mathpartir}
\usepackage{prooftrees}
\usepackage{color}
\usepackage{hyperref}
\usepackage{placeins}


% for finite state automata
\usepackage{tikz}
\usetikzlibrary{automata,positioning}

%%%%%%%%%%%%%%%%%%%%%%%%%%%%%%%%%%%%%%%%%%%%%%%%%%%%%%%%%
% Minted
%%%%%%%%%%%%%%%%%%%%%%%%%%%%%%%%%%%%%%%%%%%%%%%%%%%%%%%%%

\usepackage[cache=false]{minted}

%%%%%%%%%% C
\newmintinline{c}{
  fontsize=\small,
  breaklines=true
}

\newminted{c}{
  frame=single,
  framesep=2mm,
  fontsize=\scriptsize,
  mathescape
}

\newminted[ccodeline]{c}{
  frame=single,
  framesep=2mm,
  fontsize=\scriptsize,
  mathescape,
  linenos
}

%%%%%%%%% CMAKE
\newminted{cmake}{
  frame=single,
  framesep=2mm,
  fontsize=\scriptsize,
  mathescape,
  linenos,
  breaklines=true
}

% End minted
%%%%%%%%%%%%%%%%%%%%%%%%%%%%%%%%%%%%%%%%%%%%%%%%%%%%%%%%%


\author{Sylvain Julmy}
\date{\today}

\setlength{\parindent}{0pt}

\begin{document}

\begin{center}
  \Large{
    Operating Systems\\
    Spring 2018
  }
  
  \noindent\makebox[\linewidth]{\rule{\linewidth}{0.4pt}}
  S05
  \noindent\makebox[\linewidth]{\rule{\linewidth}{0.4pt}}
    AssistantArgs* args = arguments;
  \begin{flushleft}
    Professor : Philippe Cudré-Mauroux

    Assistant : Ines Arous
  \end{flushleft}
  
  \noindent\makebox[\linewidth]{\rule{\linewidth}{0.4pt}}

  Submitted by Sylvain Julmy
  
  \noindent\makebox[\linewidth]{\rule{\textwidth}{1pt}}
\end{center}

\section*{Exercise 2}

Total memory : $2^{32}$ Bytes.

Page size : $8192 = 2^{13}$ Bytes.

Entries : $2^{32} / 2^{13} = 2^{32-13} = 2^{19} = 524'288$ entries.

Time needed to load the page table : $10^{-9} \cdot 100 \cdot 524'288 = 0.052 =
52$ ms.

If a process get $100$ msec, then $52\%$ of the time is spent on loading page tables.

\section*{Exercise 3}

Page size : $32 - 11 - 9 = 12$ and $2^{12} = 4096 = 4$ KB.
Number of pages : $9 + 11 = 20$ and $2^{20}$ number of pages is available.

\section*{Exercise 4}

\subsection*{1)}

Because the TLB is an hardware device, an entry in the TLB can be overwritten
because changes are save to the memory directly and not in the TLB.

\subsection*{2)}

When the modified bit of a TLB is changed to $1$. The modification has to be
written in the page table entry in memory.

\subsection*{3)}

There are two situations : the TLB context have to be reloaded on every context
switch and the other from the previous answer.

\section*{Exercise 5}

See attachment.

\end{document}
