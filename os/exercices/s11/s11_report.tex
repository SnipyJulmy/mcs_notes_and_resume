\documentclass[a4paper,11pt]{report}

\usepackage{fullpage}

\usepackage{amsmath}
\usepackage{bussproofs}
\usepackage{mathpartir}
\usepackage{prooftrees}
\usepackage{color}
\usepackage{hyperref}
\usepackage{placeins}
\usepackage{tabularx}
\usepackage[normalem]{ulem}
\useunder{\uline}{\ul}{}


% for finite state automata
\usepackage{tikz}
\usetikzlibrary{automata,positioning}

%%%%%%%%%%%%%%%%%%%%%%%%%%%%%%%%%%%%%%%%%%%%%%%%%%%%%%%%%
% Minted
%%%%%%%%%%%%%%%%%%%%%%%%%%%%%%%%%%%%%%%%%%%%%%%%%%%%%%%%%

\usepackage[cache=false]{minted}

%%%%%%%%%% C
\newmintinline{c}{
  fontsize=\small,
  breaklines=true
}

\newminted{c}{
  frame=single,
  framesep=2mm,
  fontsize=\scriptsize,
  mathescape
}

\newminted[ccodeline]{c}{
  frame=single,
  framesep=2mm,
  fontsize=\scriptsize,
  mathescape,
  linenos
}

%%%%%%%%% CMAKE
\newminted{cmake}{
  frame=single,
  framesep=2mm,
  fontsize=\scriptsize,
  mathescape,
  linenos,
  breaklines=true
}

% End minted
%%%%%%%%%%%%%%%%%%%%%%%%%%%%%%%%%%%%%%%%%%%%%%%%%%%%%%%%%


\author{Sylvain Julmy}
\date{\today}

\setlength{\parindent}{0pt}

\begin{document}

\begin{center}
  \Large{
    Operating Systems\\
    Spring 2018
  }
  
  \noindent\makebox[\linewidth]{\rule{\linewidth}{0.4pt}}
  S09
  \noindent\makebox[\linewidth]{\rule{\linewidth}{0.4pt}}
  \begin{flushleft}
    Professor : Philippe Cudré-Mauroux

    Assistant : Ines Arous
  \end{flushleft}
  
  \noindent\makebox[\linewidth]{\rule{\linewidth}{0.4pt}}

  Submitted by Sylvain Julmy
  
  \noindent\makebox[\linewidth]{\rule{\textwidth}{1pt}}
\end{center}

\section*{Exercise 2}

\subsection*{\texttt{a)}}

A Type 1 hypervisor is the only programm that is running in the privileged mode.
A Type 1 hypervisor has to hold multiple copies of the hardware called virtual
machines, these hypervisor  run directly on the host's hardware to control the
hardware and to manage guest operating systems.

A Type 2 hypervisor has to relies on an existent operating system to manage
existing resources.  guest operating system runs as a process on the host.
Type-2 hypervisors abstract guest operating systems from the host operating
system.

\subsection*{\texttt{b)}}

Virtual mahcine don't care about the disk partition, the hypervisor divide the
partition and give to each of its virtual machine one of them.

\section*{Exercise 3}

 Paravirtualization is a virtualization technique that presents to virtual
 machines a software interface, which is similar yet not identical to the
 underlying hardware-software interface.

 The paravirtualization offers a set of hypercalls which allow the guest to
 communicate directly with the hypervisor (like system call on an operating
 system).

\section*{Exercise 4}

\subsection*{\texttt{a)}}

The host PC need to support multiple virtual machines operating system and all
of its applications, as well as the programm of the hypervisor (functions, data
structures, ...). One way to reduce the memory usage is to detect the identical
memory segment and share them among multiple virtual machines. For example, if
multiple virtual machines are running the same OS, we can use only one memory
segment for the OS and share it among multiple virtual machine.

\subsection*{\texttt{b)}}

The idea is to avoid storing the data twice, the technique is to analyze the
memory of each virtual machines on a host and create a hash value, then
identical hash value represent the same memory segment and therefore we can
remove on of them and share the last one among multiple instances.

\end{document}
