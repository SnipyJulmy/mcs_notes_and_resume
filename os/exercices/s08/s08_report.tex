\documentclass[a4paper,11pt]{report}

\usepackage{fullpage}

\usepackage{amsmath}
\usepackage{bussproofs}
\usepackage{mathpartir}
\usepackage{prooftrees}
\usepackage{color}
\usepackage{hyperref}
\usepackage{placeins}


% for finite state automata
\usepackage{tikz}
\usetikzlibrary{automata,positioning}

%%%%%%%%%%%%%%%%%%%%%%%%%%%%%%%%%%%%%%%%%%%%%%%%%%%%%%%%%
% Minted
%%%%%%%%%%%%%%%%%%%%%%%%%%%%%%%%%%%%%%%%%%%%%%%%%%%%%%%%%

\usepackage[cache=false]{minted}

%%%%%%%%%% C
\newmintinline{c}{
  fontsize=\small,
  breaklines=true
}

\newminted{c}{
  frame=single,
  framesep=2mm,
  fontsize=\scriptsize,
  mathescape
}

\newminted[ccodeline]{c}{
  frame=single,
  framesep=2mm,
  fontsize=\scriptsize,
  mathescape,
  linenos
}

%%%%%%%%% CMAKE
\newminted{cmake}{
  frame=single,
  framesep=2mm,
  fontsize=\scriptsize,
  mathescape,
  linenos,
  breaklines=true
}

% End minted
%%%%%%%%%%%%%%%%%%%%%%%%%%%%%%%%%%%%%%%%%%%%%%%%%%%%%%%%%


\author{Sylvain Julmy}
\date{\today}

\setlength{\parindent}{0pt}

\begin{document}

\begin{center}
  \Large{
    Operating Systems\\
    Spring 2018
  }
  
  \noindent\makebox[\linewidth]{\rule{\linewidth}{0.4pt}}
  S08
  \noindent\makebox[\linewidth]{\rule{\linewidth}{0.4pt}}
  \begin{flushleft}
    Professor : Philippe Cudré-Mauroux

    Assistant : Ines Arous
  \end{flushleft}
  
  \noindent\makebox[\linewidth]{\rule{\linewidth}{0.4pt}}

  Submitted by Sylvain Julmy
  
  \noindent\makebox[\linewidth]{\rule{\textwidth}{1pt}}
\end{center}

\section*{Exercise 2}

\paragraph{Which page will NRU replace ?}

NRU will remove page 1 or 2.

\paragraph{Which page will FIFO replace ?}

FIFO will remove page 3, which was loaded at tick $110$.

\paragraph{Which page will second chance replace ?}

LRU will remove page $1$.

\paragraph{Which page will LRU replace ?}

Second chance will remove page $2$.

\section*{Exercise 3}

\subsection*{a)}

All referenced bit are set to $0$.

  \begin{table}[h]
\centering
\caption{Table entries at tick $9$.}
\label{table:ex3-1}
\begin{tabular}{ccccc}
\hline
\multicolumn{1}{|c|}{\textbf{Page}} & \multicolumn{1}{c|}{\textbf{Timestamp}} & \multicolumn{1}{c|}{\textbf{V}} & \multicolumn{1}{c|}{\textbf{R}} & \multicolumn{1}{c|}{\textbf{M}} \\ \hline
0                                   & 6                                       & 1                               & 0                               & 1                               \\
1                                   & 9                                       & 1                               & 1                               & 0                               \\
2                                   & 9                                       & 1                               & 1                               & 1                               \\
3                                   & 7                                       & 1                               & 0                               & 0                               \\
4                                   & 4                                       & 0                               & 0                               & 0                              
\end{tabular}
\end{table}

\begin{table}[h]
\centering
\caption{Table entries at tick $10$.}
\label{table:ex3-2}
\begin{tabular}{ccccc}
\hline
\multicolumn{1}{|c|}{\textbf{Page}} & \multicolumn{1}{c|}{\textbf{Timestamp}} & \multicolumn{1}{c|}{\textbf{V}} & \multicolumn{1}{c|}{\textbf{R}} & \multicolumn{1}{c|}{\textbf{M}} \\ \hline
0                                   & 6                                       & 1                               & 0                               & 1                               \\
1                                   & \color{red}{10}                                      & 1                               & \color{red}{0}                              & 0                               \\
2                                   & \color{red}{10}                                      & 1                               & \color{red}{0}                               & 1                               \\
3                                   & 7                                       & 1                               & 0                               & 0                               \\
4                                   & 4                                       & 0                               & 0                               & 0                              
\end{tabular}
\end{table}

\subsection*{b)}

\begin{table}[h]
\centering
\caption{Table entries at tick $9$.}
\label{table:ex3-3}
\begin{tabular}{ccccc}
\hline
\multicolumn{1}{|c|}{\textbf{Page}} & \multicolumn{1}{c|}{\textbf{Timestamp}} & \multicolumn{1}{c|}{\textbf{V}} & \multicolumn{1}{c|}{\textbf{R}} & \multicolumn{1}{c|}{\textbf{M}} \\ \hline
0                                   & 6                                       & 1                               & 0                               & 1                               \\
1                                   & 9                                       & 1                               & 1                               & 0                               \\
2                                   & 9                                       & 1                               & 1                               & 1                               \\
3                                   & 7                                       & 1                               & 0                               & 0                               \\
4                                   & 4                                       & 0                               & 0                               & 0                              
\end{tabular}
\end{table}

\begin{table}[h]
\centering
\caption{Contents of the new table entries.}
\label{table:ex3-4}
\begin{tabular}{cccccl}
\hline
\multicolumn{1}{|c|}{\textbf{Page}} & \multicolumn{1}{c|}{\textbf{Timestamp}} & \multicolumn{1}{c|}{\textbf{V}} & \multicolumn{1}{c|}{\textbf{R}} & \multicolumn{1}{c|}{\textbf{M}} & \multicolumn{1}{l|}{\textbf{Explanation}}                                  \\ \hline
0                                   & 6                                       & 1                               & 0                               & 1                               & Not in the working set and not removed, go to the next page \\
1                                   & 9                                       & 1                               & \color{red}{0}                  & 0                               & Set R to $0$ and go to the next page                                       \\
2                                   & 9                                       & 1                               & \color{red}{0}                  & 1                               & Set R to $0$ and go to the next page                                       \\
  3                                   & 7                         & \color{red}{0}                               & 0                  & 0                               & Age > tau and clean --> replace it                                         \\
4                                   & 4                                       & 0                               & 0                               & 0                               & Age > tau and dirty --> claim it                                          
\end{tabular}
\end{table}

\section*{Exercise 4}


\subsection*{a)}

Address to read : $0x\underbrace{C}_{4 \text{ bits}}\underbrace{0DED}_{16 \text{ bits}}\underbrace{DAB}_{12 \text{ bits}}$.

\begin{itemize}
\item Segment value : $0xC$
\item Page value : $0x0DED$
\item Offset value : $0xDAB$
\end{itemize}

\subsection*{b)}

\begin{enumerate}
\item The segment value is used to find the segment descriptor.
\item Lookup for the page table entry $0x0DED$ and get its page number $n$.
\item Add $0xDAB$ to $n$ shifted $12$ bits to the left.
\item Access the memory.
\end{enumerate}

\subsection*{c)}

\begin{enumerate}
\item The segment value is used to find the segment descriptor.
\item If the segment is not in the memory, a segment fault is created.
\item If the page is not in the memory, a page fault is created.
\item The page is loaded in the memory.
\item Set the PTE's present bit to $1$.
\end{enumerate}

\section*{Exercise 5}

\subsection*{a)}

\begin{itemize}
\item $m$-bit is set when writing into the frame $\rightarrow$ non-modified
  frames don't have to be written back.
\item $r$-bit is set when accessing the frame. The bit is periodicaly reset.
\end{itemize}

\subsection*{b)}

Each page belongs to a category. There are $4 = 2^2$ categories in total : $00$,
$01$, $10$ and $11$. The page with the lowest category is evicted at random
(over all the page in the same category).

\section*{Exercise 6}

The implementation of NRU, NFU and FIFO are available in appendices.

\end{document}
