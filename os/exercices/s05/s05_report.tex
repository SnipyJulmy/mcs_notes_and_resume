\documentclass[a4paper,11pt]{report}

\usepackage{fullpage}

\usepackage{amsmath}
\usepackage{bussproofs}
\usepackage{mathpartir}
\usepackage{prooftrees}
\usepackage{color}
\usepackage{hyperref}
\usepackage{placeins}


% for finite state automata
\usepackage{tikz}
\usetikzlibrary{automata,positioning}

%%%%%%%%%%%%%%%%%%%%%%%%%%%%%%%%%%%%%%%%%%%%%%%%%%%%%%%%%
% Minted
%%%%%%%%%%%%%%%%%%%%%%%%%%%%%%%%%%%%%%%%%%%%%%%%%%%%%%%%%

\usepackage[cache=false]{minted}

%%%%%%%%%% C
\newmintinline{c}{
  fontsize=\small,
  breaklines=true
}

\newminted{c}{
  frame=single,
  framesep=2mm,
  fontsize=\scriptsize,
  mathescape
}

\newminted[ccodeline]{c}{
  frame=single,
  framesep=2mm,
  fontsize=\scriptsize,
  mathescape,
  linenos
}


%%%%%%%%% CMAKE
\newminted{cmake}{
  frame=single,
  framesep=2mm,
  fontsize=\scriptsize,
  mathescape,
  linenos,
  breaklines=true
}

% End minted
%%%%%%%%%%%%%%%%%%%%%%%%%%%%%%%%%%%%%%%%%%%%%%%%%%%%%%%%%


\author{Sylvain Julmy}
\date{\today}

\setlength{\parindent}{0pt}

\begin{document}

\begin{center}
  \Large{
    Operating Systems\\
    Spring 2018
  }
  
  \noindent\makebox[\linewidth]{\rule{\linewidth}{0.4pt}}
  S05
  \noindent\makebox[\linewidth]{\rule{\linewidth}{0.4pt}}
    AssistantArgs* args = arguments;
  \begin{flushleft}
    Professor : Philippe Cudré-Mauroux

    Assistant : Ines Arous
  \end{flushleft}
  
  \noindent\makebox[\linewidth]{\rule{\linewidth}{0.4pt}}

  Submitted by Sylvain Julmy
  
  \noindent\makebox[\linewidth]{\rule{\textwidth}{1pt}}
\end{center}

\section*{Exercice 2}

\subsection*{a)}

The problem appears when we try to add multiple consummer. We have to add
multiple waiting bit in order to manage multiple processes.

We can also image a world where the wakeup bit is set to $1$ and the signal is
not lost. Then, when the process would have to go to sleep, it would not and
produce a buffer overflow when creating to much item.

\subsection*{b)}

It would be a mutual exclusion problem with the count variable. We admit that
the producer wakeup the consummer and then try to increment the variable $count$
by $1$. Normally, the result would be $1 + 1 = 2$ for the count variable.
Instead, the consummer process would compute the expression $1 - 1 = 0$ when
decrementing the $count$ variable and then would overwrite the result, so the
$count$ variable would have value $0$ instead of $1$ or $2$ if no race condition
appears.

\section*{Exercice 3}

The code for exercise 3 is available in appendix.

\section*{Exercice 4}

At first, it is not always easy to express synchronization in term of predicate
proposition. We rather want to express synchronization in term of index in the
program (the position before executing the next statement). For example, we
would have to express position in the program using boolean variable that are
assigned to $true$ when a certain position is reach.

Additionally, the implementation of \verb+WAITUNTIL+ itself require to check
many time the condition. So there is a loop on the condition like the following
:
\begin{ccode}
while(cond);
\end{ccode}

\section*{Exercice 5}

\subsection*{a)}
The mutex $m_2$ is playing the role of signal between function
\cinline+sem_down+ and \cinline+sem_up+. When we lock $m_2$, the thread goes to
sleep and unlocking $m_2$ wakeup the thread.

The mutex $m_3$ garanteed that the value of $counter$ can't be modified by
another thread. When the program is going inside \cinline+sem_down+, the value
of $counter$ has to be $-1$ because it is the condition for the program to go
inside the $if$ of \cinline+sem_up+.

If we remove $m_3$, we could have an increment of $counter$ between the unlock
of $m_1$ and the lock of $m_2$, then $counter = 0$ and $m_2$ will stay blocked
because we can't go inside the if of \cinline+sem_up+.

\subsection*{b)}

We can remove the mutexes $m_2$ and $m_3$ and the controls structures inside
\cinline+sem_up+ and \cinline+sem_down+.

\end{document}
