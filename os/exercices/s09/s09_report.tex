\documentclass[a4paper,11pt]{report}

\usepackage{fullpage}

\usepackage{amsmath}
\usepackage{bussproofs}
\usepackage{mathpartir}
\usepackage{prooftrees}
\usepackage{color}
\usepackage{hyperref}
\usepackage{placeins}
\usepackage{tabularx}
\usepackage[normalem]{ulem}
\useunder{\uline}{\ul}{}


% for finite state automata
\usepackage{tikz}
\usetikzlibrary{automata,positioning}

%%%%%%%%%%%%%%%%%%%%%%%%%%%%%%%%%%%%%%%%%%%%%%%%%%%%%%%%%
% Minted
%%%%%%%%%%%%%%%%%%%%%%%%%%%%%%%%%%%%%%%%%%%%%%%%%%%%%%%%%

\usepackage[cache=false]{minted}

%%%%%%%%%% C
\newmintinline{c}{
  fontsize=\small,
  breaklines=true
}

\newminted{c}{
  frame=single,
  framesep=2mm,
  fontsize=\scriptsize,
  mathescape
}

\newminted[ccodeline]{c}{
  frame=single,
  framesep=2mm,
  fontsize=\scriptsize,
  mathescape,
  linenos
}

%%%%%%%%% CMAKE
\newminted{cmake}{
  frame=single,
  framesep=2mm,
  fontsize=\scriptsize,
  mathescape,
  linenos,
  breaklines=true
}

% End minted
%%%%%%%%%%%%%%%%%%%%%%%%%%%%%%%%%%%%%%%%%%%%%%%%%%%%%%%%%


\author{Sylvain Julmy}
\date{\today}

\setlength{\parindent}{0pt}

\begin{document}

\begin{center}
  \Large{
    Operating Systems\\
    Spring 2018
  }
  
  \noindent\makebox[\linewidth]{\rule{\linewidth}{0.4pt}}
  S09
  \noindent\makebox[\linewidth]{\rule{\linewidth}{0.4pt}}
  \begin{flushleft}
    Professor : Philippe Cudré-Mauroux

    Assistant : Ines Arous
  \end{flushleft}
  
  \noindent\makebox[\linewidth]{\rule{\linewidth}{0.4pt}}

  Submitted by Sylvain Julmy
  
  \noindent\makebox[\linewidth]{\rule{\textwidth}{1pt}}
\end{center}

\section*{Exercise 2}
\begin{gather*}
  100 MHz = 100 \cdot 10^6 Hz \rightarrow \text{$10^8$ times per second, $32$
    bits goes through the bus} \\
  10^8 \cdot 4 = 4\cdot10^8 \text{ bytes/s} = 400 \text{ Mbytes/s} \\
  \text{DMA transfer rate} = 40 \text{ Mbytes/s} \\
\end{gather*}

For every second, the DMA used $400 / 40 = 10$. The DMA reduced the transfer of
instructions by $10\%$.

\section*{Exercise 3}

% Please add the following required packages to your document preamble:
\begin{table}[h]
\centering
\caption{Advantages and disadvantages of placing functionality in device
  controller rather than directly in the kernel.}
\label{my-label}
\begin{tabularx}{\textwidth}{|X|c|X|}
\cline{1-1} \cline{3-3}
\textbf{Advantages}                                                                & \textbf{} & \textbf{Disadvantages}                                            \\ \cline{1-1} \cline{3-3} 
Reduce the system's workload                                                       &           & Can slow down the system if this one is idle                      \\ \cline{1-1} \cline{3-3} 
Compilation target is known so we can optimize for the hardware                    &           & Firmware has to be updated by hand (in most case)                 \\ \cline{1-1} \cline{3-3} 
Kernel is smaller and could contains less bugs                                     &           & Can't easely modify the firmware                                  \\ \cline{1-1} \cline{3-3} 
Device is separated from the kernel, so error won't cause the whole system to fail &           & Depending on the device we use, the implementation could be awful \\ \cline{1-1} \cline{3-3} 
\end{tabularx}
\end{table}

\section*{Exercise 4}

A character device driver is one that transfers data directly to and from a user
process. A block device is accessed by block of data, provide buffered access to
hardware devices, and provide some abstraction from their specifics.

Network device are none of them, such a device have some specific interface to
the kernel w.r.t. packet transmission. A network device does not implements the
classis $read$ and $write$ operations.

Yes, a file systems is mount logically and then multiple block devices could be
used to map the data.

\section*{Exercise 5}

Assuming the $acquire$ function is blocking until the mutex is acquired, the
following implementation prevent deadlocks.

\begin{ccode}
void transaction(Account from, Account to, double amount)
{
    mutex lock1, lock2;
    lock1 = get_lock(from);
    lock2 = get_lock(to);

    acquire(lock1);
    withdraw(from, amount);
    release(lock1);

    acquire(lock2);
    deposit(to, amount);
    release(lock2);
}
\end{ccode}

\end{document}
