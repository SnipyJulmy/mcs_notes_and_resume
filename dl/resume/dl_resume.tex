\documentclass[a4paper,11pt]{report}

\usepackage[T1]{fontenc}
\usepackage[english]{babel}
\usepackage{etoolbox}
\usepackage{sourcecodepro}
\usepackage{amsmath}
\usepackage{amsthm}
\usepackage{fullpage}

\newcommand*{\alc}{$\mathcal{ALC}$ }
\newcommand*{\sso}{\sqsubseteq}
\newcommand*{\tb}{\mathcal{T}}
\newcommand*{\ab}{\mathcal{A}}

\title{Resume : Description Logic}

\begin{document}

\maketitle

$C$ is the set of concept names and $R$ the set of role names. Every concept names is a concept description (CD).

Formal notation :
\begin{align*}
& C \sqcap D \rightarrow \text{conjonction} \\
& C \sqcup D \rightarrow \text{disjonction} \\
& \neg C \rightarrow \text{negation} \\
& \exists r.C \rightarrow \text{existential restriction} \\
& \forall r.C \rightarrow \text{value restriction}
\end{align*}

An interpretation $I = (\Delta,\cdot )$ such that $\Delta \neq \emptyset$, ($\Delta$ is called the domain of the interpretation) and with the following :
\begin{align*}
& A \in C \rightarrow A^I \subseteq \Delta \\
& r \in R \rightarrow r^I \subseteq \Delta \times \Delta \\
& \top^I = \Delta \\
& \bot^I = \Delta \\
& (C \sqcap D)^I = C^I \cap D^I \\
& (C \sqcup D)^I = C^I \cup D^I  \\
& (\neg C)^I = \Delta \setminus C^I \\
& (\exists r.C)^I = \{d \in \Delta \mid \exists e \in \Delta \text{ with } (d,e) \in r^I \text{ and } e \in C^I \}\\
& (\forall r.C)^I = \{ d \in \Delta \mid \forall e \in \Delta, \text{ if } (d,e) \in r^I, \text{ then }e \in C^I \}
\end{align*}

We call $C^I$ the extension of $C$ in $I$ and $b \in \Delta^I$ an $r-filler$ of $a$ in $I$ if $(a,b) \in r^I$.

\newtheorem{Lemma}{Lemma}
\begin{Lemma}
Let $I$ be an interpretation, $C$, $D$ concepts and $r$ a role. Then
\begin{align*}
& \top^I = (C \sqcup \neg C)^I \\
& \bot^I = (C \sqcap \neg C)^I \\
& (\neg \neg C)^I = C^I \\
& (\neg(C \sqcap D))^I = (\neg C \sqcup \neg D)^I \\
& (\neg(C \sqcup D))^I = (\neg C \sqcap \neg D)^I \\
& (\neg(\exists r.C))^I = (\forall r.\neg C)^I \\
& (\neg(\forall r.C))^I = (\exists r.\neg C)^I
\end{align*}
\end{Lemma}

Because the $\sqcup$ operator can be tricky sometimes, we use the following relation to replace the $\sqcup$ by $\sqcap$ : $C \sqcup D \rightarrow \neg(\neg C \sqcap \neg D)$.

\section*{$ALC$ $TBoxes$}

For $C$ and $D$ possibly compound $ALC$ concepts, an expression of form $C \sqsubseteq D$ is called an $ALC$ general concept inclusion (GCI). We use $C \equiv D$ has an abbreviation for $ C\tau \sqsubseteq D, D \sqsubseteq C$. A finite set of GCI is called an $ALC$ $TBox$ and noted $\tau$.

\begin{Lemma}
If $\tau \in \tau'$ for two $TBox$ $\tau$ and $\tau'$, then each model of $\tau'$ is also a model of $\tau$. 
\end{Lemma}

Example of a $TBox$ $T_{ex}$ :
\begin{align*}
T_{ex} = \{ Course &\sqsubseteq \neg Person,\\
UGC &\sqsubseteq Course,\\
PGC &\sqsubseteq Course,\\
Teacher &\equiv Person \sqcap \exists teaches.Course,\\
\exists teaches.T &\sqsubseteq Person,\\
Student &\equiv Person \sqcap \exists attends.Course ,\\
\exists attends.T &\sqsubseteq Person \}
\end{align*}

\section*{$ALC$ $ABoxes$}

Let $I$ be a set of individual names disjoint from $R$ and $C$. For $a,b \in I$ individual names, $C$ a possibly compound $ALC$ concept, and $r \in R$ a role name, an expression of the form $a:C$ is called an $ALC$ concept assertion and $(a,b) : r$ is called and $ALC$ role assertion.

A finite set of concept and role assertion is called an $ALC$ $ABox$. An interpretation function $\cdot^I$ is additionally required to map every individual name $a\in I$ to an element $a^I \ in \Delta^I$. An interpretation $I$ satisfies a concept assertion $a:C$ if $a^I \in C^I$ and a role assertion $(a,b):r$ if $(a^I,b^I) \in r^I$.

An interpretation that satisfies each concept assertion and each role assertion in a $ABox$ $A$ is called a model of $A$.

Example of a $ABox$ $A_{ex}$ :
\begin{align*}
A_{ex} = \{ Mary &: Person, \\
CS600 &: Course,\\
Ph456 &: Course \sqcap PGC,\\
Hugo &: Person,\\
Betty &: Person \sqcap Teacher,\\
(Mary,CS600) &: teaches,\\
(Hugo,Ph456) &: teaches,\\
(Betty,Ph456) &: attends,\\
(Mary,Ph456) &: attends \}
\end{align*}

Then we can create an interpretation $I$ of this $ABox$ which is a model:
\begin{align*}
\Delta^I &= \{h,m,c6,p4\}, \\
Mary^I &= m \\
Betty^I &= Hugo^I = h, \\
CS600^I &=  c6,\\
Ph456^I &=  p4,\\
Person^I &= \{h,m,c6,p4\},\\
Teacher^I &= \{h,m\},\\
Course^I &= \{c6,p4\},\\
PGC^I &= \{p4\},\\
UGC^I &= \{cs\},\\
Student^I &= \emptyset,\\
teaches^I &= \{(m,c6),(h,p4)\},\\
attends^I &= \{(h,p4),(m,p4)\}
\end{align*}

Note that isn't a model of the previous example of a $TBox$.

\chapter{A Basic Description Logic}

\section{TBox}

\newtheorem{GCI}{Definition}[section]
\begin{GCI}
  For $C$ and $D$ possibly compound \alc concepts, an expression of the form $C
  \sso D$ is called an \alc \textit{general concept inclusion} and abbreviated
  GCI. We use $C \equiv D$ for $C \sso D, D \sso C$.
  \begin{itemize}
  \item A finite set of GCI is called an \alc TBox.
  \item An interpretation $I$ satisfies a GCI $C \sso D$ if $C^I \subseteq D^I$.
  \item An interpretation that satisfies each GCI in a TBox $\mathcal{T}$ is called a model of $\mathcal{T}$.
  \end{itemize}
\end{GCI}

\begin{Lemma}
  If $\tb \subseteq \tb'$ for two TBoxes $\tb$ and $\tb'$, then each model of
  $\tb'$ is a model of $\tb$.
\end{Lemma}

\section{ABox}

\newtheorem{ABOX}{Definition}[section]
\begin{ABOX}
  Let $\mathbf{I}$ be a set of \textit{individual names} disjoint from
  $\mathbf{R}$ and $\mathbf{C}$. For $a,b \in \mathbf{I}$ individual names, $C$
  a possibly compound \alc concept, and $r \in \mathbf{R}$ a role name, an
  expression of the form :
  \begin{itemize}
  \item $a : C$ is called an \alc concept assertion
  \item $(a,b) :r$ is called an \alc role assertion
  \end{itemize}
\end{ABOX}

A finite set if \alc concept and role assertion is called an \alc ABox.

An interpretation function $\cdot^I$ is additionally required to map every
individual name $a\in \mathbf{I}$ ti and element $a^I \in \Delta^I$. An
interpretation $I$ satisfies
\begin{itemize}
\item a concept assertion $a:C$ if $a^I \in C^I$
\item a role assertion $(a,b) : r$ if $(a^I,b^I) \in r^I$
\end{itemize}

\newtheorem{KB}{Definition}[section]
\begin{KB}
  An \alc \textit{knowledge base} $\mathcal{K} = (\mathcal{T},\mathcal{A})$
  consists of an \alc TBox $\tb$ and an \alc ABox $\ab$. An interpretation that
  is both a model of $\tb$ and $\ab$ is called a model of $\mathcal{K}$.
\end{KB}

\end{document}
