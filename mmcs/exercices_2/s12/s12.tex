\documentclass[a4paper,11pt]{report}

\usepackage{amsmath}
\usepackage{amsthm}
\usepackage{fullpage}
\usepackage{tikz}

\usetikzlibrary{graphs,graphs.standard}

\makeatletter
\pgfmathdeclarefunction{alpha}{1}{%
  \pgfmathint@{#1}%
  \edef\pgfmathresult{\pgffor@alpha{\pgfmathresult}}%
}

\usepackage{bussproofs}
\usepackage{mathpartir}
\usepackage{prooftrees}
\usepackage{color}

\newcommand*{\contract}[2]{contraction of $#1$ with $#2$}

\newcommand*{\MP}{Modus Ponens }
\newcommand*{\MPr}[2]{\MP of \ref{#1} and \ref{#2}}

\author{Sylvain Julmy}
\date{\today}

\setlength{\parindent}{0pt}

\begin{document}

\begin{center}
  \Large{
    Mathematical Methods for Computer Science 1\
    Fall 2017
  }
  \noindent\makebox[\linewidth]{\rule{\linewidth}{0.4pt}}

  Series 12
  \vspace*{1.4cm}

  Sylvain Julmy
  
  \noindent\makebox[\linewidth]{\rule{\linewidth}{0.4pt}}
\end{center}

\section*{\texttt{1}}

The grammar $G = (V,T,P,S)$ where

\begin{align*}
  V &= \{P,A,B\}\\
  T &= \{0,1\} \\
  P &= \{ \\
    & P \to AB \\
    & A \to AA\ |\ 0\ |\ 10 \\
    & B \to 1\ |\ \epsilon \\
    &\}\\
  S &= P \\
\end{align*}

generates the language $(0 + 10)^*(1 + \epsilon)$

\section*{\texttt{2}}

\subsection*{\texttt{a)}}

We use the following PDA in order to recognize $L$ :

\begin{align*}
  P &= (\{q_0\},\{0,1\},\{Z_0,0,1\},\delta,q_0,Z_0,\emptyset) \\
\end{align*}

where $\delta$ is defined as the following :

\begin{align*}
  \delta(q_0,0,Z_0) &= \{(q_0,0)\} \\
  \delta(q_0,1,Z_0) &= \{(q_0,1)\} \\
  \delta(q_0,0, 0) &= \{(q_0,00)\} \\
  \delta(q_0,0, 1) &= \{(q_0,\epsilon)\} \\
  \delta(q_0,1, 0) &= \{(q_0,\epsilon)\} \\
  \delta(q_0,1, 1) &= \{(q_0,11)\} \\
\end{align*}

\subsection*{\texttt{b)}}

We use the following PDA in order to recognize $L$ :

\begin{align*}
  P &= (\{q_0,q_1,q_2\},\{0,1\},\{Z_0,0,1\},\delta,q_0,Z_0,\{q_2\}) \\
\end{align*}

where $\delta$ is defined as the following :

\begin{align*}
  \delta(q_0,0,Z_0) &= \{(q_1,Z_0 0)\} \\
  \delta(q_0,1,Z_0) &= \{(q_1,Z_0 1)\} \\
  \delta(q_1,0, 0) &= \{(q_1,00)\} \\
  \delta(q_1,0, 1) &= \{(q_1,\epsilon)\} \\
  \delta(q_1,1, 0) &= \{(q_1,\epsilon)\} \\
  \delta(q_1,1, 1) &= \{(q_1,11)\} \\
  \delta(q_1,\epsilon,Z_0) &= \{q_2,Z_0\} \\
\end{align*}

\section*{\texttt{3}}

\subsection*{\texttt{a)}}

Using the algorithm described in lecture, we convert $S \to 0S0\ |\ 1S1\ |\
\epsilon$ to the following PDA :

\begin{align*}
  P &= (\{q\},\{0,1,\epsilon\},\{0,1,\epsilon,S\},\delta,q,S,\emptyset) \\
\end{align*}

where $\delta$ is defined as the following :

\begin{align*}
  \delta(q,\epsilon,S) &= \{(q,0S0),(q,1S1),(q,\epsilon)\} \\
  \delta(q,0,0) &= \{(q,\epsilon)\} \\
  \delta(q,1,1) &= \{(q,\epsilon)\} \\
\end{align*}

\subsection*{\texttt{b)}}

The word is accepted :
\begin{enumerate}
\item Top of the stack is $S$, consume $\epsilon$ from the string, $S$ from the stack and push $0S0$.
\item Top of the stack is $0S0$, consume $0$ from the string, $0$ from the stack and push $\epsilon$.
\item Top of the stack is $S0$, consume $\epsilon$ from the string, $S$ from the stack and push $1S1$.
\item Top of the stack is $1S10$, consume $1$ from the string, $1$ from the stack and push $\epsilon$.
\item Top of the stack is $S10$, consume $\epsilon$ from the string, $S$ from the stack and push $1S1$.
\item Top of the stack is $1S110$, consume $1$ from the string, $1$ from the stack and push $\epsilon$.
\item Top of the stack is $S110$, consume $\epsilon$ from the string, $S$ from the stack and push $0S0$.
\item Top of the stack is $0S0110$, consume $0$ from the string, $0$ from the stack and push $\epsilon$.
\item Top of the stack is $S0110$, consume $S$ from the string, $S$ from the stack and push $\epsilon$.
\item Top of the stack is $0110$, consume $0$ from the string, $0$ from the stack and push $\epsilon$.
\item Top of the stack is $110$, consume $1$ from the string, $1$ from the stack and push $\epsilon$.
\item Top of the stack is $10$, consume $1$ from the string, $1$ from the stack and push $\epsilon$.
\item Top of the stack is $0$, consume $0$ from the string, $0$ from the stack and push $\epsilon$.
\item The stack is empty, the word is accepted.
\end{enumerate}

Stops before the word is read completely :

\begin{enumerate}
\item Top of the stack is $S$, consume $\epsilon$ from the string, $S$ from the stack and push $0S0$.
\item Top of the stack is $0S0$, consume $0$ from the string, $0$ from the stack and push $\epsilon$.
\item Top of the stack is $S0$, consume $\epsilon$ from the string, $S$ from the stack and push $1S1$.
\item Top of the stack is $1S10$, consume $1$ from the string, $1$ from the stack and push $\epsilon$.
\item Top of the stack is $S10$, consume $\epsilon$ from the string, $S$ from the stack and push $\epsilon$.
\item Top of the stack is $10$, consume $1$ from the string, $1$ from the stack and push $\epsilon$.
\item Top of the stack is $0$, consume $0$ from the string, $0$ from the stack and push $\epsilon$.
\item The stack is empty $\to$ end of the processing without reading the whole word.
\end{enumerate}

\section*{\texttt{4}}

Let $M$ be the PDA that accept the language $L$, then we have the transition
function $\delta$ of $M$ is a map $Q \times (\Sigma \cup \epsilon ) \times \Gamma
\mapsto Q \times \Gamma^*$.

The idea is to create additional state when a single transition is pushing
more than $1$ symbol on the stack. Each of those additional transition would
push a symbol on the stack.

For each transition $\delta(q_i,a,A_i) \to (q_j,A_iA_{i+1}\dots A_{i+n})$ where
$n > 1$, $a \in \Sigma \cup \epsilon$, $A_i \in S$ we create additional state with
corresponding transition function.


\[
  \delta(q_i,a,A_i) \to (q_j,A_jA_{j+1}\dots A_{j+n})
\]

is transformed into

\begin{align*}
  \delta(q_i,a,A_i) &\to (q_j^{(1)},A_jA_{j+1}) \\
  \delta(q_j^{(1)},\epsilon,A_{j+1}) &\to (q_j^{(2)},A_{j+1}A_{j+2}) \\
  \dots \\
  \delta(q_j^{(n-1)}, \epsilon,A_{j+n-1}) &\to (q_j,A_{j+n-1}A_{j+n}) \\
\end{align*}

The language accepted by such a restricted PDA is the same as the original PDA
because we add state that did not exist and only transition between those state
in a very specific way. Clearly, state from the original PDA can't reach the new
state $q_j^{(k)}$ because they simply don't exist in the original PDA. The
created state can't reach the states from the original PDA because we only
establish transition with the $q_i$ and $q_j$ state with those.

We just simulate a push of multiple symbol by multiple push of one symbol.

\section*{\texttt{5}}

A PDA with a bounded stack height has a finite number of possibility of storage.
Therefore, we can construct a finite state automata from such a PDA. A PDA is an
$\epsilon$-automata which has access to a stack in order to store information.
The idea is to construct an $\epsilon$-automata from a PDA with a bounded stack
height.

The idea is to create state that represents the content of the stack.

I have no time left to demonstrate the idea...

\end{document}
