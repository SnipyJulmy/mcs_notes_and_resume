\documentclass[a4paper,11pt]{report}

\usepackage{amsmath}
\usepackage{fullpage}
\usepackage{mathpartir}

\author{Sylvain Julmy}
\date{\today}

\setlength{\parindent}{0pt}

\newcommand*{\bin}[2]{\begin{pmatrix}#1 \\ #2\end{pmatrix}}

\begin{document}

\begin{center}
  \Large{
    Mathematical Methods for Computer Science 2\\
    Spring 2018
  }
  \noindent\makebox[\linewidth]{\rule{\linewidth}{0.4pt}}

  Series 1
  \vspace*{1.4cm}

  Sylvain Julmy
  
  \noindent\makebox[\linewidth]{\rule{\linewidth}{0.4pt}}
\end{center}

\section*{\texttt{1}}

\subsection*{2.1)}

\begin{align*}
  (1 + \sqrt{5})^n &= \bin{n}{0} 1^{n} \sqrt{5}^{0} + \bin{n}{1} 1^{n-1} \sqrt{5}^{1} + \dots + \bin{n}{n-1} 1^{1} \sqrt{5}^{n-1} + \bin{n}{n} 1^{0} \sqrt{5}^{n} \\
                   &= 1 + \bin{n}{1} \sqrt{5} + \bin{n}{2} \sqrt{5}^{2} + \dots + \bin{n}{n-1} \sqrt{5}^{n-1} + \bin{n}{n} \sqrt{5}^{n} \\
                   &= \sum_{i=0}^{n} \bin{n}{i} \sqrt{5}^{i}
\end{align*}

\subsection*{a.2)}

\begin{align*}
  (1 - \sqrt{5})^n &= (1 + (-\sqrt{5}))^n \\
                   &= \bin{n}{0} 1^{n} (-\sqrt{5})^{0} + \bin{n}{1} 1^{n-1} (-\sqrt{5})^{1} + \dots + \bin{n}{n-1} 1^{1} (-\sqrt{5})^{n-1} + \bin{n}{n} 1^{0} (-\sqrt{5})^{n} \\
                   &= 1 - \bin{n}{1} \sqrt{5} + \bin{n}{2} \sqrt{5}^{2} - \dots + \bin{n}{n-1} (-\sqrt{5})^{n-1} + \bin{n}{n} (-\sqrt{5})^{n} \\
                   &= \sum_{i=0}^{n} (-1)^i \bin{n}{i} \sqrt{5}^{i}
\end{align*}

\subsection*{b)}

\begin{align*}
  a_n &= \frac{\bin{n}{1} + 5 \bin{n}{3} + 5^2 \bin{n}{5} + \dots}{2^{n-1}} \\
\end{align*}

The proof is by expanding the Binet's formula using the binomial coefficient. So
we have

\begin{align*}
  a_n &= \frac{1}{\sqrt{5}} ((\frac{1 + \sqrt{5}}{2})^n - (\frac{1 - \sqrt{5}}{2})^n) \\
\end{align*}

and after expansion we obtain

\begin{align*}
  a_n &=  \frac{1}{2^n \sqrt{5}} \left(\sum_{i=0}^{n}\left(\bin{n}{k}(\sqrt{5}^k)\right) - \sum_{i=0}^{n}\left(\bin{n}{k}(\sqrt{5}^k)(-1)^k\right)\right) \\
      &= \frac{1}{2^n \sqrt{5}} \left(\sum_{i=0}^{n}\left(\bin{n}{k} (\sqrt{5})^k (1-(-1)^k))\right) \right) \\
\end{align*}

When $k$ is even, $1-(-1)^k = 0$ and $1-(-1)^k = 2$ when $k$ is odd. So we
obtain

\begin{align*}
  a_n &= \frac{1}{2^n \sqrt{5}} \left(\sum_{2j+1=1}^{2j+1=n}\left(\bin{n}{2j+1} (\sqrt{5})^{2j+1} \cdot 2 \right) \right) \\
      &= \frac{1}{2^{n-1} \sqrt{5}} \sum_{j=0} \left( \bin{n}{2j+1}(5^j \sqrt{5}) \right) \\
      &= \frac{1}{2^{n-1}} \sum_{j=0} \left( 5^j \bin{n}{2j+1} \right) \\
      &= \frac{5^0 \bin{n}{1} + 5^1 \bin{n}{3} + 5^2 \bin{n}{5} \dots}{2^{n-1}}
\end{align*}


\section*{\texttt{2}}

\begin{align*}
  a_1 &= 1 \\
  a_2 &= 5 \\
  a_n &= 5a_{n-1} - 6a_{n-2}
\end{align*}

We find that
\begin{gather*}
  a_2 = 5a_1 - 6a_0 = 5 \\
  5 = 5 * 1 - 6 * a_0 = 5 - 6 * a_0 \\
  6 * a_0 = 0 \\
  a_0 = 0
\end{gather*}

\begin{mathpar}
  X = (1, \lambda, \lambda^2) \\
  \lambda^n = 5 \lambda^{n-1} - 6 \lambda^{n-2} \\
  \lambda^2 = 5 \lambda - 6 \\
  -\lambda^2 + 5 \lambda - 6 = 0
\end{mathpar}

We compute the discriminant :

\begin{mathpar}
  \Delta = 25 - (4 * -1 * -6) = 25 - 24 = 1 \\
  \lambda_1 = \frac{-5 + \sqrt{1}}{2 * (-1)} = \frac{-4}{-2} = 2 \\
  \lambda_2 = \frac{-5 - \sqrt{1}}{2 * (-1)} = \frac{-6}{-2} = 3 \\
\end{mathpar}

And we obtain

\begin{mathpar}
  A = C_1 \lambda_1 + C_2 \lambda_2 \\
  a_n = C_1 \lambda_1^n + C_2 \lambda_2^n \\
\end{mathpar}

For $n=0$
\begin{mathpar}
  a_0 = C_1 + C_2
\end{mathpar}

For $n=1$
\begin{mathpar}
  a_1 = C_1 \lambda_1 + C_2 \lambda_2
\end{mathpar}

and we obtain the following system :
\[
  \begin{cases}
    C_1 + C_2 = 0\\
    C_1 \lambda_1 + C_2 \lambda_2 = 1
  \end{cases}
  =
  \begin{cases}
    C_1 + C_2 = 0\\
    2 C_1 + 3 C_2 = 1 
  \end{cases}
\]

Resolution :

\begin{mathpar}
  C_1 = - C_2 \\
  2 (- C_2) + 3 C_2 = 1 \\
  C_2 = 1 \\
  C_1 = -1
\end{mathpar}

Finally we have

\begin{align*}
  a_n &= C_1 \lambda_1^n + C_2 \lambda_2^n \\
      &= (-1) * 2^n + 1 * 3^n \\
      &= 3^n - 2^n
\end{align*}

\section*{\texttt{3}}

\begin{align*}
  a_1 &= 1 \\
  a_2 &= 2 \\
  a_3 &= 5 \\
  a_n &= 3 a_{n-2} - 2 a_{n-3}
\end{align*}

We find that
\begin{mathpar}
  a_3 = 3 a_1  - 2 a_0 \\
  5 = 3 * 1 - 2 * a_0 \\
  2 = - 2 * a_0 \\
  a_0 = -1
\end{mathpar}

\begin{mathpar}
  \lambda^n = 3 \lambda^{n-2} - 2 \lambda^{n-3}
\end{mathpar}

We pick $n=3$

\begin{gather*}
  \lambda^3 = 3\lambda - 2 \\
  -\lambda^3 + 3\lambda - 2 = 0
\end{gather*}

We find that $1$ is a root of the equation and we obtain

\begin{gather*}
  -\lambda^3 + 3\lambda - 2 = (\lambda - 1)(-\lambda^2 - \lambda + 2)
  = -(\lambda - 1)^2(\lambda + 2) \\
  \lambda_1 = 1 \\
  \lambda_2 = -2 \\
\end{gather*}

\begin{align*}
  P(\lambda) &= (\lambda - \lambda_1)^2 (\lambda - \lambda_2) \\
             &= (\lambda - 1)^2 (\lambda - (-2)) \\
\end{align*}

\begin{align*}
  a_n &= C_1 \lambda_1^n + C_2 n \lambda_1^n + C_3 \lambda_2^n \\
      &= C_1 * 1^n + C_2 * n * 1^n + C_3 * (-2)^n \\
      &= C_1 + n * C_2 + (-2)^n * C_3
\end{align*}

For $n=0$

\[
  a_0 = C_1 + C_3 = -1
\]

For $n=1$
\[
  a_1 = C_1 + C_2 - 2 C_+ 1
\]

For $n=2$
\[
  a_2 = C_1 + 2 C_2 + 4 C_3 = 2
\]

And we obtain the following system :

\[
  \begin{cases}
    C_1 + C_3 = -1 \\
    C_1 + C_2 - 2 C_3 = 1 \\
    C_1 + 2 C_2 + 4 C_3 = 2
  \end{cases}
\]

\begin{mathpar}
  C_1 = - 1 - C_3
\end{mathpar}

\[
  \begin{cases}
    - C_3 - 1 + C_2 - 2 C_3 = 1 \\
    - C_3 - 1 + 2 C_2 + 4 C_3 = 2
  \end{cases}
  =
  \begin{cases}
    C_2 - 3 C_3 - 1 = 1 \\
    3 C_3 - 1 + 2 C_2 = 2
  \end{cases}
\]

\begin{mathpar}
  C_2 = 3 C_3 + 2
\end{mathpar}

\begin{mathpar}
  3 C_3 - 1 + 2 (3 C_3 + 2) = 2 \\
  3 C_3 - 1 + 4 + 6 C_3 = 2 \\
  9 C_3 + 3 = 2 \\
  9 C_3 = -1 \\
  C_3 = -\frac{1}{9} \\
  C_2 = 3 * -\frac{1}{9} + 2 = \frac{5}{3} \\
  C_1 = -1 - \frac{1}{9} = -\frac{8}{9}\\
\end{mathpar}

Finally we have
\begin{align*}
  a_n &= C_1 \lambda_1^n = C_2 n \lambda_1^n + C_3 \lambda_2^n \\
      &= -\frac{4}{3} * 1^n + 2 * n * 1^n + (-\frac{1}{6}) * (-2)^n \\
      &= -\frac{8}{9} + n \frac{5}{3} + (-2)^n \cdot -\frac{1}{9}
\end{align*}


\section*{\texttt{4}}

\subsection*{a)}
It's clear that for $n = 1$, there is only one possible way of placing the
domino. So $b_1 = 1 = a_2$ which is correct. The same for $b_2$, we could either
place both two dominos horizontally or vertically, then we obtain $b_2 = 2 =
a_3$ which is also correct.

Then we use a recurence relation, we got a rectangle of $2 \times n$ denote by
$R_n$. At the first step, we could place a domine either horizontally or
vertically, which let us have two cases :
\begin{itemize}
\item Horizontally : if the domino is placed horizontally, then we obtain a
  rectangle $R_{n-1}$ to fill, which we know how to do it.
\item Vertically : if the domino is placed vertically, then we are forced to
  place the second one next to him (bottom or top) in order to fill the hole.
  Then we obtain a rectangle $R_{n-2}$  to fill, which we know how to do it.
\end{itemize}

Now to count the number of ways to arrange an $R_n$ rectangle with dominos, we
count the number of ways to arrange $R_{n-1}$ and $R_{n-2}$ and sum them. Which
is exactly the recurence relation of the Fibonacci's sequence.

\subsection*{b)}

It is the same problem as before. Any edge that is in the matching represent a
domino tilled on a rectangle $2 \times n$. So the number of perfect matching in
$P_{2,n}$ is given by $a_{n+1}$.

\section*{\texttt{5}}

\subsection*{a)}
The number of different compositions of a positive integer $n$ with all summands
equal to $1$ or $2$ is given by $a_{n+1}$. For $n=0$, there is $1$ solution
(empty compositions of $1$ and $2$). For $n = 1$, there is $1$ solution.
For $n=2$, there are $2$ solutions, which is the fibonacci sequence.

Then, for any $n$, we use the construction of $n$ from $1$ and $2$ as a
recursive one. We could either pick $1$ or $2$ at the beginning :
\begin{itemize}
\item If we pick $1$, we count the number of different compositions with $1$ and
  $2$ of $n-1$.
\item If we pick $2$, we count the number of different compositions with $1$ and
  $2$ of $n-2$.
\end{itemize}

Then we add the number of different compositions of $n-1$ and $n-2$, and sum
them. Which is exactly the fibonacci recursive relation.

\subsection*{b)}

\[
  a_{n+1} = \bin{n}{0} + \bin{n-1}{1} + \bin{n-2}{2} + \dots + \bin{n-k}{k} =
  \sum_{k \geq 0} \bin{n-k}{k}
\]

The proof is by induction on $n$ :

\paragraph{Base cases : }
\begin{itemize}
\item If $n = 0$, we have $1$, which is $a_1$.
\item If $n = 1$, we have $1$, which is $a_2$.
\item If $n = 2$, we have $2$, which is $a_3$.
\end{itemize}

\paragraph{Inductive step : }

We use the following equation for the proof :

\[
  \bin{n}{k} + \bin{n}{k+1} = \bin{n+1}{k+1}
\]

We suppose $n > 1$, then we have for $n+1$

\begin{align*}
  \sum_{k \geq 0} \bin{n + 1 - k}{k}
  &= \sum_{k \geq 0} \bin{n-k}{k} + \sum_{k \geq 0} \bin{n-k}{k-1} \\
  &= \sum_{k \geq 0} \bin{n-k}{k} + \sum_{k \geq 0} \bin{n - 1 - (k - 1)}{k-1} \\
  &= \sum_{k \geq 0} \bin{n-k}{k} + \sum_{j \geq 0} \bin{n - 1 - j}{j} \\
  &= a_{n+1} + a_{n} = a_{n + 2}
\end{align*}

\end{document}


