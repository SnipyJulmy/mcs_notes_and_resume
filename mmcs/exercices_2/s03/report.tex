\documentclass[a4paper,11pt]{report}

\usepackage{amsmath}
\usepackage{fullpage}

\author{Sylvain Julmy}
\date{\today}

\setlength{\parindent}{0pt}

\begin{document}

\begin{center}
  \Large{
    Mathematical Methods for Computer Science 2\\
    Spring 2018
  }
  \noindent\makebox[\linewidth]{\rule{\linewidth}{0.4pt}}

  Series 3
  \vspace*{1.4cm}

  Sylvain Julmy
  
  \noindent\makebox[\linewidth]{\rule{\linewidth}{0.4pt}}
\end{center}

\section*{\texttt{1}}

\subsection*{a)}

By extending the brackets of 

\[
  (1-x)(1-x^2)(1-x^4)(1-x^8)\cdot\dots
\]

We can see that any $x^n$ is obtained only one time. For example, in order to
obtain $x^7$, we have to multiply $x$, $x^2$, $x^4$ and pick the $1$ in all the
other factors. If we don't pick a $1$, we obtain a $n$ which is greater that
$7$. By the way, it is the encoding of the integer in the binary format.

\subsection*{b)}

The result obtained is the sum from $a)$ and the sign in front of $x^n$ is plus
only when the number we encode into the binary format is encoded with an even
number of bits.

If the number of bits is even, it means we have take an even number of $x^{2^k}$
from the product and then an even number of negative $x^{2^k}$ which leads to a
positive $x^n$.

\section*{\texttt{2}}

\subsection*{a)}

\begin{align*}
  \sum_{n=0}^\infty a_n x^n = (1 + x + x^2 + x^3 + \dots)^k
  &= (\frac{1}{(1-x)})^k \\
  &= \frac{1}{(1-x)^k} \\
  &= (1-x)^{-k} \\
  &= \sum_{n=0}^\infty \binom{n+k-1}{k-1} x^n
\end{align*}

\subsection*{b)}

\begin{align*}
  \frac{1}{1-x-x^2} &= \frac{1}{1-(x+x^2)} \\
                    &= 1 + (x+x^2)x + (x+x^2)x^2 + (x+x^2)x^3 + (x+x^2)x^4 + \dots \\
                    &= 1 + x + 2x^2 + 3x^3 + 5x^4 + 8x^5 + 13x^6 + \dots
\end{align*}

The coefficient in front of each $x^n$ follow the fibonnaci sequence. The
summands are represented by the factors $(x + x^2)$, which means the summands
$1$ and $2$. Then, by expanding we see that the number of compositions with
summands $1$ and $2$.


\section*{\texttt{3}}

\subsection*{a)}

\subsection*{b)}


\section*{\texttt{4}}

\subsection*{a)}

The generating function for $p_{n,k}$ is :

\begin{align*}
  \sum_{n=0}^\infty p_n x^n
  &= (1 + x + x^2 + x^3 + \dots)(1 + x^2 + x^4 + x ⁶ + \dots)(1 + x^3 + x^6 + x^9 + \dots)\dots \\
  &= a_0 + a_1x + a_2x^2 + \dots \\
  &= \prod_{k=1}^\infty (\frac{1}{1 - x^k}) \\
\end{align*}

The $x^n$ term in the series $a_0 + a_1x + a_2x^2 + \dots$ count the number of
ways to obtain $n$ where $n = a_1 + 2 a_2 + 3 a_3 + \dots$, which is the number
of partition of $n$.

\subsection*{b)}

If we need to have the number of partitions of $n$ into exactly $k$ parts, we
multiply the factor by $x^k$ :

\[
  \prod_{k=1}^\infty (\frac{x}{1 - x^k})
\]

\subsection*{c)}

\[
  p_{n-k,\geq k} = p_{n,k}
\]

\paragraph{Algebraic proof : }



\section*{\texttt{5}}

\subsection*{a)}

\[
  p_{n-k,\geq k} = p_{n,k}
\]

\paragraph{Proof by bijection : } we consider two sets $P_1$ and $P_2$ where
$P_1$ is the set of partitions of $n-k$ with no parts greater than $k$ and $P_2$
is the set of partitions of $n$ with parts into exactly $k$ parts.

Clearly, we have $P_1 \cap P_2 = \emptyset$

We can construct the set $P_1$ from the set $P_2$ by using the following
construction for each parts :

\begin{itemize}
\item create the Ferrers diagram of the part
\item separate $k$ into $K =  \underbrace{\{1,1,\dots,1\}}_{k}$
\item for each $k_i \in K$ add $1$ square to the first row of the diagram, then
  one to the second row, one to the third, hand so on.
\item We add one square to the row only if the row index is smaller or equals to
  $k$.
\item If we are at the last row, we get back to the first one and repeat the
  process until we have no $k_i$ left.
\end{itemize}

This way, each element of $P_1$ is associated with an unique element in $P_2$.

\subsection*{b)}

Let
\begin{align*}
  f(x) &= (1+x+x^2)(1+x^2+x^4)(1+x^3+x^6)\cdot\dots \\
  g(x) &= (1+x+x^2+\dots)(1+x^2+x^4+\dots)(1+x^4+x^8\dots)\cdot\dots
\end{align*}

Then we have

\begin{align*}
  g(x) &= (1+x+x^2+\dots)(1+x^2+x^4+\dots)(1+x^4+x^8\dots)\cdot\dots \\
       &= (\frac{1}{1-x})(\frac{1}{1-x²})(\frac{1}{1-x^4})\cdots\\
       &= \frac{(\frac{1}{1-x^3})}{(\frac{1}{1-x^3})(\frac{1}{1-x})(\frac{1}{1-x^2})(\frac{1}{1-x^4})\cdots}\\
       &= \frac{\prod (1-x^{3i})}{\prod (1-x^i)}\\
       &= (\frac{1-x^3}{1-x})(\frac{1-x^6}{1-x^2})(\frac{1-x^9}{1-x^3})(\frac{1-x^{12}}{1-x^5})(\frac{1-x^{15}}{1-x^6})\cdots\\
       &= (1+x+x^2)(1+x^2+x^4)(1+x^3+x^6)\cdot\dots = f(x)\\
\end{align*}


\end{document}