\documentclass[a4paper,11pt]{report}

\usepackage{amsmath}
\usepackage{fullpage}
\usepackage{tikz}

\usetikzlibrary{graphs,graphs.standard}

\makeatletter
\pgfmathdeclarefunction{alpha}{1}{%
  \pgfmathint@{#1}%
  \edef\pgfmathresult{\pgffor@alpha{\pgfmathresult}}%
}

\author{Sylvain Julmy}
\date{\today}

\setlength{\parindent}{0pt}

\begin{document}

\begin{center}
  \Large{
    Mathematical Methods for Computer Science 2\
    Spring 2018
  }
  \noindent\makebox[\linewidth]{\rule{\linewidth}{0.4pt}}

  Series 4
  \vspace*{1.4cm}

  Sylvain Julmy
  
  \noindent\makebox[\linewidth]{\rule{\linewidth}{0.4pt}}
\end{center}

\section*{\texttt{1}}

\subsection*{a)}

\begin{align*}
  p(n;k) &= p(n;\leq k) - p(n;\leq k-1) \\
  p(n;\leq k) &= p(n;k) + p(n;\leq k-1)
\end{align*}

By definition, the sum of the number of partition of $n$ in exactly $k$ parts
and the number of partition of $n$ in at most $k-1$ parts is the number of
partition of $n$ in at most $k-1$ parts. The set $P_1$ is the partitions of $n$
in exactly $k$ parts and the set $P_2$ is the partitions of $n$ in at most $k-1$
parts are different : $P_1 \cap P_2 = \emptyset$.

\subsection*{b)}

We know that $p(n-k;\leq k) = p(n;k)$. So, by substition of this equation
in the previous one, we obtain
\[
  p(n;\leq k) = p(n;\leq k-1) + p(n-k;\leq k)
\]

\section*{\texttt{2}}

\subsection*{a)}

\[
  q(n;k) = p(n-\binom{k}{2};k)
\]

At first, $\binom{k}{2}$ is used to compute the area of the top-left triangle
(in blue on the figure provided) :

\[
  \binom{k}{2} = \frac{k!}{(k-2)!2!} = \frac{k(k-1)}{2} = 1 + 2 + \dots + k
\]

Then, because we remove the triangle, we have a partitions of $n -
\binom{k}{2}$, the number of row in the triangle is not touch so there is $k$
parts left.

So there is a bijection from each partition of $q(n;k)$ to $p(n -
\binom{k}{2};k)$.

\section*{\texttt{3}}

The proof is by bijection. We take any partition of $n-l$ and take its Ferrers
diagrams. The diagram has $k-1$ parts and none of them is exceeding $l$.

By adding a new top row of $k$ nodes and deleting the first column (in this
order), we obtain a new diagram with $k$ parts and none of them is exceeding
$l-1$.

Then, by taking the conjugate of the new diagram, we obtain a partition of $n-k$
into exactly $l-1$ parts of size $\leq k$.

It is a one-to-one transformation so the bijection is established.

Note : I have work with Mr. Lauper and Mr. Papinutto for this exercice.

\section*{\texttt{4}}

In order to weight any integer, we can find a way to encode any integer using
the ternary system. We use a representation similar to the encoding in binary,
except that we don't encode power of $2$ with $0$ and $1$, but power of $3$ with
$1$, $0$ and $-1$.

We encode the first $10$ numbers as follow :
\begin{align*}
  0 &= 0\\
  1 &= (1)3^0\\
  2 &= (1)3^1 + (-1)3^0\\
  3 &= (1)3^1\\
  4 &= (1)3^1 + (1)3^0\\
  5 &= (1)3^2 + (-1)3^1 + (-1)3^0\\
  6 &= (1)3^2 + (-1)3^1 + (0)3^0\\
  7 &= (1)3^2 + (-1)3^1 + (1)3^0\\
  8 &= (1)3^2 + (0)3^1 + (-1)3^0\\
  9 &= (1)3^2 + (0)3^1 + (0)3^0\\
\end{align*}

Then, to encode any integer (in base $10$) in this format, we can use the following generating
function.

\[
  (a_na_{n-1}\dots a_1a_0)_{10} = \sum_{k=0}^{\infty} a_k10^k
\]

Where $a_k$ are the digit of the original representation.

\section*{\texttt{5}}

The proof is to show how to encore any integer $n$ into a sum of fibonnaci numbers :
\begin{enumerate}
\item Find the greatest fibonnaci number $f_m$ which is smaller than $n$.
\item If $n - f_m$ is a fibonnaci number, then we are done.
\item Else, we repeat the process until the substraction is $0$.
\end{enumerate}

The encoding does not hold any two consecutive fibonnaci number, because if it
is the case, we can replace both of them by another one due to the recurence
relation of the fibonnaci numbers : $a_{n-2} + a_{n-1} = a_n$

\end{document}