\documentclass[a4paper,11pt]{report}

\usepackage{amsmath}
\usepackage{fullpage}
\usepackage{tikz}

\usepackage{bussproofs}
\usepackage{mathpartir}
\usepackage{prooftrees}
\usepackage{color}

\makeatletter
\pgfmathdeclarefunction{alpha}{1}{%
  \pgfmathint@{#1}%
  \edef\pgfmathresult{\pgffor@alpha{\pgfmathresult}}%
}

\author{Sylvain Julmy}
\date{\today}

\setlength{\parindent}{0pt}

\begin{document}

\begin{center}
  \Large{
    Mathematical Methods for Computer Science 2\
    Fall 2017
  }
  \noindent\makebox[\linewidth]{\rule{\linewidth}{0.4pt}}

  Series 5
  \vspace*{1.4cm}

  Sylvain Julmy
  
  \noindent\makebox[\linewidth]{\rule{\linewidth}{0.4pt}}
\end{center}

\section*{\texttt{1}}

Exercice $1$ is in appendix.

\section*{\texttt{2}}

\subsection*{a)}
\[
  (f(x))^2 = \frac{f(x)-1}{x}
\]

\begin{align*}
  x \cdot (f(x))^2 &= x \cdot \sum_{n=0}^{\infty} (C_0C_n + C_1C_{n-1} + \dots + C_nC_0)x^n\\
                   &= \cdot \sum_{n=0}^{\infty} (C_0C_n + C_1C_{n-1} + \dots + C_nC_0)x^{n+1}\\
                   &= \sum_{n=0}^{\infty} C_{n+1}x^{n+1}\\
                   &= f(x)-1
\end{align*}

Then
\[
  (f(x))^2 = \frac{f(x)-1}{x}
\]


\subsection*{b)}
\[
  f(x) = \frac{1 - \sqrt{1-4x}}{2x}
\]

From $(f(x))^2$ we get that $x \cdot (f(x))^2 - f(x) + 1 = 0$ , then

\begin{gather*}
  \Delta = (-1)^2 - 4 \cdot x \cdot 1 = 1 - 4x \\
  f(x)_{1,2} = \frac{1 \pm \sqrt{1-4x}}{2x}
\end{gather*}

We consider only the ``-'' sign :

\begin{gather*}
  \lim_{x \to 0} \frac{1 + \sqrt{1-4x}}{2x} = \lim_{x \to 0}
  \frac{2(1-4x)^{-\frac{1}{2}}}{2} = \frac{2 \cdot (1)^{-\frac{1}{2}}}{2} =
  \frac{2}{2} = 1
\end{gather*}

Finally, we have

\[
  f(x) = \frac{1-\sqrt{1-4x}}{2x}
\]

\subsection*{c)}

\[
  f(x) = \frac{1-\sqrt{1-4x}}{2x}
\]

\begin{gather*}
  \sqrt{1-4x} = [y/-4x] (1+y)^{\frac{1}{2}} \\
  (1+y)^{\frac{1}{2}} = \sum_{k = 0}^{n} \binom{\frac{1}{2}}{k}y^k
\end{gather*}

Then we have

\begin{align*}
  \binom{\frac{1}{2}}{k} &= \frac{\frac{1}{2}(\frac{1}{2}-1)(\frac{1}{2}-2)\cdots(\frac{1}{2}-k+1)}{k!}\\
                         &= \frac{1(-1)(-3)\dots(3-2k)}{2^kk!}\\
                         &= \frac{(-1)^{k-1}}{2^kk!}(2k-3)!!\\
                         &= \frac{(-1)^{k-1}}{2^kk!(2k-1)}(2k-1)!!\\
                         &= \frac{(-1)^{k-1}}{2^kk!(2k-1)}\cdot\frac{(2k)!}{(2k)!!}\\
                         &= \frac{(-1)^{k-1}}{2^kk!(2k-1)}\cdot\frac{(2k)!}{2^kk!}\\
                         &= \frac{(-1)^{k-1}}{4^k(2k-1)}\cdot\frac{(2k)!}{k!k!}\\
                         &= \frac{(-1)^{k-1}}{4^k(2k-1)}\binom{2n}{n}\\
\end{align*}

So we have

\begin{align*}
  \frac{1 + \sqrt{1-4x}}{2x} &= \frac{1 + \sum_{n=0}^{\infty}{\binom{2n}{n}\frac{x^n}{2n-1}}}{2x} \\
                             &= \frac{1+(-1) + \sum_{n=1}^{\infty}\binom{2n}{n}\frac{x^n}{2n-1}}{2x}\\
                             &= \sum_{n=1}^{\infty}\binom{2n}{n}\frac{x^{n-1}}{(2n-1)2}\\
                             &= \sum_{n=0}^{\infty} \binom{2n+2}{n+1}\frac{x^n}{2(2n-1)}\\
                             &= \sum_{n=0}^{\infty} \frac{(2n+2)!}{(n+1)!(n+1)!}\frac{x^n}{2(2n-1)}\\
                             &= \sum_{n=0}^{\infty} \frac{(2n+2)(2n+1)(2n)!x^n}{(n+1)(n+1)(n!)(n!)(2n+1)2}\\
                             &= \sum_{n=0}^{\infty} \binom{2n}{n}\frac{x^n}{n+1}\\
\end{align*}

Which is the generating function for Catalan number.

\section*{\texttt{3}}

We show a bijection between a rooted tree of $n+1$ vertices to a Dyck path. We
already know that the number of Dyck path from $(0,0)$ to $(n,n)$. So showing
the bijection show that the number of rooted trees on $n+1$ vertices is equal to
$C_n$.

To encode a rooted tree into a Dyck path, we start from the root of the tree and
we traverse the tree in depth first and coming back to the root at the end.
Each time we increase the depth (go to a
child from its parent), we note a $0$ for the Dyck path and each time we
decrease the depth (go to the parent of a child), we note a $1$. Clearly, the
number of $1$ can't be greater than then number of $0$ at any point in the
sequence because we can't go back to a parent without going to its children.

For example, for the following tree

\begin{forest}
  [$x_1$
  [$x_2$ [$x_3$][$x_4$]]
  [$x_5$]
  ]
\end{forest}

We have the walk $x_1x_2x_3x_2x_4x_2x_1x_5x_1$ which is the following encoding
for a Dyck path : $00101101$.

We use the inverse algorithm to construct a rooted tree from a Dyck path : we
start from the root of the rooted tree and use the symbol of the Dyck path. When
there is a $0$, we create a left child (or a right child if a left child already
exist) and go to it. When there is a $1$, we simply go back to the parent.

For example, the following Dyck path $0001011011001011$ produce the following
rooted tree :

\begin{forest}
  [$\bullet$
  [$\bullet$ 
  [$\bullet$ [$\bullet$] [$\bullet$] ]
  [$\bullet$]
  ]
  [$\bullet$ [$\bullet$] [$\bullet$]]
  ]
\end{forest}

Clearly, such a walk for a rooted tree is unique for a tree, so the encoding of
the Dyck path is unique too.

\section*{\texttt{4}}

We show a bijection between a table of handshake without crossing arms and a
bracket-expression. Counting the number of bracket-expression is the same as
$C_n$ so the number of table of handshake with $2n$ personn is $C_n$ too.

For any ways of arranging handshake between people, we can construct a
bracket-expression. We start from any person across the table and turn
counter-clockwise to visit every point across the table.

When visiting a point, we maintain a list of visited points. There are two cases :
\begin{itemize}
\item If the point is handshaked by a point which is already present in the
  list, we add a closed bracket to our construction.
\item Otherwise, add an open bracket to our construction.
\end{itemize}

In order to encode a bracket-expression into a table of handshake, we use a
table of $2n$ points and we pick a starting point. We visit each point in the
counter-clockwise order. We maintain a stack (we remember the point in the order
of their visit time, the closest visited point in time is the first one)
of the point we visit. When visiting the points and iterating over the
bracket-expressions, there are two cases :
\begin{itemize}
\item If the current item is an opening bracket, we just push it on the stack.
\item If the current item is a closing bracket, we match it with the one we pop
  out of the stack.
\end{itemize}

Any bracket-expression can be converted to a table of handshake and the invert
too.

\section*{\texttt{5}}

The proof is by induction on the structure. There is two distinct base case :
\begin{itemize}
\item $x_0$ and  no variables :
  \begin{enumerate}
  \item we don't add any dot operator but add the bracket, so we have $(x_0)$.
  \item we remove everything except the closing bracket, so we have $)$.
  \item we replace nothing by an opening bracket.
  \item finally we obtain $)$.
  \end{enumerate}
  That's corresponds to $C_1 = 1$ and $C_0 = 1$.
\item $(x_0x_1)$ : we apply the transformation and obtain :
  \[
    (x_0x_1) \mapsto (x_0 \cdot x_1) \mapsto \cdot ) \mapsto ()
  \]
  Which corresponds to $C_2 = 2$.
\end{itemize}

For the inductive step, we consider a bracket-variable expression $e$ which is
of one of the following form :
\begin{itemize}
\item One of the base case : no variable, $x_i$ or $x_ix_{i+1}$.
\item $e = ((x_ix_{i+1})e_k)$
\item $e = (e_k(x_ix_{i+1}))$
\item $e = (x_ix_{i+1})(x_{i+2}x_{i+3})$
\end{itemize}

Where $i$ and $k$ are any possible indices. We denote by $enc(e)$ the encoding
of the bracket-variable expression $e$ using the presented algorithm. Then, for
each of the cases mentionned :

\paragraph{$e = ((x_ix_{i+1})e_k)$ :}
\[
  ((x_ix_{i+1})e_k) \mapsto ((x_i \cdot x_{i+1}) \cdot enc(e_k)) \mapsto \cdot )
  \cdot enc(e_k) ) \mapsto ()(enc(e_k))
\]

\paragraph{$e = (e_k(x_ix_{i+1}))$ :}
\[
  (e_k(x_ix_{i+1})) \mapsto (enc(e_k) \cdot (x_i \cdot x_{i+1})) \mapsto
  enc(e_k) \cdot  \cdot )) \mapsto enc(e_k)(())
\]

\paragraph{$e = (e_k)(e_{k+1})$ :}
\[
  (enc(e_k))(enc(e_{k+1})) \mapsto enc(e_k))enc(e_{k+1}))
\]
In this case, the encoding of both expression will produce an opening bracket.

In the inductive step, each encoding of the expression produces different
results so the encoding of a bracket-variable expression is uniquely defined.

We use the reverse algorithm to encode bracket expression into bracket-variable
expression :

\paragraph{Base case :} $() \mapsto \cdot ) \mapsto (x_i \cdot x_{i+1}) \mapsto
(x_ix_{i+1})$

The inductive step is the same as before but in the other direction, each
encoding is unique.

\end{document}