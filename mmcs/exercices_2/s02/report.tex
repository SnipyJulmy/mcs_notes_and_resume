\documentclass[a4paper,11pt]{report}

\usepackage{amsmath}
\usepackage{fullpage}

\author{Sylvain Julmy}
\date{\today}

\setlength{\parindent}{0pt}

% Tikz
\usepackage{tikz}
\usetikzlibrary{arrows}
\usetikzlibrary{shapes}
\newcommand*\circled[1]{\tikz[baseline=(char.base)]{
            \node[shape=circle,draw,inner sep=2pt] (char) {#1};}}

\begin{document}

\begin{center}
  \Large{
    Mathematical Methods for Computer Science 2\\
    Spring 2018
  }
  \noindent\makebox[\linewidth]{\rule{\linewidth}{0.4pt}}

  Series 2
  \vspace*{1.4cm}

  Sylvain Julmy
  
  \noindent\makebox[\linewidth]{\rule{\linewidth}{0.4pt}}
\end{center}

\section*{\texttt{1}}

\subsection*{a)}

We have
\[
  (1-x + 2x^3) = (a_0 + a_1 x + a_2 x^2 + a_3 x^3 + \dots)
\]

we use the following equation to find $a_0$, $a_1$, $a_2$ and $a_3$

\[
  (1-x+2x^3)(a_0 + a_1 x + a_2 x^2 + a_3 x^3 + \dots) = 1
\]

\begin{align*}
  & (1-x+2x^3)(a_0 + a_1 x + a_2 x^2 + a_3 x^3 + \dots) = \\
  & a_0 - a_0 x + 2 a_0 x^3 + \\
  & a_1 x - a_1 x^2 + 2 a_1 x^4 + \\
  & a_2 x^2 - a_2 x^3 + 2 a_2 x^5 + \\
  & a_3 x^3 - a_3 x^4 + 2 a_3 x^6 + \\
  & a_4 x^4 - a_4 x^5 + 2 a_4 x^7 + \dots = 1
\end{align*}

So we can extract the following system :

\[
  \begin{cases}
    a_0 = 1 \\
    -a_0 x + a_1 x = 0 \\
    -a_1 x^2 + a_2 x^2 = 0 \\
    2 a_0 x^3 - a_2 x^3 + a_3 x^3 = 0
  \end{cases}
\]

which leads to the following :

\begin{gather*}
  a_0 = 1 \\
  -1 x + a_1 x = 0 \longrightarrow a_1 = 1 \\
  -x^2 + a_2 x^2 = 0 \longrightarrow a_2 = 1 \\
  2x^3 - x^3 + a_3 x^3 = 0 \longrightarrow a_3 = -1
\end{gather*}

\section*{\texttt{2}}

\subsection*{a)}

\begin{align*}
  a_0 &= 1 \qquad a_{n+1} = 2 a_n + 3 \\
  a_1 &= 5 \\
  a_2 &= 13 \\
  a_3 &= 29 \\
  \dots \\
  a_{n+1} - 2a_n &= 3
\end{align*}

\begin{align*}
  F(x) &= a_0 + a_1 x + a_2 x^2 + \dots \\
  x F(x) &=     a_0 x + a_1x^2 + \dots \\
  F(x) - 2 x F(x) &= (a_0 + a_1 x + a_2 x^2 + \dots ) - (2 a_0 x + 2 a_1 x^2 + \dots) \\
       &= a_0 + (a_1 - 2 a_0) x + (a_2 - 2 a_1) x^2 + \dots \\
       &= 1 + \underbrace{(5-2)}_3 x + \underbrace{(13-10)}_3 x^2 + \dots \\
       &= 1 + 3x + 3x^2 + \dots \\
       &= 1 + 3x(1 + x + x^2 + x^3 + \dots) \\
       &= 1 + \frac{3x}{1-x} = \frac{(1-x) + 3x}{1-x} = \frac{2x+1}{1-x}
\end{align*}

\begin{align*}
  F(x) = \frac{2x+1}{(1-x)(1-2x)} = \frac{A}{1-x} + \frac{B}{1-2x} \\
  2x + 1 = A(1-2x) + B(1-x) = A - 2Ax + B - Bx = 2x + 1
\end{align*}

We extract the following system :

\[
  \begin{cases}
    A + B = 1 \\
    -2 A - B = 2
  \end{cases}
\]

\begin{gather*}
  A = 1 - B \\
  (-2)(1-B) - B  = 2 \\
  -2 + 2B -B = 2 \\
  B = 4 \\
  A = 1 - 4 = -3
\end{gather*}

\[
  F(x) = \frac{-3}{1-x} + \frac{4}{2x+1} =
\]

\subsection*{b)}

\begin{align*}
  F(x) &= \frac{-3}{1-x} + \frac{4}{2x+1} = -3 \frac{1}{1-x} + 4 \frac{1}{2x+1} \\
       &= -3 \frac{1}{1-x} + 4 \frac{1}{1-(-2x)} \\
       &= -3 \sum_{k=0}^{\infty} x^k + 4 \sum_{k=0}^{\infty} (-1)^k\ 2^kx^k \\
       &= 4 \sum_{k=0}^{\infty} (-1)^k\ 2^kx^k - 3 \sum_{k=0}^{\infty} x^k \\
       &= \sum_{k=0}^{\infty} (-1)^k \cdot 4 \cdot 2^kx^k - \sum_{k=0}^{\infty} 3 \cdot 1^k \cdot x^k \\
       &= \sum_{k=0}^{\infty} 4(2^k-1) - 3 x^k
\end{align*}

\[
  a_n = 4(2^n - 1) + 1
\]

\section*{\texttt{3}}

\[
  \frac{x^2 + x}{(x-1)(x+2)} = \frac{A}{(x+2)} + \frac{B}{(x+2)^2} + \frac{C}{(x-1)}
\]

\begin{align*}
  x^2 + x &= A(x-1)(x+2) + B(x-1) + C(x+2)^2 \\
          &= A(x^2 -2 + x) + Bx + B + C(x^2 + 4 + 4x)\\
          &= Ax^2 + Ax - 2A + Bx -B+Cx^2 + 4C+ 4Cx\\
\end{align*}

We extract the following system :

\[
  \begin{cases}
    -2A - B + 4C = 0 \\
    A + B + 4C = 1 \\
    A + C = 1
  \end{cases}
\]

\begin{gather*}
  A + C = 1 \longrightarrow A = 1 - C \\
  \longrightarrow \\
  \begin{cases}
    -2 + 2C - B + 4C = 0 \\
    1 + 3C + B = 1 \\
  \end{cases}\\
  \longrightarrow \\
  B = -3C \\
\end{gather*}

\begin{gather*}
  -2 + 2C + 3C + 4C = 0 \\
  9C = 2 \\
  C = \frac{2}{9}\\
\end{gather*}

\begin{gather*}
  B = -3 \frac{2}{9} = -\frac{6}{9} = - \frac{2}{3}\\
  A = 1 - C \\
  A = 1 - \frac{2}{9} = \frac{7}{9}
\end{gather*}

\[
  \frac{x^2 + x}{(x-1)(x+2)} = \frac{7}{9(x+2)} - \frac{2}{3(x+2)^2} + \frac{2}{9(x-1)}
\]

\section*{\texttt{4}}

\subsection*{a)}

\begin{align*}
  (2n - 1)!! &= \frac{(2n)!}{2^nn!} \\
  \longrightarrow \\
  (2n - 1)!! 2^n n! &= (2n!)
\end{align*}

\begin{align*}
  (2n-1) !! 2^n n! &= \left((2n-1)(2n-3)\dots(1)\right) \cdot 2^n \cdot n! \\
  \longrightarrow \\
  2^n &= \underbrace{2 \cdot 2 \cdot 2 \cdot 2 \cdot \dots \cdot 2}_{n} \\
  n! &= \underbrace{1 \cdot 2 \cdot 3 \cdot 4  \cdot \dots \cdot n}_{n} \\
  2^n \cdot n! &= \underbrace{2 \cdot 4 \cdot 6 \cdot 8 \cdot \dots \cdot 2n}_{n} = (2n)(2n-2)(2n-4)\dots(2) \\
  (2n-1) !! 2^n n! &= (2n-1)(2n-2)(2n-3)(2n-4)\dots(2)(1) = (2n)!
\end{align*}

Therefore

\[
  (2n - 1)!! = \frac{(2n)!}{2^nn!} 
\]

\subsection*{b)}

\[
  \sqrt{1+x} = (1+x)^{\frac{1}{2}} = \sum_{k=0}^\infty \binom{\alpha}{k}x^k = 1
  + \sum_{k=1}^\infty \binom{\alpha}{k}x^k
\]

\begin{align*}
  \binom{\alpha}{k} = \binom{\frac{1}{2}}{k}
  = \frac{\frac{1}{2} (\frac{1}{2} - 1) (\frac{1}{2} - 3) \dots ((\frac{1}{2} - k + 1))}{k!}
\end{align*}

We expand the sum :

\begin{gather*}
  1 - \frac{(\frac{1}{2}) (\frac{1}{2})}{1!} + \frac{(\frac{1}{2})
    (\frac{1}{2})(\frac{3}{2})}{2!} - \frac{(\frac{1}{2})
    (\frac{1}{2})(\frac{3}{2})(\frac{5}{2})}{3!} + \dots = \\
  1 - \frac{(-1)!!}{1! * 2^1} + \frac{1!!}{2! * 2^2} - \frac{3!!}{3! 2^3} +
  \dots =
  1 + \sum_{k=1}^{\infty} \frac{(-1)^{k-1} (2k-3)!!}{k! 2^k}
\end{gather*}

\subsection*{c)}

\begin{align*}
  a_0 &= 1 \\
  a_1x &= -\frac{1}{2} x \\
  a_2x^2 &= \frac{1}{8} x^2 
\end{align*}

\section*{\texttt{5}}

\subsection*{a)}

We know that
\[
  \frac{1}{1-x} = 1 + x + x^2 + x^3 + \dots
\]

So we have

\begin{align*}
  \frac{d}{dx} (\frac{1}{1-x}) &= \frac{1}{(1-x)^2} = 1 + 2x + 3x^2 + 4x^3 + \dots \\
  \frac{d}{dx} (\frac{1}{1+x}) &= \frac{-1}{(1+x)^2} = -1 + 2x - 3x^2 + 4x^3 - \dots \\
  \frac{1}{(1-x)^2} + \frac{1}{(1+x)^2}&= 4x + 8x^3 + 12x^5 + 16x^7 + \dots \\ 
                               &= 4x\underbrace{(1 + 2x^2 + 3x^4 + 4x^6)}_{F(x)} \\
\end{align*}

\begin{align*}
  F(x) &= \frac{1}{4x}( \frac{1}{(1-x)^2} + \frac{1}{(1+x)^2}) \\
       &= \frac{1}{4x}(\frac{(1+x)^2 - (1-x)^2}{((1-x)^2)((1+x)^2)}) \\
       &= \frac{1}{4x}(\frac{1 + x^2 + 2x - 1 - x^2 + 2x}{((1-x)^2)((1+x)^2)})\\
       &= \frac{1}{4x}(\frac{4x}{((1-x)^2)((1+x)^2)})\\
       &= \frac{1}{((1-x)^2)((1+x)^2)}\\
       &= \frac{1}{(1 + x^2 - 2x)(1 + x^2 + 2x)}\\
       &= \frac{1}{1 + x^2 + 2x + x^2 + x^4 + 2x^3 -2x -2x^3 - 4x^2}\\
       &= \frac{1}{1 + -2x^2 + x^4}\\
       &= \frac{1}{(1-x^2)^2}\\
\end{align*}

\[
  1 - 2x + 3x^2 - 4x^3 + 5x^4 - \dots = \frac{d}{dx} (\frac{1}{1+x}) = -\frac{1}{(1+x)^2}
\]

\subsection*{b)}

\[
  \underbrace{1 + 2x + 3x² + 4x^3 + \dots}_{\frac{1}{(1-x)^2}} \cdot
  \underbrace{1 - 2x + 3x^2 -4x^3 + \dots}_{\frac{1}{(1+x)^2}} =
  \underbrace{1 + 2x^2 + 3x^4 + 4x^6}_{\frac{1}{(1-x^2)^2}}
\]

Hence

\begin{align*}
  (1-x)^2 \cdot (1+x)^2 &= (1 + x^2 + 2x) \cdot (1 + x^2 - 2x) \\
                        &= (1-x^2)^2
\end{align*}

\end{document}


