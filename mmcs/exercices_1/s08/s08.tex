\documentclass[a4paper,11pt]{report}

\usepackage{amsmath}
\usepackage{fullpage}
\usepackage{tikz}

\usetikzlibrary{graphs,graphs.standard}

\makeatletter
\pgfmathdeclarefunction{alpha}{1}{%
  \pgfmathint@{#1}%
  \edef\pgfmathresult{\pgffor@alpha{\pgfmathresult}}%
}

\usepackage{bussproofs}
\usepackage{mathpartir}
\usepackage{prooftrees}
\usepackage{color}

\newcommand*{\contract}[2]{contraction of $#1$ with $#2$}

\author{Sylvain Julmy}
\date{\today}

\setlength{\parindent}{0pt}

\begin{document}

\begin{center}
  \Large{
    Mathematical Methods for Computer Science 1\
    Fall 2017
  }
  \noindent\makebox[\linewidth]{\rule{\linewidth}{0.4pt}}

  Series 8
  \vspace*{1.4cm}

  Sylvain Julmy
  
  \noindent\makebox[\linewidth]{\rule{\linewidth}{0.4pt}}
\end{center}

\section*{\texttt{1}}

\subsection*{(a)}

In order to show that $\{\rightarrow,\neg\}$ is complete, we have to found the
equivalent formula of $a \vee b$ and $a \wedge b$ ($\neg a$ and $a \rightarrow b$
are trivial...).

\subsubsection*{$a \vee b$}

\begin{gather*}
  (\neg a) \rightarrow b = (\neg \neg a) \vee b = a \vee b
\end{gather*}

and the truth table to validate the result :

\begin{center}
  \begin{tabular}{@{ }c@{ }@{ }c | c@{ }@{ }c@{ }@{ }c@{ }@{ }c@{ }@{ }c@{ }@{ }c}
    a & b &  & $\neg$ & a & $\rightarrow$ & b & \\
    \hline 
    T & T &  & F & T & \textcolor{red}{T} & T & \\
    T & F &  & F & T & \textcolor{red}{T} & F & \\
    F & T &  & T & F & \textcolor{red}{T} & T & \\
    F & F &  & T & F & \textcolor{red}{F} & F & \\
  \end{tabular}
\end{center}

\subsubsection*{$a \wedge b$}

\begin{align*}
  \neg (a \rightarrow \neg b) &= \neg (\neg a \vee \neg b) \\
                              & = \neg \neg (a \wedge b) \\
                              &= (a \wedge b)
\end{align*}

and the truth table to validate the result :

\begin{center}
  \begin{tabular}{@{ }c@{ }@{ }c | c@{ }@{}c@{}@{ }c@{ }@{ }c@{ }@{ }c@{ }@{ }c@{ }@{}c@{ }}
    a & b & $\neg$ & ( & a & $\rightarrow$ & $\neg$ & b & )\\
    \hline 
    T & T & \textcolor{red}{T} &  & T & F & F & T & \\
    T & F & \textcolor{red}{F} &  & T & T & T & F & \\
    F & T & \textcolor{red}{F} &  & F & T & F & T & \\
    F & F & \textcolor{red}{F} &  & F & T & T & F & \\
  \end{tabular}
\end{center}

\subsection*{(b)}

In order to show that $\{\uparrow\}$ is complete, we have to found the
equivalent formula of $a \rightarrow b$ and $\neg a$, because we just showed
that $\{\neg,\rightarrow\}$ is complete.

We know that $a \uparrow b \leftrightarrow \neg (a \wedge b)$.

\subsubsection*{$\neg a$}

\[
  a \uparrow a = \neg (a \wedge a) = \neg a
\]

and the truth table to validate the result :
\begin{center}
  \begin{tabular}{@{ }c | c@{ }@{}c@{}@{ }c@{ }@{ }c@{ }@{ }c@{ }@{}c@{ }}
    a & $\neg$ & ( & a & $\&$ & a & )\\
    \hline 
    T & \textcolor{red}{F} &  & T & T & T & \\
    F & \textcolor{red}{T} &  & F & F & F & \\
  \end{tabular}
\end{center}

\subsubsection*{$a \rightarrow b$}

\begin{align*}
  a \rightarrow b &= \neg a \vee b\\
                  &= \neg \neg (\neg a \vee b)\\
                  &= \neg (a \wedge \neg b)\\
                  &= \neg (a \wedge \underbrace{\neg (b \wedge b)}_{b \uparrow b})\\
                  &= \underbrace{\neg (a \wedge (b \uparrow b))}_{a \uparrow (b \uparrow b)}\\
                  &= a \uparrow (b \uparrow b)
\end{align*}

and the truth table to validate the result :

\begin{center}
  \begin{tabular}{@{ }c@{ }@{ }c | c@{ }@{ }c@{ }@{ }c@{ }@{}c@{}@{ }c@{ }@{ }c@{ }@{ }c@{ }@{}c@{}@{ }c}
    a & b &  & a & $\uparrow$ & ( & b & $\uparrow$ & b & ) & \\
    \hline 
    T & T &  & T & \textcolor{red}{T} &  & T & F & T &  & \\
    T & F &  & T & \textcolor{red}{F} &  & F & T & F &  & \\
    F & T &  & F & \textcolor{red}{T} &  & T & F & T &  & \\
    F & F &  & F & \textcolor{red}{T} &  & F & T & F &  & \\
  \end{tabular}
\end{center}

\section*{\texttt{2}}

Note : we denote $\top$ the formula which represent $true$ and $\bot$ the
formula which represent $false$ using the following equivalence :

\begin{align*}
  \top &= (\neg A \vee A) \\
  \bot &= (\neg A \wedge A)
\end{align*}

We can also use the following equivalence :

\[
  (A \wedge B) \vee A \leftrightarrow A
\]

We can also use the following equivalence :

\[
  (A \wedge B) \vee A \leftrightarrow A
\]

and use thr truth table for proving the equivalence :

\begin{center}
  \begin{tabular}{@{ }c@{ }@{ }c | c@{ }@{}c@{}@{}c@{}@{ }c@{ }@{ }c@{ }@{ }c@{ }@{}c@{}@{ }c@{ }@{ }c@{ }@{}c@{}@{ }c@{ }@{ }c@{ }@{ }c}
    a & b &  & ( & ( & a & $\&$ & b & ) & $\lor$ & a & ) & $\leftrightarrow$ & a & \\
    \hline 
    1 & 1 &  &  &  & 1 & 1 & 1 &  & 1 & 1 &  & \textcolor{red}{1} & 1 & \\
    1 & 0 &  &  &  & 1 & 0 & 0 &  & 1 & 1 &  & \textcolor{red}{1} & 1 & \\
    0 & 1 &  &  &  & 0 & 0 & 1 &  & 0 & 0 &  & \textcolor{red}{1} & 0 & \\
    0 & 0 &  &  &  & 0 & 0 & 0 &  & 0 & 0 &  & \textcolor{red}{1} & 0 & \\
  \end{tabular}
\end{center}

\subsection*{(a)}

\begin{align*}
  (p \rightarrow q) \wedge ((q \vee r) \rightarrow p) &= (\neg p \vee q) \wedge (\neg(q \vee r) \vee p) \\
                                                      &= (\neg p \vee q) \wedge ((\neg q \wedge \neg r) \vee p)\\
                                                      &= (\neg p \vee q) \wedge (p \vee \neg q) \wedge (p \vee \neg r)\\
                                                      &= ((p \wedge (\neg p \vee q)) \vee (\neg q \wedge (\neg p \vee q))) \wedge (p \vee \neg r)\\
                                                      &= (\underbrace{(p \wedge \neg p)}_{\bot} \vee (p \wedge q) \vee (\neg q \wedge \neg p) \vee \underbrace{(\neg q \wedge q)}_{\bot}) \wedge (p \vee \neg r)\\
                                                      &= ((p \wedge q)\vee(\neg q \wedge \neg p)) \wedge (p \vee \neg r)\\
                                                      &= ((p \vee \neg r) \wedge (\neg q \wedge \neg p)) \vee ((p \vee \neg r) \wedge (p \wedge q))\\
                                                      &= \underbrace{(\neg q \wedge \neg p \wedge p)}_{\bot} \vee (\neg q \wedge \neg r \wedge \neg p)\vee \underbrace{(p \wedge q \wedge p)}_{p \wedge q} \vee (p \wedge q \wedge \neg r)\\
                                                      &= (\neg q \wedge \neg r \wedge \neg p) \vee (p \wedge q) \vee (p \wedge q \wedge \neg r)\\
                                                      &= (\neg q \wedge \neg r \wedge \neg p) \vee (p \wedge q)
\end{align*}

\subsection*{(b)}

Truth table of $\phi = (p \rightarrow q) \wedge ((q \vee r) \rightarrow p)$ :

\begin{center}
  \begin{tabular}{@{ }c@{ }@{ }c@{ }@{ }c | c@{ }@{}c@{}@{ }c@{ }@{ }c@{ }@{ }c@{ }@{}c@{}@{ }c@{ }@{}c@{}@{}c@{}@{ }c@{ }@{ }c@{ }@{ }c@{ }@{}c@{}@{ }c@{ }@{ }c@{ }@{}c@{}@{ }c}
    p & q & r &  & ( & p & $\rightarrow$ & q & ) & $\&$ & ( & ( & q & $\lor$ & r & ) & $\rightarrow$ & p & ) & \\
    \hline 
    1 & 1 & 1 &  &  & 1 & 1 & 1 &  & \textcolor{red}{1} &  &  & 1 & 1 & 1 &  & 1 & 1 &  & \\
    1 & 1 & 0 &  &  & 1 & 1 & 1 &  & \textcolor{red}{1} &  &  & 1 & 1 & 0 &  & 1 & 1 &  & \\
    1 & 0 & 1 &  &  & 1 & 0 & 0 &  & \textcolor{red}{0} &  &  & 0 & 1 & 1 &  & 1 & 1 &  & \\
    1 & 0 & 0 &  &  & 1 & 0 & 0 &  & \textcolor{red}{0} &  &  & 0 & 0 & 0 &  & 1 & 1 &  & \\
    0 & 1 & 1 &  &  & 0 & 1 & 1 &  & \textcolor{red}{0} &  &  & 1 & 1 & 1 &  & 0 & 0 &  & \\
    0 & 1 & 0 &  &  & 0 & 1 & 1 &  & \textcolor{red}{0} &  &  & 1 & 1 & 0 &  & 0 & 0 &  & \\
    0 & 0 & 1 &  &  & 0 & 1 & 0 &  & \textcolor{red}{0} &  &  & 0 & 1 & 1 &  & 0 & 0 &  & \\
    0 & 0 & 0 &  &  & 0 & 1 & 0 &  & \textcolor{red}{1} &  &  & 0 & 0 & 0 &  & 1 & 0 &  & \\
  \end{tabular}
\end{center}

We pick the row where $\phi = 1$ and we construct the DNF form :

\begin{gather*}
  (p \wedge q \wedge r) \vee (\neg p \wedge \neg q \wedge \neg r) \vee (p \wedge
  q) \\ = \\ (\neg p \wedge \neg q \wedge \neg r) \vee (p \wedge
  q)
\end{gather*}

Because

\[
  (p \wedge q \wedge r) \vee (p \wedge r) \leftrightarrow  (p \wedge r)
\]

\section*{\texttt{3}}

Using De Morgan's law, $\neg(DNF) = CNF$ and $A$ is already in $DNF$.

\subsection*{(a)}

\[
  A = (p \wedge q \wedge \neg r) \vee (p \wedge \neg q \wedge r) \vee (\neg p
  \wedge \neg q \wedge \neg r)
\]

\begin{align*}
  \neg A &= \neg((p \wedge q \wedge \neg r) \vee (p \wedge \neg q \wedge r) \vee (\neg p
           \wedge \neg q \wedge \neg r))\\
         &= (\neg(p \wedge q \wedge \neg r) \wedge \neg(p \wedge \neg q \wedge r) \wedge \neg(\neg p \wedge \neg q \wedge \neg r))\\
         &= (\neg p \vee \neg q \vee r) \wedge (\neg p \vee q \vee \neg r) \wedge (p \vee q \vee r)
\end{align*}

\subsection*{(b)}

We can turn out the formulation into the following formula :

\begin{align*}
         &(A \leftrightarrow B) && \text{$A$ is equivalent to $B$}\\
  \wedge &(A \leftrightarrow (C \vee F)) && \text{$A$ contains a sub-formula $F$}\\
  \wedge &(A \leftrightarrow (C \vee G)) && \text{$B$ is the sub-formula $A$ with $F$ replace by $B$}\\
  \rightarrow &(G \leftrightarrow F) && \text{implies that $F$ is equivalent to $G$}
\end{align*}

Using the following truth table, we can see that the formula is not a tautology,
because $F$ and $G$ can be different in the formula but $A$ and $B$ still
remains equivalent.

\begin{center}
  \begin{tabular}{@{ }c@{ }@{ }c@{ }@{ }c@{ }@{ }c@{ }@{ }c | c@{ }@{}c@{}@{}c@{}@{ }c@{ }@{ }c@{ }@{ }c@{ }@{}c@{}@{ }c@{ }@{}c@{}@{}c@{}@{ }c@{ }@{ }c@{ }@{}c@{}@{ }c@{ }@{ }c@{ }@{ }c@{ }@{}c@{}@{}c@{}@{ }c@{ }@{}c@{}@{ }c@{ }@{ }c@{ }@{}c@{}@{ }c@{ }@{ }c@{ }@{ }c@{ }@{}c@{}@{}c@{}@{}c@{}@{}c@{}@{ }c@{ }@{}c@{}@{ }c@{ }@{ }c@{ }@{ }c@{ }@{}c@{}@{ }c}
    A & B & C & F & G &  & ( & ( & A & $\leftrightarrow$ & B & ) & $\wedge$ & ( & ( & A & $\leftrightarrow$ & ( & C & $\vee$ & F & ) & ) & $\wedge$ & ( & B & $\leftrightarrow$ & ( & C & $\vee$ & G & ) & ) & ) & ) & $\rightarrow$ & ( & G & $\leftrightarrow$ & F & ) & \\
    \hline 
    1 & 1 & 1 & 1 & 1 &  &  &  & 1 & 1 & 1 &  & 1 &  &  & 1 & 1 &  & 1 & 1 & 1 &  &  & 1 &  & 1 & 1 &  & 1 & 1 & 1 &  &  &  &  & \textcolor{red}{1} &  & 1 & 1 & 1 &  & \\
    1 & 1 & 1 & 1 & 0 &  &  &  & 1 & 1 & 1 &  & 1 &  &  & 1 & 1 &  & 1 & 1 & 1 &  &  & 1 &  & 1 & 1 &  & 1 & 1 & 0 &  &  &  &  & \textcolor{red}{0} &  & 0 & 0 & 1 &  & \\
    1 & 1 & 1 & 0 & 1 &  &  &  & 1 & 1 & 1 &  & 1 &  &  & 1 & 1 &  & 1 & 1 & 0 &  &  & 1 &  & 1 & 1 &  & 1 & 1 & 1 &  &  &  &  & \textcolor{red}{0} &  & 1 & 0 & 0 &  & \\
    1 & 1 & 1 & 0 & 0 &  &  &  & 1 & 1 & 1 &  & 1 &  &  & 1 & 1 &  & 1 & 1 & 0 &  &  & 1 &  & 1 & 1 &  & 1 & 1 & 0 &  &  &  &  & \textcolor{red}{1} &  & 0 & 1 & 0 &  & \\
    1 & 1 & 0 & 1 & 1 &  &  &  & 1 & 1 & 1 &  & 1 &  &  & 1 & 1 &  & 0 & 1 & 1 &  &  & 1 &  & 1 & 1 &  & 0 & 1 & 1 &  &  &  &  & \textcolor{red}{1} &  & 1 & 1 & 1 &  & \\
    1 & 1 & 0 & 1 & 0 &  &  &  & 1 & 1 & 1 &  & 0 &  &  & 1 & 1 &  & 0 & 1 & 1 &  &  & 0 &  & 1 & 0 &  & 0 & 0 & 0 &  &  &  &  & \textcolor{red}{1} &  & 0 & 0 & 1 &  & \\
    1 & 1 & 0 & 0 & 1 &  &  &  & 1 & 1 & 1 &  & 0 &  &  & 1 & 0 &  & 0 & 0 & 0 &  &  & 0 &  & 1 & 1 &  & 0 & 1 & 1 &  &  &  &  & \textcolor{red}{1} &  & 1 & 0 & 0 &  & \\
    1 & 1 & 0 & 0 & 0 &  &  &  & 1 & 1 & 1 &  & 0 &  &  & 1 & 0 &  & 0 & 0 & 0 &  &  & 0 &  & 1 & 0 &  & 0 & 0 & 0 &  &  &  &  & \textcolor{red}{1} &  & 0 & 1 & 0 &  & \\
    1 & 0 & 1 & 1 & 1 &  &  &  & 1 & 0 & 0 &  & 0 &  &  & 1 & 1 &  & 1 & 1 & 1 &  &  & 0 &  & 0 & 0 &  & 1 & 1 & 1 &  &  &  &  & \textcolor{red}{1} &  & 1 & 1 & 1 &  & \\
    1 & 0 & 1 & 1 & 0 &  &  &  & 1 & 0 & 0 &  & 0 &  &  & 1 & 1 &  & 1 & 1 & 1 &  &  & 0 &  & 0 & 0 &  & 1 & 1 & 0 &  &  &  &  & \textcolor{red}{1} &  & 0 & 0 & 1 &  & \\
    1 & 0 & 1 & 0 & 1 &  &  &  & 1 & 0 & 0 &  & 0 &  &  & 1 & 1 &  & 1 & 1 & 0 &  &  & 0 &  & 0 & 0 &  & 1 & 1 & 1 &  &  &  &  & \textcolor{red}{1} &  & 1 & 0 & 0 &  & \\
    1 & 0 & 1 & 0 & 0 &  &  &  & 1 & 0 & 0 &  & 0 &  &  & 1 & 1 &  & 1 & 1 & 0 &  &  & 0 &  & 0 & 0 &  & 1 & 1 & 0 &  &  &  &  & \textcolor{red}{1} &  & 0 & 1 & 0 &  & \\
    1 & 0 & 0 & 1 & 1 &  &  &  & 1 & 0 & 0 &  & 0 &  &  & 1 & 1 &  & 0 & 1 & 1 &  &  & 0 &  & 0 & 0 &  & 0 & 1 & 1 &  &  &  &  & \textcolor{red}{1} &  & 1 & 1 & 1 &  & \\
    1 & 0 & 0 & 1 & 0 &  &  &  & 1 & 0 & 0 &  & 0 &  &  & 1 & 1 &  & 0 & 1 & 1 &  &  & 1 &  & 0 & 1 &  & 0 & 0 & 0 &  &  &  &  & \textcolor{red}{1} &  & 0 & 0 & 1 &  & \\
    1 & 0 & 0 & 0 & 1 &  &  &  & 1 & 0 & 0 &  & 0 &  &  & 1 & 0 &  & 0 & 0 & 0 &  &  & 0 &  & 0 & 0 &  & 0 & 1 & 1 &  &  &  &  & \textcolor{red}{1} &  & 1 & 0 & 0 &  & \\
    1 & 0 & 0 & 0 & 0 &  &  &  & 1 & 0 & 0 &  & 0 &  &  & 1 & 0 &  & 0 & 0 & 0 &  &  & 0 &  & 0 & 1 &  & 0 & 0 & 0 &  &  &  &  & \textcolor{red}{1} &  & 0 & 1 & 0 &  & \\
    0 & 1 & 1 & 1 & 1 &  &  &  & 0 & 0 & 1 &  & 0 &  &  & 0 & 0 &  & 1 & 1 & 1 &  &  & 0 &  & 1 & 1 &  & 1 & 1 & 1 &  &  &  &  & \textcolor{red}{1} &  & 1 & 1 & 1 &  & \\
    0 & 1 & 1 & 1 & 0 &  &  &  & 0 & 0 & 1 &  & 0 &  &  & 0 & 0 &  & 1 & 1 & 1 &  &  & 0 &  & 1 & 1 &  & 1 & 1 & 0 &  &  &  &  & \textcolor{red}{1} &  & 0 & 0 & 1 &  & \\
    0 & 1 & 1 & 0 & 1 &  &  &  & 0 & 0 & 1 &  & 0 &  &  & 0 & 0 &  & 1 & 1 & 0 &  &  & 0 &  & 1 & 1 &  & 1 & 1 & 1 &  &  &  &  & \textcolor{red}{1} &  & 1 & 0 & 0 &  & \\
    0 & 1 & 1 & 0 & 0 &  &  &  & 0 & 0 & 1 &  & 0 &  &  & 0 & 0 &  & 1 & 1 & 0 &  &  & 0 &  & 1 & 1 &  & 1 & 1 & 0 &  &  &  &  & \textcolor{red}{1} &  & 0 & 1 & 0 &  & \\
    0 & 1 & 0 & 1 & 1 &  &  &  & 0 & 0 & 1 &  & 0 &  &  & 0 & 0 &  & 0 & 1 & 1 &  &  & 0 &  & 1 & 1 &  & 0 & 1 & 1 &  &  &  &  & \textcolor{red}{1} &  & 1 & 1 & 1 &  & \\
    0 & 1 & 0 & 1 & 0 &  &  &  & 0 & 0 & 1 &  & 0 &  &  & 0 & 0 &  & 0 & 1 & 1 &  &  & 0 &  & 1 & 0 &  & 0 & 0 & 0 &  &  &  &  & \textcolor{red}{1} &  & 0 & 0 & 1 &  & \\
    0 & 1 & 0 & 0 & 1 &  &  &  & 0 & 0 & 1 &  & 0 &  &  & 0 & 1 &  & 0 & 0 & 0 &  &  & 1 &  & 1 & 1 &  & 0 & 1 & 1 &  &  &  &  & \textcolor{red}{1} &  & 1 & 0 & 0 &  & \\
    0 & 1 & 0 & 0 & 0 &  &  &  & 0 & 0 & 1 &  & 0 &  &  & 0 & 1 &  & 0 & 0 & 0 &  &  & 0 &  & 1 & 0 &  & 0 & 0 & 0 &  &  &  &  & \textcolor{red}{1} &  & 0 & 1 & 0 &  & \\
    0 & 0 & 1 & 1 & 1 &  &  &  & 0 & 1 & 0 &  & 0 &  &  & 0 & 0 &  & 1 & 1 & 1 &  &  & 0 &  & 0 & 0 &  & 1 & 1 & 1 &  &  &  &  & \textcolor{red}{1} &  & 1 & 1 & 1 &  & \\
    0 & 0 & 1 & 1 & 0 &  &  &  & 0 & 1 & 0 &  & 0 &  &  & 0 & 0 &  & 1 & 1 & 1 &  &  & 0 &  & 0 & 0 &  & 1 & 1 & 0 &  &  &  &  & \textcolor{red}{1} &  & 0 & 0 & 1 &  & \\
    0 & 0 & 1 & 0 & 1 &  &  &  & 0 & 1 & 0 &  & 0 &  &  & 0 & 0 &  & 1 & 1 & 0 &  &  & 0 &  & 0 & 0 &  & 1 & 1 & 1 &  &  &  &  & \textcolor{red}{1} &  & 1 & 0 & 0 &  & \\
    0 & 0 & 1 & 0 & 0 &  &  &  & 0 & 1 & 0 &  & 0 &  &  & 0 & 0 &  & 1 & 1 & 0 &  &  & 0 &  & 0 & 0 &  & 1 & 1 & 0 &  &  &  &  & \textcolor{red}{1} &  & 0 & 1 & 0 &  & \\
    0 & 0 & 0 & 1 & 1 &  &  &  & 0 & 1 & 0 &  & 0 &  &  & 0 & 0 &  & 0 & 1 & 1 &  &  & 0 &  & 0 & 0 &  & 0 & 1 & 1 &  &  &  &  & \textcolor{red}{1} &  & 1 & 1 & 1 &  & \\
    0 & 0 & 0 & 1 & 0 &  &  &  & 0 & 1 & 0 &  & 0 &  &  & 0 & 0 &  & 0 & 1 & 1 &  &  & 0 &  & 0 & 1 &  & 0 & 0 & 0 &  &  &  &  & \textcolor{red}{1} &  & 0 & 0 & 1 &  & \\
    0 & 0 & 0 & 0 & 1 &  &  &  & 0 & 1 & 0 &  & 0 &  &  & 0 & 1 &  & 0 & 0 & 0 &  &  & 0 &  & 0 & 0 &  & 0 & 1 & 1 &  &  &  &  & \textcolor{red}{1} &  & 1 & 0 & 0 &  & \\
    0 & 0 & 0 & 0 & 0 &  &  &  & 0 & 1 & 0 &  & 1 &  &  & 0 & 1 &  & 0 & 0 & 0 &  &  & 1 &  & 0 & 1 &  & 0 & 0 & 0 &  &  &  &  & \textcolor{red}{1} &  & 0 & 1 & 0 &  & \\
  \end{tabular}
\end{center}

\section*{\texttt{4}}

\subsection*{(a)}

$(p \rightarrow q) \rightarrow ((p \rightarrow \neg q) \rightarrow \neg p)$ is a
tautology :

\begin{align*}
  (p \rightarrow q) \rightarrow ((p \rightarrow \neg q) \rightarrow \neg p)
  &= (\neg p \vee q) \rightarrow ((\neg p \vee \neg q) \rightarrow \neg p) \\
  &= (\neg p \vee q) \rightarrow (\neg (\neg p \vee \neg q) \vee \neg p) \\
  &= (\neg p \vee q) \rightarrow ((p \wedge q) \vee \neg p) \\
  &= (\neg p \vee q) \rightarrow (\underbrace{(\neg p \vee p)}_{\top} \wedge (\neg p \vee q)) \\
  &= (\neg p \vee q) \rightarrow (\neg p \vee q) \\
  &= \top
\end{align*}

$A \rightarrow A$ is a tautology :

\[
  A \rightarrow A = \neg A \vee A = \top
\]

So $(\neg p \vee q)$ can be subsitute by $A$, then we would have $A \rightarrow A$. 

\subsection*{(b)}

$\neg(p \rightarrow q) \vee (\neg p \vee q)$ is a tautology :

\begin{align*}
  \neg(p \rightarrow q) \vee (\neg p \vee q)
  &= \neg(\neg p \vee q) \vee (\neg p \vee q)\\
  &= (\neg \neg p \wedge \neg q) \vee (\neg p \vee q)\\
  &= (p \wedge \neg q) \vee (\neg p \vee q)\\
\end{align*}

Here, we put $A = (p \wedge \neg q)$, so $\neg A = \neg(p \wedge \neg q) = (\neg
p \vee \neg \neg q) = (\neg p \vee q)$. So we have $A \vee \neg A$ (by
subsitution) and this is a tautology.

\subsection*{(c)}

$\phi = (p \vee q \vee r) \wedge (p \vee q \vee \neg s)$ is not a tautology. For
example, the following interpretation $I$ does not satisfies the formula :

\[
  I =
  \{
  p \mapsto \bot,
  q \mapsto \bot,
  r \mapsto \bot,
  s \mapsto \top
  \}
\]

\[
  \phi^I =  \underbrace{(\bot \vee \bot \vee \bot)}_{\bot} \wedge
  \underbrace{(\bot \vee \bot \vee \neg \top)}_{\bot} = \bot \wedge \bot = \bot
\]

\section*{\texttt{5}}

\subsection*{(a)}

\[
(\neg p \wedge \neg q \wedge \neg r) \vee (\neg p \wedge q \wedge r) \vee (p \wedge q \wedge r)
\]

We can transform $(\neg p \wedge q \wedge r) \vee (p \wedge q \wedge r)$
into $(q \wedge r)$, because if $p$ is $true$, we don't look at $(\neg p \wedge
q \wedge r)$ because it would be $false$ anyway and if $p$ is $false$, we don't
look at $(p \wedge q \wedge r)$ because it would be false anyway.

\[
(\neg p \wedge \neg q \wedge \neg r) \vee (q \wedge r)
\]

We can't simplify this anymore.

\subsection*{(b)}

\[
  (\neg p \vee \neg q \vee \neg r) \wedge (\neg p \vee q \vee r) \wedge (p \vee q \vee r)
\]

We can use the same principle as before, if $p$ is $true$, $(p \vee q \vee r)$
would be $true$ and $(\neg p \vee q \vee r)$ would be transform to $(q \vee r)$,
else if $p$ is $false$, $(\neg p \vee q \vee r)$ would be $true$ and $(p \vee q
\vee r)$ would be equal to $(q \vee r)$.


\[
  (\neg p \vee \neg q \vee \neg r) \wedge (p \vee q)
\]

\subsection*{(c)}

We can use the following function $\phi$, we denote the arguments of $\phi$ by
$\lambda = \{\underbrace{a,b,\dots}_{|\lambda|}\}$) :

\[
  \phi(\lambda) = \bigwedge_{\alpha \in \lambda} \alpha = a \wedge b \wedge c
  \wedge \cdots
\]

The DNF form of such a formula can't be less than $|\lambda|$, because each
variable needs to appear in the DNF form. If one of the variable would be false,
the whole formula is going to be false.
\end{document}
