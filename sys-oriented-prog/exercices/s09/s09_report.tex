\documentclass[a4paper,11pt]{report}

\usepackage{fullpage}

\usepackage{amsmath}
\usepackage{bussproofs}
\usepackage{mathpartir}
\usepackage{prooftrees}
\usepackage{placeins}
\usepackage{color}

% Minted
\usepackage[cache=false]{minted}

\newmintinline{c}{
  fontsize=\small,
  breaklines=true
}

\newminted{c}{
  frame=single,
  framesep=2mm,
  fontsize=\scriptsize,
  mathescape
}

\newminted[clinecode]{c}{
  frame=single,
  framesep=2mm,
  fontsize=\scriptsize,
  mathescape,
  linenos
}

\newcommand*{\BBox}[1]{\draw (#1 + 0.5,0.5) -- (#1 + 1.5,0.5) -- (#1 + 1.5,-0.2)
  -- (#1 + 0.5,-0.2) -- cycle;}
\newcommand*{\SBox}[1]{}

\newcommand*{\equal}{=}

% for finite state automata
\usepackage{tikz}
\usetikzlibrary{automata,positioning}

\author{Sylvain Julmy}
\date{\today}

\setlength{\parindent}{0pt}

\begin{document}

\begin{center}
  \Large{
    System-oriented Programming\\
    Spring 2018
  }
  
  \noindent\makebox[\linewidth]{\rule{\linewidth}{0.4pt}}
  S08
  \noindent\makebox[\linewidth]{\rule{\linewidth}{0.4pt}}

  \begin{flushleft}
    Professor : Philippe Cudré-Mauroux

    Assistant : Michael Luggen
  \end{flushleft}
  
  \noindent\makebox[\linewidth]{\rule{\linewidth}{0.4pt}}

  Submitted by Sylvain Julmy
  
  \noindent\makebox[\linewidth]{\rule{\textwidth}{1pt}}
\end{center}

\section*{Exercise 1}

We use the \verb+man+ command to obtain info on the command :

\verb+man kill+ :

\begin{verbatim}
The  command kill sends the specified signal to the specified processes
or process groups.

If no signal is specified, the TERM signal is sent.  The default action
for  this  signal  is  to terminate the process.  This signal should be
used in preference to the KILL signal (number 9), since a  process  may
install  a  handler  for  the  TERM signal in order to perform clean-up
steps before terminating in an orderly fashion.  If a process does  not
terminate  after  a TERM signal has been sent, then the KILL signal may
be used; be aware that the latter signal cannot be caught, and so  does
not  give  the  target  process the opportunity to perform any clean-up
before terminating.
\end{verbatim}

\verb+kill+ is usefull when we want to send some specific signal to a process,
for testing purpose for example. The command is also used in order to terminate
process that are not more under control or when process are running even when
closing the GUI.

\verb+man ps+

\begin{verbatim}
ps displays information about a selection of the active processes.
\end{verbatim}

The \verb+ps+ command give us information about process, for example the process
tree and its hierarchy, the thread used, hand to on.

\verb+man top+

\begin{verbatim}
The  top program provides a dynamic real-time view of a running system.  It can
display system summary information$ as well as a list of processes or threads
currently being managed by the Linux kernel.  The types of system summary
information shown and the types, order and size of information displayed for
processes are all user configurable and  that  configuration  can  be  made
persistent across restarts.
\end{verbatim}

\verb+top+ is some kind of a better version of \verb+ps+, which display in real
time the informations about all the processes of the system. One can control and
send signal to any process using \verb+top+.

\section*{Exercise 2}

\subsection*{1a)}

Tt makes no sense to set permissions on yourself more restrictive than group or
other, therefore : \verb+chmod 466$+.

\subsection*{1b)}

Only root can do something about a \verb+chmod 000$+ file and the owner of the
file can change the flag.

\subsection*{2)}

\begin{itemize}
\item[a] \verb+705+
\item[b] \verb+770+
\item[c] \verb+702+
\end{itemize}

\subsection*{3)}

The permission access of a USB device is \verb+drwxr-xr-x+, it is mount on
\verb+/run/media/<user>/+, the device is mounted using a directory to access its
contains. We can have device information using $fdisk$ :

\begin{verbatim}
Disque /dev/sdb : 7.5 GiB, 8053063680 octets, 15728640 secteurs
Unités : secteur de 1 × 512 = 512 octets
Taille de secteur (logique / physique) : 512 octets / 512 octets
taille d'E/S (minimale / optimale) : 512 octets / 512 octets
Type d'étiquette de disque : dos
Identifiant de disque : 0x6f20736b
\end{verbatim}

\end{document}
