\documentclass[a4paper,11pt]{report}

\usepackage{fullpage}

\usepackage{amsmath}
\usepackage{bussproofs}
\usepackage{mathpartir}
\usepackage{prooftrees}
\usepackage{color}

% for finite state automata
\usepackage{tikz}
\usetikzlibrary{automata,positioning}

\author{Sylvain Julmy}
\date{\today}

\setlength{\parindent}{0pt}

\begin{document}

\begin{center}
  \Large{
    System-oriented Programming\\
    Spring 2018
  }
  
  \noindent\makebox[\linewidth]{\rule{\linewidth}{0.4pt}}
  S01
  \noindent\makebox[\linewidth]{\rule{\linewidth}{0.4pt}}

  \begin{flushleft}
    Professor : Philippe Cudré-Mauroux

    Assistant : Michael Luggen
  \end{flushleft}
  
  \noindent\makebox[\linewidth]{\rule{\linewidth}{0.4pt}}

  Submitted by Sylvain Julmy
  
  \noindent\makebox[\linewidth]{\rule{\textwidth}{1pt}}
\end{center}

\section*{Exercice 4}

The \verb|[OPTION]| available for \verb|wc| offers the possibility to print only
some specific information as well as displaying the help and the version of the
software.

The \verb|[FILE]| part catch the file on which to operate the software, by using
the \verb|--files0-from=F|, we can use a file that specify all the filename on
which to operate the software. If the \verb|[FILE]| is not specified or if the
\verb|--files0-from=F| is empty, then the software read from the standard input.

We can imagine two enhancements for both of that part of the command :

\begin{itemize}
\item \verb|[OPTION]| : we could use some kind of filter to count the words from
  a file, for example only counting the word which contains an ``a''.
\item \verb|[FILE]| : we could use a filter on the file of the current directory
  in order to choose among them the ones to process. For example, using a
  regular expression of the filename or on the content of each file.
\end{itemize}

\end{document}