\documentclass[a4paper,11pt]{report}

\usepackage{fullpage}

\usepackage{amsmath}
\usepackage{bussproofs}
\usepackage{mathpartir}
\usepackage{prooftrees}
\usepackage{placeins}
\usepackage{color}

% Minted
\usepackage[cache=false]{minted}

\newmintinline{c}{
  fontsize=\small,
  breaklines=true
}

\newminted{c}{
  frame=single,
  framesep=2mm,
  fontsize=\scriptsize,
  mathescape
}

\newminted[clinecode]{c}{
  frame=single,
  framesep=2mm,
  fontsize=\scriptsize,
  mathescape,
  linenos
}

\newcommand*{\BBox}[1]{\draw (#1 + 0.5,0.5) -- (#1 + 1.5,0.5) -- (#1 + 1.5,-0.2)
  -- (#1 + 0.5,-0.2) -- cycle;}
\newcommand*{\SBox}[1]{}

\newcommand*{\equal}{=}

% for finite state automata
\usepackage{tikz}
\usetikzlibrary{automata,positioning}

\author{Sylvain Julmy}
\date{\today}

\setlength{\parindent}{0pt}

\begin{document}

\begin{center}
  \Large{
    System-oriented Programming\\
    Spring 2018
  }
  
  \noindent\makebox[\linewidth]{\rule{\linewidth}{0.4pt}}
  S08
  \noindent\makebox[\linewidth]{\rule{\linewidth}{0.4pt}}

  \begin{flushleft}
    Professor : Philippe Cudré-Mauroux

    Assistant : Michael Luggen
  \end{flushleft}
  
  \noindent\makebox[\linewidth]{\rule{\linewidth}{0.4pt}}

  Submitted by Sylvain Julmy
  
  \noindent\makebox[\linewidth]{\rule{\textwidth}{1pt}}
\end{center}

\section*{Exercise 1}

\begin{ccode}
#include <stdlib.h>
#include <assert.h>
#include <string.h>
#include <ctype.h>

char* to_upper_a(char* string)
{
    for (int i = 0; i < strlen(string); i++)
    {
        string[i] = (char) toupper(string[i]);
    }
    return string;
}

char* to_upper_b(char* string)
{
    char* start = string;
    while (*string != '\0')
    {
        if (*string >= 'a' && *string <= 'z')
            *string += ('A' - 'a');
        string++;
    }
    return start;
}

int main(void)
{
    char* a = malloc(4 * sizeof(char));
    strcpy(a, "asd");
    assert(strcmp(to_upper_a(a), "ASD") == 0);
    strcpy(a, "asd");
    assert(strcmp(to_upper_b(a), "ASD") == 0);
    return EXIT_SUCCESS;
}
\end{ccode}

\newpage

\section*{Exercise 2}

\subsection*{a)}
\begin{ccode}
#include <stdlib.h>
#include <unistd.h>
#include <stdio.h>
#include <stdbool.h>

int main(void)
{
    while(true)
    {
        fprintf(stdout,"Hey !\n");
        sleep(1);
        fprintf(stderr,"Ho !\n");
        sleep(1);
    }
    return EXIT_SUCCESS;
}
\end{ccode}

\subsection*{b)}

The following bash command generate data for \verb+stdout+ and \verb+stdin+,
then both of the stream are redirected to their corresponding file.

\begin{verbatim}
{echo "output" ; echo "error" >&2;} 1>> hey.txt 2>> ho.tx
\end{verbatim}

\end{document}
