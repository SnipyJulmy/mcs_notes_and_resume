\documentclass[a4paper,11pt]{report}

\usepackage{fullpage}

\usepackage{amsmath}
\usepackage{bussproofs}
\usepackage{mathpartir}
\usepackage{prooftrees}
\usepackage{placeins}
\usepackage{color}

% Minted
\usepackage[cache=false]{minted}

\newmintinline{c}{
  fontsize=\small,
  breaklines=true
}

\newminted{c}{
  frame=single,
  framesep=2mm,
  fontsize=\scriptsize,
  mathescape
}

\newminted[clinecode]{c}{
  frame=single,
  framesep=2mm,
  fontsize=\scriptsize,
  mathescape,
  linenos
}

\newcommand*{\BBox}[1]{\draw (#1 + 0.5,0.5) -- (#1 + 1.5,0.5) -- (#1 + 1.5,-0.2)
  -- (#1 + 0.5,-0.2) -- cycle;}
\newcommand*{\SBox}[1]{}

\newcommand*{\equal}{=}

% for finite state automata
\usepackage{tikz}
\usetikzlibrary{automata,positioning}

\author{Sylvain Julmy}
\date{\today}

\setlength{\parindent}{0pt}

\begin{document}

\begin{center}
  \Large{
    System-oriented Programming\\
    Spring 2018
  }
  
  \noindent\makebox[\linewidth]{\rule{\linewidth}{0.4pt}}
  S10
  \noindent\makebox[\linewidth]{\rule{\linewidth}{0.4pt}}

  \begin{flushleft}
    Professor : Philippe Cudré-Mauroux

    Assistant : Michael Luggen
  \end{flushleft}
  
  \noindent\makebox[\linewidth]{\rule{\linewidth}{0.4pt}}

  Submitted by Sylvain Julmy
  
  \noindent\makebox[\linewidth]{\rule{\textwidth}{1pt}}
\end{center}

\section*{Exercise 1}

What is the meaning of the following four expressions :
\begin{itemize}
\item \verb+.a+ : any single character followed by an "a".
\item \verb+\.a+ : a dot followed by an "a".
\item \verb+[a]*+ : zero or more "a" consecutively.
\item \verb+[^a]*+ : non-"a" character, zero or more times, consecutively.
\end{itemize}

What is the meaning of the following line commands :
\begin{enumerate}
\item print all line (and line number) from ``hello.c'' which contains a left bracket or a right
  bracket.
\item print all line (and line number) from ``hello.c'' which contains a character.
\item same as \verb+cat hello.c+ but with the ine number, print the whole file.
\item print the number of line from ``hello.c'' that end with a ``;''.
\item print the number of line from ``hello.c'' that don't end with a ``;''.
\item print all file name in the current directory (and sub-directory,
  recursively) where its name contains any character followed by a ``c''.
\item print all file name in the current directory (and sub-directory,
  recursively) where its name contains a dot followed by a ``c''.
\item print all file name in the current directory (and sub-directory,
  recursively) where its name end by a dot and a ``c''.
\item print all the file name which contains a line with ``my\_stack.h'', for
  example showing all the file that include ``my\_stack.h''.
\end{enumerate}

\paragraph{"Hello World" : } \verb+egrep -rl 'Hello world'+

\paragraph{"Pattern" : } \verb+egrep -rli 'Pattern'+

\paragraph{"Hello world" in .c files :} \verb+egrep -l 'Hello World' `find . -printf "%p\n" | egrep '\.c'`+

\begin{itemize}
\item \verb+("University" AND "Fribourg")+
\item \verb+("University" OR "Fribourg")+
\item \verb+"University Fribourg"+
\item \verb+("University Fribourg" OR "University Bern")+
\item \verb+("University Fribourg" OR "University Bern") NOT "University Geneva"+
\end{itemize}


Output of \verb+echo $USER "$USER" '$USER'+ :

\begin{verbatim}
snipy snipy $USER
\end{verbatim}

The \verb+$USER+ in single quote is not interpreted by the shell.


\section*{Additional features}

\begin{itemize}
\item Deletion of a triplet (delete the matching triplet).
\item Implementation of a static index, which needs to be rebuild if the database change.
\item Implementation of a dynamic index, which index the data on each insert and delte.
\item The possibility to precise the index we want to use.
\item Implementation of a simple shell,in order to use the database in (kind of)
  real-time, without having to hard code it.
\item I have much more idea, but it is limited to $5$ here :).
\end{itemize}

\end{document}
