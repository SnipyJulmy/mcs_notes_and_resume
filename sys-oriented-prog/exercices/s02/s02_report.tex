\documentclass[a4paper,11pt]{report}

\usepackage{fullpage}

\usepackage{amsmath}
\usepackage{bussproofs}
\usepackage{mathpartir}
\usepackage{prooftrees}
\usepackage{color}

% Minted
\usepackage[cache=false]{minted}

\newmintinline{c}{
  fontsize=\small,
  breaklines=true
}

% for finite state automata
\usepackage{tikz}
\usetikzlibrary{automata,positioning}

\author{Sylvain Julmy}
\date{\today}

\setlength{\parindent}{0pt}

\begin{document}

\begin{center}
  \Large{
    System-oriented Programming\\
    Spring 2018
  }
  
  \noindent\makebox[\linewidth]{\rule{\linewidth}{0.4pt}}
  S02
  \noindent\makebox[\linewidth]{\rule{\linewidth}{0.4pt}}

  \begin{flushleft}
    Professor : Philippe Cudré-Mauroux

    Assistant : Michael Luggen
  \end{flushleft}
  
  \noindent\makebox[\linewidth]{\rule{\linewidth}{0.4pt}}

  Submitted by Sylvain Julmy
  
  \noindent\makebox[\linewidth]{\rule{\textwidth}{1pt}}
\end{center}

\section*{Exercice 4}

\begin{itemize}
\item \verb+./wcount+
  
  This command would just invoke the program \verb+./wcount+, which would read
  the standard input (keyboard) in order to obtain something to process. We just
  have to type something, including <ENTER> char and then we send the EOF char
  by <CTRL+D>. The ouput is printed on the standard output.
  
\item \verb+./wcount < wcount.c+

  This command invoke the program \verb+./wcount+, but this time the standard
  input of the program is the file named \verb+wcount+. The ouput is printed
  on the standard output.
  
\item \verb+./wcount < wcount > test+

  This command invoke the program \verb+./wcount+, but this time the standard
  input of the program is the file named \verb+wcount.c+. The ouput is
  redirected to a file named \verb+test+.
  
\item \verb+cat wcount.c | ./wcount+

  The \verb+cat+ command will print the content of the file \verb+wcount.c+ on
  the standard output, but using the \verb+|+ operator, the standard output is
  redirected into the standard input of the \verb+./wcount+ program.
  
\item \verb+grep { wcount.c+

    The \verb+grep+ program would filter and print the content of the file
    \verb+wcount.c+. This command would print all the line of \verb+wcount.c+
    that contains a \verb+{+ symbol.
    
\item \verb+grep { wcount.c | ./wcount+

    It is the same as before except that the standard output of the \verb+grep+
    command is redirected to the standard input of the \verb+./wcount+ command.
    So the whole command would count the number of word, letter, ... of all the
    line of \verb+wcount.c+ that contains a \verb+{+ symbol.
    
\item \verb+grep -l { * | ./wcount+

    The \verb+-l+ option of \verb+grep+ is suppressing the normal output of the
    command and instead print the name of all the file from which the output
    would normally be printed. The \verb+*+ symbol represent all the file of the
    current directory. So the \verb+grep+ command would output all the filename
    of the current directory that contains the \verb+{+ symbol.
    
\end{itemize}

\section*{Exercice 5}

\begin{itemize}
\item \cinline|printf("%c %i\n", c, c);|

  We don't need any type casting here, \cinline+c+ is a variable of type
  \cinline|char| which is automaticaly seen as an integer when using the
  \verb+%i+ formatter. $65$ is the ascii value of ``A''.

  \paragraph{Output : } \verb+A 65+
  
\item \cinline|printf("%c %i\n", i, i);|

  We don't need any type casting here, \cinline+i+ is a variable of type
  \cinline+int+ which is automaticaly downgraded into a \cinline+char+ by simply
  ``cutting'' the extra-part of the \cinline+integer+ and put it into a
  \cinline+char+.
  
  \paragraph{Output : } \verb+A 65+
  
\item \cinline|printf("%f %i\n", pi, (int)pi);|

  This time we need an explicit type casting from a \cinline+float+ into an
  \cinline+integer+ because the representation, in memory, of floating point
  number and integer number is different and the compiler has to to some
  additionnal work in order to correctly transform the float value (by rounding
  the value from $3.14$ to $3$) into an integer. Without the explicit type
  casting, the number would have been read directly (the significand, the base
  and the exponant would'nt have been read separetely).
  
  \paragraph{Output : } \verb+3.140000 3+

\end{itemize}

\section*{Exercice 6}

First, the decimal value of the ascii character \verb+@+ is $64$. The output of
the program is the following

\begin{verbatim}
  @ 64 100 40
  @ 64 100 40
  @ 64 100 40
\end{verbatim}

Three times the line printed is the same because :

\begin{itemize}
\item the decimal value of \verb+@+ is 64,
\item the decimal value of \verb+\100+ is 64,
\item the decimal value of \verb+\x40+ is 64.
\end{itemize}

\section*{Exercice 7}

The output of the program is the following :

\begin{verbatim}
  1 0
  0 1 2
\end{verbatim}

In C, each enum fields is represented by an integer value from $0$ to $n-1$
where $n$ is the number of fields in the enum. Then, \verb+TRUE+ as the value
$1$ and \verb+FALSE+ as the value $0$, that's why the first line is \verb+1 0+.

For the second line, $C_1$ is initialize to \verb+RED+, which is the first field
of the \verb+color_tag+ enum, so the value of \verb+RED+ is $0$. $C_2$ is
initialize to $C_1 + 1$ which is $0 + 1 = 1$. Finally $C_3$ is initialize to
\verb+BLUE+ which is the third field of the \verb+color_tag+ enum, so the value
of \verb+BLUE+ is $3$. Then outputing $C_1$, $C_2$ and $C_3$ would be \verb+0 1 2+.

\section*{Exercice 8}

We denote by $n$ any integer not equal to $0$.

\subsection*{\cinline+p || !q+}

\begin{center}
  \begin{tabular}{@{ }c@{ }@{ }c | c@{ }@{ }c@{ }@{ }c@{ }@{ }c@{ }@{ }c@{ }@{ }c}
    p & q &  & p & \verb+||+ & \verb+!+ & q & \\
    \hline 
    $n$ & $n$ &  & $n$ & \textcolor{red}{$1$} & $0$ & $n$ & \\
    $n$ & $0$ &  & $n$ & \textcolor{red}{$1$} & $1$ & $0$ & \\
    $0$ & $n$ &  & $0$ & \textcolor{red}{$0$} & $0$ & $n$ & \\
    $0$ & $0$ &  & $0$ & \textcolor{red}{$1$} & $1$ & $0$ & \\
  \end{tabular}
\end{center}

\subsection*{\cinline+p && (p == q)+}

\begin{center}
  \begin{tabular}{@{ }c@{ }@{ }c | c@{ }@{ }c@{ }@{ }c@{ }@{}c@{}@{ }c@{ }@{ }c@{ }@{ }c@{ }@{}c@{}@{ }c}
    p & q &  & p & \verb+&&+ & ( & p & \verb+==+ & q & ) & \\
    \hline 
    $n$ & $n$ &  & $n$ & \textcolor{red}{$1$} &  & $n$ & $1$ & $n$ &  & \\
    $n$ & $0$ &  & $n$ & \textcolor{red}{$0$} &  & $n$ & $0$ & $0$ &  & \\
    $0$ & $n$ &  & $0$ & \textcolor{red}{$0$} &  & $0$ & $0$ & $n$ &  & \\
    $0$ & $0$ &  & $0$ & \textcolor{red}{$0$} &  & $0$ & $1$ & $0$ &  & \\
  \end{tabular}
\end{center}

\subsection*{\cinline+p && (p = q) || (p = !q)+}

First we add parentheses to clearly show the order of evaluation :

\cinline+( p && (p = q) ) || (p = !q)+

\begin{center}
  \begin{tabular}{@{ }c@{ }@{ }c | c@{ }@{}c@{}@{ }c@{ }@{ }c@{ }@{}c@{}@{ }c@{ }@{ }c@{ }@{ }c@{ }@{}c@{}@{}c@{}@{ }c@{ }@{}c@{}@{ }c@{ }@{ }c@{ }@{ }c@{ }@{ }c@{ }@{}c@{}@{ }c}
    p & q &  & ( & p & \verb+&&+ & ( & p & \verb+=+ & q & ) & ) & \verb+||+ & ( & p & \verb+=+ & \verb+!+ & q & ) & \\
    \hline 
    $n$ & $n$ &  &  & $n$ & $1$ &  & $n$ & $n$ & $n$ &  &  & \textcolor{red}{$1$} &  & $n$ & $0$ & $0$ & $n$ &  & \\
    $n$ & $0$ &  &  & $n$ & $0$ &  & $n$ & $0$ & $0$ &  &  & \textcolor{red}{$1$} &  & $0$ & $1$ & $1$ & $0$ &  & \\
    $0$ & $n$ &  &  & $0$ & $0$ &  & $0$ & $n$ & $n$ &  &  & \textcolor{red}{$0$} &  & $n$ & $0$ & $0$ & $n$ &  & \\
    $0$ & $0$ &  &  & $0$ & $0$ &  & $0$ & $0$ & $0$ &  &  & \textcolor{red}{$1$} &  & $0$ & $1$ & $1$ & $0$ &  & \\
  \end{tabular}
\end{center}

\section*{Exercice 9}

\section*{Exercice 10}

\section*{Exercice 11}

\section*{Exercice 12}

\section*{Exercice 13}
    
\end{document}
